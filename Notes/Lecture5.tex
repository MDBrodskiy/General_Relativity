%%%%%%%%%%%%%%%%%%%%%%%%%%%%%%%%%%%%%%%%%%%%%%%%%%%%%%%%%%%%%%%%%%%%%%%%%%%%%%%%%%%%%%%%%%%%%%%%%%%%%%%%%%%%%%%%%%%%%%%%%%%%%%%%%%%%%%%%%%%%%%%%%%%%%%%%%%%%%%%%%%%
% Written By Michael Brodskiy
% Class: General Relativity and Cosmology
% Professor: J. Blazek
%%%%%%%%%%%%%%%%%%%%%%%%%%%%%%%%%%%%%%%%%%%%%%%%%%%%%%%%%%%%%%%%%%%%%%%%%%%%%%%%%%%%%%%%%%%%%%%%%%%%%%%%%%%%%%%%%%%%%%%%%%%%%%%%%%%%%%%%%%%%%%%%%%%%%%%%%%%%%%%%%%%

\include{Includes.tex}

\title{Lecture 5 — Einstein's Equations and Schwarzschild}
\date{\today}
\author{Michael Brodskiy\\ \small Professor: J. Blazek}

\begin{document}

\maketitle

\begin{itemize}

  \item We may begin with Einstein's equations

    $$G_{\mu\nu}=8\pi G T_{\mu\nu}$$

    \begin{itemize}

      \item Note that the tensors form second order non-linear differential equations

    \end{itemize}

  \item Symmetric Tensor ($n=4$ dimensions)

    \begin{itemize}

      \item $\frac{n^2-n}{n}+n=\frac{n^2+n}{2}=10$ degrees of freedom

      \item General relativity sees diffeomorphism invariance, so four of the degrees of freedom are removed (since it isn't $x^{\mu}\to x^{\mu'}$)

      \item Solving these differential equations is extremely complex, so we will make some assumptions to simplify analysis:

        \begin{itemize}

          \item Boundary conditions and initial conditions

          \item Limits

          \item Simplify through symmetry

        \end{itemize}

    \end{itemize}

  \item Symmetric General Relativity

    \begin{itemize}

      \item Spherical symmetry and static

      \item Homogenous and isotropic: FLRW universe/cosmology

      \item $T_{\mu\nu}=0$, small perturbations are gravitational waves

      \item In Newtonian mechanics, with three masses $M_1$, $M_2$, and much smaller $M_3$, we may write: $\Phi_{M_3}=\Phi_{M_1}+\Phi_{M_2}$

        \begin{itemize}

          \item In General Relativity, $g_{\mu\nu}$ depends on $M_1$ and $M_2$, as well as the binding energy between

        \end{itemize}

    \end{itemize}

  \item Schwarzschild

    \begin{itemize}
        
      \item Only vacuum solution with spherical symmetry

      \item We assume a spherical system that is static

      \item We use Minkowski space, with spherical coordinates:

        $$ds^2=-dt^2+dx^2+dy^2+dz^2=-dt^2+dr^2+r^2\underbrace{(d\theta^2+\sin^2(\theta)\,d\theta^2)}_{d\Omega^2}$$

        \begin{itemize}

          \item We may rescale this with functions of $r$:

            $$ds^2=-A(r)\,dt^2+B(r)\,dr^2+C(r)r^2\,d\Omega^2$$

          \item Furthermore, we define $r\to\sqrt{C(r)}$

            $$ds^2=-A(r)\,dt^2+B(r)\,dr^2+r^2\,d\Omega^2$$
            $$ds^2=-e^{2\alpha(r)}\,dt^2+e^{\beta(r)}\,dr^2+r^2\,d\Omega^2$$

          \item For the diagonal metric, we way find $\Gamma^{\phi}_{r\phi}=(1/r)$, then continuing Christoffel calculations, using Riemann, and then contracting to Ricci, we find:

            $$R_{tt}=e^{2(\alpha-\beta)}\left[ \partial_r^{2}\alpha+(\partial_r\alpha)^2-\partial_r\alpha\partial_r\beta+\frac{2}{r}\partial_r\alpha \right]$$
            $$R_{rr}=-\partial^2_r\alpha-(\partial_r\alpha)^2+\partial_r\alpha\partial_r\beta+\frac{2}{r}\partial_r\beta$$
            $$R_{\theta\theta}=e^{-2\beta}\left[ r(\partial_r\beta-\partial_r\alpha)-1 \right]+1$$
            $$R_{\phi\phi}=\sin^2(\theta)R_{\theta\theta}$$

          \item We want a $T_{\mu\nu}=0$ solution, which implies $G_{\mu\nu}=0$, which then implies $R_{\mu\nu}=0$

            \begin{itemize}

              \item This is known as ``Ricci Flat'' (not really flat)

            \end{itemize}

          \item We move terms around to find:

            $$e^{2(\beta-\alpha)}R_{tt}+R_{rr}=0$$
            $$\frac{2}{r}\left( \partial_r\alpha+\partial_r\beta \right)=0$$
            $$\alpha=-\beta$$

          \item Taking:

            $$R_{\theta\theta}=0\quad\text{ and }\quad -e^{2\alpha}\left[ 2r\partial_r\alpha+1 \right]+1=0$$

          \item We get:

            $$e^{2\alpha}\left[ 2r\partial_r\alpha+1 \right]=\partial_r(re^{2\alpha})$$

          \item We define $A(r)=e^{2\alpha}$ and $y(r)=rA(r)$, which gives:

            $$y=r+C\Rightarrow A(r)=1+\frac{c}{r}$$
            $$A(r)=1-\frac{R_s}{r}$$

          \item Where $R_s$ is the Schwarzschild radius, which allows us to write:
            $$ds^2=-\left( 1-\frac{R_s}{r} \right)\,dt^2+\left( 1-\frac{R_s}{r} \right)^{-1}\,dr^2+r^2\,d\Omega$$

        \end{itemize}

    \end{itemize}

  \item Black Holes

    \begin{itemize}

      \item Using Newtonian $g_{tt}=-(1+2\Phi)$, we get:

        $$R_s=-2\Phi$$

      \item With a point mass:

        $$\Phi=-\frac{GM}{r}$$

      \item Thus, $R_s=2GM$

      \item Schwarzschild Properties:

        \begin{enumerate}

          \item $M\to0,\quad g_{MV}\to\eta_{M\nu}$

          \item $r\to\infty,\quad g_{M\nu}\to\eta_{M\nu}$

          \item $r=0,\quad \dfrac{R_s}{r}\to\infty$

          \item $r=R_s,\quad \left( 1-\frac{R_s}{r} \right)^{-1}\to\infty$

        \end{enumerate}

      \item For Black Holes, light cones are deformed by null geodesics. We may derive:

        $$\frac{dt}{dr}=\pm\left( 1-\frac{2GM}{r} \right)^{-1}$$

        \begin{itemize}

          \item As $r\to\infty$ back to $45^{\circ}$

        \end{itemize}

    \end{itemize}

  \item Anthropic Principle

\end{itemize}

\end{document}

