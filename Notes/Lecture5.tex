%%%%%%%%%%%%%%%%%%%%%%%%%%%%%%%%%%%%%%%%%%%%%%%%%%%%%%%%%%%%%%%%%%%%%%%%%%%%%%%%%%%%%%%%%%%%%%%%%%%%%%%%%%%%%%%%%%%%%%%%%%%%%%%%%%%%%%%%%%%%%%%%%%%%%%%%%%%%%%%%%%%
% Written By Michael Brodskiy
% Class: General Relativity and Cosmology
% Professor: J. Blazek
%%%%%%%%%%%%%%%%%%%%%%%%%%%%%%%%%%%%%%%%%%%%%%%%%%%%%%%%%%%%%%%%%%%%%%%%%%%%%%%%%%%%%%%%%%%%%%%%%%%%%%%%%%%%%%%%%%%%%%%%%%%%%%%%%%%%%%%%%%%%%%%%%%%%%%%%%%%%%%%%%%%

\include{Includes.tex}

\title{Lecture 5 — Einstein's Equations and Schwarzschild}
\date{\today}
\author{Michael Brodskiy\\ \small Professor: J. Blazek}

\begin{document}

\maketitle

\begin{itemize}

  \item We may begin with Einstein's equations

    $$G_{\mu\nu}=8\pi G T_{\mu\nu}$$

    \begin{itemize}

      \item Note that the tensors form second order non-linear differential equations

    \end{itemize}

  \item Symmetric Tensor ($n=4$ dimensions)

    \begin{itemize}

      \item $\frac{n^2-n}{n}+n=\frac{n^2+n}{2}=10$ degrees of freedom

      \item General relativity sees diffeomorphism invariance, so four of the degrees of freedom are removed (since it isn't $x^{\mu}\to x^{\mu'}$)

      \item Solving these differential equations is extremely complex, so we will make some assumptions to simplify analysis:

        \begin{itemize}

          \item Boundary conditions and initial conditions

          \item Limits

          \item Simplify through symmetry

        \end{itemize}

    \end{itemize}

  \item Symmetric General Relativity

    \begin{itemize}

      \item Spherical symmetry and static

      \item Homogenous and isotropic: FLRW universe/cosmology

      \item $T_{\mu\nu}=0$, small perturbations are gravitational waves

      \item In Newtonian mechanics, with three masses $M_1$, $M_2$, and much smaller $M_3$, we may write: $\Phi_{M_3}=\Phi_{M_1}+\Phi_{M_2}$

        \begin{itemize}

          \item In General Relativity, $g_{\mu\nu}$ depends on $M_1$ and $M_2$, as well as the binding energy between

        \end{itemize}

    \end{itemize}

  \item Schwarzschild

    \begin{itemize}
        
      \item Only vacuum solution with spherical symmetry

      \item We assume a spherical system that is static

      \item We use Minkowski space, with spherical coordinates:

        $$ds^2=-dt^2+dx^2+dy^2+dz^2=-dt^2+dr^2+r^2\underbrace{(d\theta^2+\sin^2(\theta)\,d\theta^2)}_{d\Omega^2}$$

        \begin{itemize}

          \item We may rescale this with functions of $r$:

            $$ds^2=-A(r)\,dt^2+B(r)\,dr^2+C(r)r^2\,d\Omega^2$$

          \item Furthermore, we define $r\to\sqrt{C(r)}$

            $$ds^2=-A(r)\,dt^2+B(r)\,dr^2+r^2\,d\Omega^2$$
            $$ds^2=-e^{2\alpha(r)}\,dt^2+e^{\beta(r)}\,dr^2+r^2\,d\Omega^2$$

          \item For the diagonal metric, we way find $\Gamma^{\phi}_{r\phi}=(1/r)$, then continuing Christoffel calculations, using Riemann, and then contracting to Ricci, we find:

            $$R_{tt}=e^{2(\alpha-\beta)}\left[ \partial_r^{2}\alpha+(\partial_r\alpha)^2-\partial_r\alpha\partial_r\beta+\frac{2}{r}\partial_r\alpha \right]$$
            $$R_{rr}=-\partial^2_r\alpha-(\partial_r\alpha)^2+\partial_r\alpha\partial_r\beta+\frac{2}{r}\partial_r\beta$$
            $$R_{\theta\theta}=e^{-2\beta}\left[ r(\partial_r\beta-\partial_r\alpha)-1 \right]+1$$
            $$R_{\phi\phi}=\sin^2(\theta)R_{\theta\theta}$$

          \item We want a $T_{\mu\nu}=0$ solution, which implies $G_{\mu\nu}=0$, which then implies $R_{\mu\nu}=0$

            \begin{itemize}

              \item This is known as ``Ricci Flat'' (not really flat)

            \end{itemize}

          \item We move terms around to find:

            $$e^{2(\beta-\alpha)}R_{tt}+R_{rr}=0$$
            $$\frac{2}{r}\left( \partial_r\alpha+\partial_r\beta \right)=0$$
            $$\alpha=-\beta$$

          \item Taking:

            $$R_{\theta\theta}=0\quad\text{ and }\quad -e^{2\alpha}\left[ 2r\partial_r\alpha+1 \right]+1=0$$

          \item We get:

            $$e^{2\alpha}\left[ 2r\partial_r\alpha+1 \right]=\partial_r(re^{2\alpha})$$

          \item We define $A(r)=e^{2\alpha}$ and $y(r)=rA(r)$, which gives:

            $$y=r+C\Rightarrow A(r)=1+\frac{c}{r}$$
            $$A(r)=1-\frac{R_s}{r}$$

          \item Where $R_s$ is the Schwarzschild radius, which allows us to write:
            $$ds^2=-\left( 1-\frac{R_s}{r} \right)\,dt^2+\left( 1-\frac{R_s}{r} \right)^{-1}\,dr^2+r^2\,d\Omega$$

        \end{itemize}

    \end{itemize}

  \item Black Holes

    \begin{itemize}

      \item Using Newtonian $g_{tt}=-(1+2\Phi)$, we get:

        $$R_s=-2\Phi$$

      \item With a point mass:

        $$\Phi=-\frac{GM}{r}$$

      \item Thus, $R_s=2GM$

      \item Schwarzschild Properties:

        \begin{enumerate}

          \item $M\to0,\quad g_{MV}\to\eta_{M\nu}$

          \item $r\to\infty,\quad g_{M\nu}\to\eta_{M\nu}$

          \item $r=0,\quad \dfrac{R_s}{r}\to\infty$

          \item $r=R_s,\quad \left( 1-\frac{R_s}{r} \right)^{-1}\to\infty$

        \end{enumerate}

      \item For Black Holes, light cones are deformed by null geodesics. We may derive:

        $$\frac{dt}{dr}=\pm\left( 1-\frac{2GM}{r} \right)^{-1}$$

        \begin{itemize}

          \item As $r\to\infty$ back to $45^{\circ}$

        \end{itemize}

    \end{itemize}

  \item Anthropic Principle

    \begin{itemize}

      \item Observations made about the universe are implictly biased as a result of the fact that observations can only be made where the possibility of intelligence life exists

    \end{itemize}

  \item Gravitational Bending of Spacetime (Back-Reaction Term):

    $$\frac{1}{2}\left( \frac{dr}{d\lambda} \right)^2+\frac{L^2}{2r^2}-\frac{\epsilon GM}{r}\underbrace{-\frac{GML^2}{r^3}}_{\text{New Term}}+\frac{1}{2}\epsilon=\frac{1}{2}E^3$$

    \begin{itemize}

      \item Defining $V(r)$:

        $$\frac{L^2}{2r^2}-\frac{\epsilon GM}{r}-\frac{GML^2}{r^3}+\frac{1}{2}\epsilon$$

        \begin{itemize}

          \item The first term is known as the angular momentum ``barrier''

        \end{itemize}

      \item We may write:

        $$\frac{1}{2}\left( \frac{dr}{d\lambda} \right)^2+V(r)=\epsilon$$

      \item Which lets us determine that, for circular orbit, $V'(r)=0$, and for a stable circular orbit $V''(r)=0$

      \item Using Newtonian mechanics, we may see:

        $$\frac{d}{dr}\left( \frac{L^2}{2r^2}-\frac{GM}{r} \right)=-\frac{L^2}{r^3}+\frac{GM}{r^2}$$

        \begin{itemize}

          \item Rearranging terms, we come to the familiar formula:

            $$v=\sqrt{\frac{GM}{r}}$$

        \end{itemize}

      \item We may perform similar calculations with General Relativity, but the new term adds a twist:

        \begin{itemize}

          \item 2 solutions instead of 1 for massive particles

          \item Not always stable

        \end{itemize}

      \item For a massless particle ($v=c$), $\epsilon=0$:

        $$V(r)=\frac{L^2}{2r^2}-\frac{GML^2}{r^3}$$
        $$V'(r)=0\Rightarrow-\frac{L^2}{r^3}+\frac{3GML^2}{r^4}=0\tor=3GM$$
        $$V''(r)=\frac{3L^2}{r^4}-\frac{12GML^2}{r^5}$$

        \begin{itemize}

          \item Always negative! Always unstable!

        \end{itemize}

      \item For massive particles, $\epsilon=1$

        $$V(r)=\frac{1}{2}-\frac{GM}{r}+\frac{L^2}{2r^2}-\frac{GML^2}{r^3}$$
        $$V'(r)=0\Rightarrow GMr^2+L^2r+3GML^2=0\to r_L=\frac{L^2\pm\sqrt{L^4-4(GM)(3GML^2)}}{2GM}$$

        \begin{itemize}

          \item We may observe:

            \begin{enumerate}

              \item For large $L$: $\dfrac{L^2}{GM}$ is stable (goes to Newtonian), and $3GM$ is unstable (the massless case)

              \item For small $L$: $L=\sqrt{12}GM$ provides smallest circular orbit, $r_c=L^2/2GM=6GM=3R_w$; No circular orbits for smaller $L$ (particle goes to $r=0$ (I.S.C.O))

            \end{enumerate}

        \end{itemize}

      \item This new $1/r^3$ term reflects the non-linear nature of general relativity (back-reaction), which is important for small $r$

      \item For eccentricity $e$, we may write:

        $$x_1=\frac{3G^2M^2}{L^2}\left[ \left( 1+\frac{1}{2}e^2 \right)+e\phi\sin(\phi)-\frac{1}{6}e^2\cos(2\phi) \right]$$

        \begin{itemize}

          \item Term 1: The overall shift to larger $x$ (smaller $r$)

          \item Term 2: Accumulates $\phi:0,2\pi,4\pi,6\pi\cdots$

          \item Term 3: Oscillations around elliptical orbit

          \item Note that terms 1 and 3 are small

        \end{itemize}

      \item Thus, we have:

        $$x=1+e\cos\left[ (1-\alpha)\phi \right]\text{ with }\alpha=\frac{3G^2M^2}{L^2}$$

        \begin{itemize}

          \item With a period slightly less than $2\pi$

          \item For each orbit, $\Delta \phi=2\pi\alpha=6\pi\frac{G^2M^2}{L^2}$

        \end{itemize}

      \item Precession of Mercury:

        $$\Delta\phi_{mer}=5.01\cdot10^{-7}\left[ \frac{\text{rad}}{\text{orbit}} \right]=.103''\left[ \frac{1}{\text{orbit}} \right]$$

    \end{itemize}

  \item Gravitational Redshift

    \begin{itemize}

      \item Consider an observer at rest $U^i=0$:

        $$U_{\mu}U^{\mu}=-1\text{ for a time-like observer}$$
        $$\Rightarrow U^{o}=\left( 1-\frac{2GM}{r} \right)^{-\frac{1}{2}}$$

      \item For a photon traveling on a null geodesic, observer ($U^{\mu}$) sees energy: $-\dfrac{dx^{\nu}}{d\lambda}U^{\mu}g_{\mu\nu}=-p_{\mu}U^{\mu}$

        \begin{itemize}

          \item This defines $\lambda$ for a null geodesic

          \item Observed frequency:

            $$f=\left( 1-\frac{2GM}{r} \right)^{\frac{1}{2}}\frac{dt}{d\lambda}$$

          \item Recall the conserved quantities for geodesics:

            $$E=\left( 1-\frac{2GM}{r} \right)\frac{dt}{d\lambda}$$
            $$f\left( 1-\frac{2GM}{r} \right)^{\frac{1}{2}}=\text{ a constant}$$

          \item When emitted at $r_1$ and observed at $r_2$, we see:

            $$\frac{f_2}{f_1}=\left( \frac{1-\frac{2GM}{r_1}}{1-\frac{2GM}{r_2}} \right)^{\frac{1}{2}}$$

              \begin{itemize}

                \item Redshift for $r_1<r_2\Rightarrow f_2<f_1$

                \item Blueshift for $r_1>r_2\Rightarrow f_2>f_1$

              \end{itemize}

        \end{itemize}

      \item Eddington-Finkelstein Coordinates:

        \begin{itemize}

          \item Using $v=t+r^*$ and $u=t-r^*$ (where $v$ is constant ingoing null geodesics, and $u$ is constant outgoing null geodesics), we get:

            $$ds^2=-\left( 1-\frac{2GM}{r} \right)\,dv^2+(\,dv\,dr+\,dr\,dv)+r^2\,d\Omega^2$$

          \item This gives: $\text{det}(g_{\mu\nu})=-r^4\sin^2(\theta)$

          \item Null geodesics:

            $$ds^2=0=-\left( 1-\frac{2GM}{r} \right)\,dv^2+2\,dv\,dr$$
            $$dv=0\text{ or } dv\left( 1-\frac{2GM}{r} \right)=2\,dr$$
            $$\frac{dv}{dr}=0\text{ or } \frac{dv}{dr}=2\left( 1-\frac{2GM}{r} \right)^{-1}$$

        \end{itemize}

      \item Event Horizon — Surface past which particles can never escape to infinity

        \begin{itemize}

          \item Global rather than local feature of spacetime

        \end{itemize}

    \end{itemize}

  \item Kruskal Coordinates Introduce Distinct Regions of Black Holes:

    \begin{itemize}

      \item Region I: Our Universe

      \item Region II: Black Hole

      \item Region III: White Hole

      \item Region IV: Separate Universe (connected via wormhole, closes before anything can pass through)

    \end{itemize}

  \item Physical Black Holes

    \begin{itemize}

      \item This assumes perfect spherical symmetry over an infinite time

      \item Not the real universe; we lack this symmetry, so Einstein did not believe black holes actually existed

      \item Black holes do form at some finite time, but they do form!

    \end{itemize}

  \item Observational Evidence

    \begin{itemize}

      \item Black holes as ``central engines''

        \begin{itemize}

          \item Active galactic nuclei

          \item Quasars, blazars, etc, \ldots

        \end{itemize}

      \item Our galactic center

        \begin{itemize}

          \item 2020 Nobel Prize: Genzel and Ghez

          \item Mapping out star orbits near galaxy's senter (Sag A$^*$) with $M_{BH}\approx 4\cdot10^6[\text{M}_{\odot}]$

        \end{itemize}

      \item Nearby galaxy center (M87) with $M_{BH}\approx 6\cdot10^9[\text{M}_{\odot}]$

        \begin{itemize}

          \item Image using combined telescopes (mm/radio)

          \item ``Event Horizon Telescope''

        \end{itemize}

    \end{itemize}

  \item Theoretical Evidence

    \begin{itemize}

      \item Penrose and Hawking: Singularity theorems (2020 Nobel Prize for Penrose)

        \begin{itemize}

          \item Singularities can and will form once collapse reaches a certain point

          \item Trapped surfaces

        \end{itemize}

      \item No physical pressure able to stop collapse

    \end{itemize}

  \item Black Hole Theorems

    \begin{itemize}

      \item No-Hair Theorem: Stationary (Stable) Black Holes can be described with only:

        \begin{enumerate}

          \item Mass (Schwarzschild)

          \item Charge (Reissner-Nordstr\"om)

          \item Spin (Kerr)

        \end{enumerate}

      \item Cosmic Censorship Conjecture: Naked singularities (no horizon) do not form generically in gravitational collapse (e.g. no ``white holes'')

      \item Hawking's Area Theorem: The are of an event horizon will not decrease

        \begin{itemize}

          \item Quantum Mechanics complicates this picture: Hawking Radiation

          \item Information loss paradox: quantum gravity?

        \end{itemize}

    \end{itemize}

\end{itemize}

\end{document}

