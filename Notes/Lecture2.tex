%%%%%%%%%%%%%%%%%%%%%%%%%%%%%%%%%%%%%%%%%%%%%%%%%%%%%%%%%%%%%%%%%%%%%%%%%%%%%%%%%%%%%%%%%%%%%%%%%%%%%%%%%%%%%%%%%%%%%%%%%%%%%%%%%%%%%%%%%%%%%%%%%%%%%%%%%%%%%%%%%%%
% Written By Michael Brodskiy
% Class: General Relativity and Cosmology
% Professor: J. Blazek
%%%%%%%%%%%%%%%%%%%%%%%%%%%%%%%%%%%%%%%%%%%%%%%%%%%%%%%%%%%%%%%%%%%%%%%%%%%%%%%%%%%%%%%%%%%%%%%%%%%%%%%%%%%%%%%%%%%%%%%%%%%%%%%%%%%%%%%%%%%%%%%%%%%%%%%%%%%%%%%%%%%

\documentclass[12pt]{article} 
\usepackage{alphalph}
\usepackage[utf8]{inputenc}
\usepackage[russian,english]{babel}
\usepackage{titling}
\usepackage{amsmath}
\usepackage{graphicx}
\usepackage{enumitem}
\usepackage{amssymb}
\usepackage[super]{nth}
\usepackage{everysel}
\usepackage{ragged2e}
\usepackage{geometry}
\usepackage{multicol}
\usepackage{fancyhdr}
\usepackage{cancel}
\usepackage{siunitx}
\usepackage{physics}
\usepackage{tikz}
\usepackage{mathdots}
\usepackage{yhmath}
\usepackage{cancel}
\usepackage{color}
\usepackage{array}
\usepackage{multirow}
\usepackage{gensymb}
\usepackage{tabularx}
\usepackage{extarrows}
\usepackage{booktabs}
\usepackage{lastpage}
\usetikzlibrary{fadings}
\usetikzlibrary{patterns}
\usetikzlibrary{shadows.blur}
\usetikzlibrary{shapes}

\geometry{top=1.0in,bottom=1.0in,left=1.0in,right=1.0in}
\newcommand{\subtitle}[1]{%
  \posttitle{%
    \par\end{center}
    \begin{center}\large#1\end{center}
    \vskip0.5em}%

}
\usepackage{hyperref}
\hypersetup{
colorlinks=true,
linkcolor=blue,
filecolor=magenta,      
urlcolor=blue,
citecolor=blue,
}


\title{Lecture 2 — Introduction to Differential Geometry}
\date{\today}
\author{Michael Brodskiy\\ \small Professor: J. Blazek}

\begin{document}

\maketitle

\begin{itemize}

  \item Metric $\to$ measuring things (``meter'')

  \item In differential geometry, a metric defines how we calculate distance

    $$\Delta s^2=-\Delta t^2+\Delta x^2+\Delta y^2+\Delta z^2\,\quad\text{ (the metric in Minkowski space)}$$
    $$=\eta_{\mu\nu}\Delta x^{\mu}\Delta x^{\nu}\quad\quad\quad\quad\quad\quad\quad\quad\quad\quad\quad\quad\quad\quad\quad\quad\quad\quad\quad\quad$$

    \begin{itemize}

      \item A repeated index (up and down) $\to$ sum

        \begin{itemize}

          \item Greek: 0-3

            $$\begin{array}{l r} \Delta x^{\mu} & \Delta x^0\\ & \Delta x^1\\ & \Delta x^2\\ & \Delta x^3\end{array}$$

          \item Latin: 1-3

            $$\begin{array}{l r} \Delta x^{i} & \Delta x^1\\ & \Delta x^2\\ & \Delta x^3\end{array}$$

        \end{itemize}

        $$\eta_{00}\Delta x^0\Delta x^0+\eta_{01}\Delta x^0\Delta x^1+\eta_{02}\Delta x^0\Delta x^2+\eta_{10}\Delta x^1\Delta x^0+\eta_{11}\Delta x^1\Delta x^1+\cdots$$

      \item This can be written as:

        $$\eta_{\mu\nu}=\left( \begin{array}{c c c c} -1 & 0 & 0 & 0\\ 0 & 1 & 0 & 0\\ 0 & 0 & 1 & 0\\ 0 & 0 & 0 & 1\\ \end{array}\right)$$
        $$=\eta_{00}\Delta x^0\Delta x^0+=\eta_{11}\Delta x^1\Delta x^1+=\eta_{22}\Delta x^2\Delta x^2+=\eta_{33}\Delta x^3\Delta x^3$$

    \end{itemize}

\end{itemize}

\end{document}

