%%%%%%%%%%%%%%%%%%%%%%%%%%%%%%%%%%%%%%%%%%%%%%%%%%%%%%%%%%%%%%%%%%%%%%%%%%%%%%%%%%%%%%%%%%%%%%%%%%%%%%%%%%%%%%%%%%%%%%%%%%%%%%%%%%%%%%%%%%%%%%%%%%%%%%%%%%%%%%%%%%%
% Written By Michael Brodskiy
% Class: General Relativity and Cosmology
% Professor: J. Blazek
%%%%%%%%%%%%%%%%%%%%%%%%%%%%%%%%%%%%%%%%%%%%%%%%%%%%%%%%%%%%%%%%%%%%%%%%%%%%%%%%%%%%%%%%%%%%%%%%%%%%%%%%%%%%%%%%%%%%%%%%%%%%%%%%%%%%%%%%%%%%%%%%%%%%%%%%%%%%%%%%%%%

\include{Includes.tex}

\title{Lecture 2 — Introduction to Differential Geometry}
\date{September 9 — \today}
\author{Michael Brodskiy\\ \small Professor: J. Blazek}

\begin{document}

\maketitle

\begin{itemize}

  \item Metric $\to$ measuring things (``meter'')

  \item In differential geometry, a metric defines how we calculate distance

    $$\Delta s^2=-\Delta t^2+\Delta x^2+\Delta y^2+\Delta z^2=-\Delta\tau^2\,\quad\text{ (the metric in Minkowski space)}$$
    $$=\eta_{\mu\nu}\Delta x^{\mu}\Delta x^{\nu}\quad\quad\quad\quad\quad\quad\quad\quad\quad\quad\quad\quad\quad\quad\quad\quad\quad\quad\quad\quad\quad\quad\quad\quad$$

    \begin{itemize}

      \item A repeated index (up and down) $\to$ sum

        \begin{itemize}

          \item Spacetime Vector \to Greek: 0-3

            $$\vec{x}=\left( \begin{array}{c} t\\x\\y\\z\end{array} \right)\to\left(\begin{array}{x} x^0\\ x^1\\ x^2\\ x^3\end{array}\right)\to x^{\mu}$$

          \item Vector $\to$ Latin: 1-3

            $$\vec{x}=\left( \begin{array}{c} x\\y\\z\end{array} \right)\to\left(\begin{array}{x} x^1\\ x^2\\ x^3\end{array}\right)\to x^{i}$$

        \end{itemize}

        $$\eta_{00}\Delta x^0\Delta x^0+\eta_{01}\Delta x^0\Delta x^1+\eta_{02}\Delta x^0\Delta x^2+\eta_{10}\Delta x^1\Delta x^0+\eta_{11}\Delta x^1\Delta x^1+\cdots$$

      \item Tensors, index notation:

        $$\eta_{\mu\nu}=\left( \begin{array}{c c c c} -1 & 0 & 0 & 0\\ 0 & 1 & 0 & 0\\ 0 & 0 & 1 & 0\\ 0 & 0 & 0 & 1\\ \end{array}\right)$$
        $$=\eta_{00}\Delta x^0\Delta x^0+=\eta_{11}\Delta x^1\Delta x^1+=\eta_{22}\Delta x^2\Delta x^2+=\eta_{33}\Delta x^3\Delta x^3$$

        \begin{itemize}

          \item Summation convention, one up and one down (order does not matter)

        \end{itemize}

    \end{itemize}

  \item Curved Space Distance

    \begin{itemize}

      \item We know $d\neq|x_1-x_2|$

      \item $\Delta s^2=\Delta x^2=\eta_{\mu\nu}\Delta x^{\mu}\Delta x^{\nu}\Rightarrow g_{\mu\nu}\Delta x^{\mu}\Delta x^{\nu}$, where $g_{\mu\nu}$ is a metric that depends on a radius $R$ (called manifolds, encodes geometry)

      \item Note: 1D or 2D analogies are embedded in 3D

      \item In differential geometry, we will generally deal with:

        $$ds^2=g_{\mu\nu}dx^{\mu}dx^{\nu}$$

    \end{itemize}

  \item Revisiting Lorentz Transformations

    \begin{itemize}

      \item We can write a transformation in two ways:

        $$\Lambda=\left( \begin{array}{c c c c} \gamma & -v\gamma & 0 & 0\\ -v\gamma & \gamma & 0 & 0\\ 0 & 0 & 1 & 0\\ 0 & 0 & 0 & 1\end{array} \right)$$
        $$\Lambda=\left( \begin{array}{c c c c} \cosh(\phi) & -\sinh(\phi) & 0 & 0\\ -\sinh(\phi) & \cosh(\phi) & 0 & 0\\ 0 & 0 & 1 & 0\\ 0 & 0 & 0 & 1\end{array} \right)$$

        \item With four-vectors:

          $$\vec{x}=\left( \begin{array}{c} \gamma t\\ x\\ y\\ z\end{array} \right)$$
          $$\vec{x}'=\left( \begin{array}{c} \gamma t - vx\\ -vt+\gamma x\\ y\\ z\end{array} \right)$$

        \item Matrix multiplication with indices becomes:

          $$x^{\mu'}=\Lambda^{\mu'}_{\nu}x^{\nu}$$

    \end{itemize}

  \item The metric and Lorentz Transformations

    \begin{itemize}

      \item $\Delta s^2$ is invariant under boosts ($x^{\mu}\to x^{\mu'}$)

      \item $\Delta s^2=\eta_{\mu\nu}\Delta x^{\mu}\Delta x^{\nu}=\eta_{\mu'\nu'}\Delta x^{\mu'}\Delta x^{\nu'}=\eta_{\mu'\nu'}\Lambda^{\mu'}_{\mu}\Delta x^{\mu}\Lambda^{\nu'}_{\nu}\Delta x^{\nu}$

      \item This defines Lorentz Transformations (group)

    \end{itemize}

  \item What is a group?

    \begin{itemize}

      \item A set $\left\{ a,b,c,\cdots \right\}$

        \begin{enumerate}

          \item Has an operation ``$\cdot$''

          \item Operation is associative: $(a\cdot b)\cdot c=a\cdot (b\cdot c)$

          \item Set is closed: if $a\cdot b=c$, $c$ is in the group

          \item Contains an identity element: $a\cdot e=e\cdot a=a$

          \item Contains an inverse for all elements: $a\cdot a^{-1}=e$

        \end{enumerate}

      \item Can be finite or infinite

      \item Simple example: integers under addition
        '
      \item Rotations in space are a group and can be represented by matrices with multiplication:

        $$R_{z,\theta}=\left( \begin{array}{ccc} \cos(\theta) & -\sin(\theta) & 0 \\ \sin(\theta) & \cos(\theta) & 0 \\ 0 & 0 & 1 \end{array}\right)$$

        \begin{itemize}

          \item SO(3) ``special orthogonal in 3D''

        \end{itemize}

      \item Operations don't commute: non-abelian group

      \item In 4D, spatial rotations and boosts form the Lorentz transformations (group) and translations $\to$ Poincar\'e group

    \end{itemize}

  \item Vectors and Covectors

    \begin{itemize}

      \item We already have a concept of vectors:

        \begin{itemize}

          \item $\vec{v}$ exists at a single point in spacetime in the tangent space $T_p$

          \item Vectors from $T_p$ can not simply be moved to $T_q$

          \item Example: $v^{\mu}=\dfrac{d}{d\lambda}x^{\mu}$ tangent to $x^{\mu}(\lambda)$

        \end{itemize}

      \item Vector field: one vector at each spacetime point

      \item Vectors are invariant under $\Lambda$

        \begin{itemize}

          \item Example: Wind velocity at every point in space:

            \begin{itemize}

              \item Changing frames alters components, but not the vector itself (why we ``prime'' the index)

            \end{itemize}

        \end{itemize}

      \item This can be written as:

        $$\text{\footnotesize vector}\to A=\hspace{-15pt}\overbrace{A^{\mu}}^{\text{components}}\hspace{-25pt}\underbrace{\hat{e}_{(\mu)}}_{\text{basis vectors}}$$

        \begin{itemize}

          \item $\hat{e}_{(\mu)}$ does NOT refer to dual vectors

          \item $(\mu)$ is not a coordinate index

        \end{itemize}

        $$\text{\footnotesize unprimed to primed}\to\Lambda^{\nu'}_{\mu}\longleftrightarrow\Lambda^{\rho}_{\sigma'}\leftarrow\text{\footnotesize primed to unprimed}$$
        $$\Lambda^{\nu'}_{\mu}\Lambda^{\rho}_{\sigma'}=\delta_{\mu}^{\rho}\text{ (Kronecker delta)}$$

    \end{itemize}

  \item Dual Vectors (``One-forms,'' covariant vectors)

    \begin{itemize}

      \item A map from vectors to $\mathbb{R}$

      \item Ex.

        $$v=\left( \begin{array}{c}a\\b\\c\end{array} \right)\text{ and }w=(d\quad e\quad f)$$
        $$w(v)=(d\quad e\quad f)\left(\begin{array}{c}a\\b\\c\end{array}\right)=da+eb+fc$$

        \begin{itemize}

          \item $w(v)$ is not the dot product, though it is similar in spirit

        \end{itemize}

      \item Cotangent space $T_p^{*}$

      \item Similar basis structure and transformations

        $$w=w_{\mu}\hat{\theta}^{(\mu)}$$

      \item where 

        $$\hat{\theta}^{(\nu)}\left( \hat{e}_{(\mu)} \right)=\delta^{\nu}_{\mu}$$

    \end{itemize}

  \item The Gradient

    \begin{itemize}

      \item Recall $\nabla\phi=\left( \frac{\partial}{\partial x},\frac{\partial}{\partial y},\frac{\partial}{\partial z} \right)\phi\Rightarrow\frac{\partial\phi}{\partial x}\hat{x}+\frac{\partial\phi}{\partial y}\hat{y}+\frac{\partial \phi}{\partial z}\hat{z}$

      \item It is a dual vector

    \end{itemize}

  \item Tensors

    \begin{itemize}

      \item A $(k,l)$-rank tensor maps $k$ dual vectors and $l$ vectors to $\mathbb{R}$

        \begin{center}
          $$\begin{array}{l l}\text{scalar} & (0,0)\\ \text{vector} & (1,0)\\\text{dual vector} & (0,1)\\\text{metric} & (0,2)\end{array}$$
        \end{center}

      \item Tensors obey ``multi-linearity''

        $$T(a\omega+b\eta,cV+dW)=acT(\omega,V)+adT(\omega,W)+bcT(\eta,V)+bdT(\eta,W)$$

      \item Tensor Product

        $$T_2=T\otimes S(\omega^(1)\cdots\omega^{(k)}, \cdots\omega^{(k+m)},V^{(1)}\cdots V^{(l)}, V^{(l+m)})$$
        $$=T(\omega^{(1)}\cdots\omega^{(k)},V^{(1)}\cdots V^{(l)})\times S(\omega^{(k+1)}\cdots\omega^{(k+m)},V^{(l+1)}\cdots V^{(l+m)})$$

      \item Basis for a $(k,l)$ tensor:

        $$\hat{e}_{\mu_1}\otimes\cdots\otimes\hat{e}_{\mu_k}\otimes\hat{\theta}^{(\nu_1)}\otimes\cdots\otimes\hat{\theta}^{(\nu_l)}$$

        \begin{itemize}

          \item $\mu_i$ has $D$ values for $D$ dimensions ($D=4$ for us), $4^{k+l}$ total basis vectors

        \end{itemize}

        $$T=T^{\mu_1\ldots\mu_k}_{\nu_1\ldots\nu_l}\times\text{(Basis tensors)}$$

      \item Transformations under $\Lambda$ (builds from vector transforms)

        $$T^{\mu_1'\ldots\mu_k'}_{\nu_1'\ldots\nu_l^'}=\Lambda^{\mu_1'}_{\mu_1}\cdots\Lambda^{\mu_k'}_{\mu_k}\Lambda^{\nu_1}_{\nu_1'}\cdots\Lambda^{\nu_l}_{\nu_l'}T^{\mu_1\ldots\mu_k}_{\nu_1\ldots\nu_l'}$$

      \item $T$ can act on a subset

    \end{itemize}

  \item The inner product (dot product)

    $$\eta(V,W)=V\cdotW=\eta_{\mu\nu}V^{\mu}W^{\nu}$$

    \begin{itemize}

      \item The metric appears again!

    \end{itemize}

    $$\eta^{\mu\nu}\eta_{\nu\sigma}=\delta^{\mu}_{\sigma}\text{ (inverse metric)}$$

  \item Another famous tensor example: E\&M Field Strength Tensor

    $$F_{\mu\nu}=\left( \begin{array}{cccc} 0 & -E_1 & -E_2 & -E_3\\E_1& 0 & B_3 & -B_2\\ E_2 & -B_3 & 0 & B_1\\ E_3 & B_2 & -B_1 & 0\end{array} \right)$$

  \item Manipulating Tensors

    \begin{itemize}

      \item Contraction: $S^{\mu\rho}_{\sigma}=T^{\mu\nu\rho}_{\sigma\nu}$

      \item Indices are arbitrary, until they are set:

        $$T^{\mu\nu\rho}\neq T^{\nu\mu\rho}\text{ because we compare them}$$

      \item The metric raises/lowers indices:

        \begin{itemize}

          \item Recall inner product:

            $$\eta_{\mu\nu}V^{\mu}U^{\nu}=V_{\nu}U^{\nu}$$

        \end{itemize}

      \item Symmetric versus Anti-symmetric Tensors:

        $$S^{\mu\nu}=S^{\nu\mu}\quad\quad S^{\mu\nu}_{\sigma}=S^{\sigma\nu}_{\mu}$$
        $$A^{\mu\nu}=-A^{\nu\mu}\quad\quad A^{\mu\nu}_{\sigma}=-A^{\sigma\nu}_{\mu}$$

        \begin{itemize}

          \item Levi-Civita Symbol (Tensor ``density'') is anti-symmetric

            $$\tilde{\varepsilon}_{\mu\nu\rho\sigma}=\left\{\begin{array}{l l} +1 & \text{even perm }0123\text{ (eg 0312)}\\ -1 & \text{odd perm }0123\end{array}$$

        \end{itemize}

    \end{itemize}

\end{itemize}

\end{document}

