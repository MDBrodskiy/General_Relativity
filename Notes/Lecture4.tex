%%%%%%%%%%%%%%%%%%%%%%%%%%%%%%%%%%%%%%%%%%%%%%%%%%%%%%%%%%%%%%%%%%%%%%%%%%%%%%%%%%%%%%%%%%%%%%%%%%%%%%%%%%%%%%%%%%%%%%%%%%%%%%%%%%%%%%%%%%%%%%%%%%%%%%%%%%%%%%%%%%%
% Written By Michael Brodskiy
% Class: General Relativity and Cosmology
% Professor: J. Blazek
%%%%%%%%%%%%%%%%%%%%%%%%%%%%%%%%%%%%%%%%%%%%%%%%%%%%%%%%%%%%%%%%%%%%%%%%%%%%%%%%%%%%%%%%%%%%%%%%%%%%%%%%%%%%%%%%%%%%%%%%%%%%%%%%%%%%%%%%%%%%%%%%%%%%%%%%%%%%%%%%%%%

\include{Includes.tex}

\title{Lecture 4 — Manifolds and Curved Spacetime}
\date{\today}
\author{Michael Brodskiy\\ \small Professor: J. Blazek}

\begin{document}

\maketitle

\begin{itemize}

  \item We now move from Minkowski to General Space:

    $$\eta_{\mu\nu}\to g_{\mu\nu}$$

  \item Differentiable Manifolds

    \begin{itemize}

      \item Manifold: A space (in $n$-dimensions) that looks locally like $\mathbb{R}^n$ and can be constructed by smoothly stitching together these regions

      \item Rotations in $\mathbb{R}^n$ $\to$ Lie Groups are manifolds with a group structure

      \item To be more precise, we have a set $M$ with a set of (all possible) charts of open subsets to $\mathbb{R}^n$

        \begin{itemize}
            
          \item Chart $\leftrightarrow$ coordinate system

        \end{itemize}

      \item These charts must be smooth, continuous, invertible, and differentiable

    \end{itemize}

\end{itemize}

\end{document}

