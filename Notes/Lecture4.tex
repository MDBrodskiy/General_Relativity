%%%%%%%%%%%%%%%%%%%%%%%%%%%%%%%%%%%%%%%%%%%%%%%%%%%%%%%%%%%%%%%%%%%%%%%%%%%%%%%%%%%%%%%%%%%%%%%%%%%%%%%%%%%%%%%%%%%%%%%%%%%%%%%%%%%%%%%%%%%%%%%%%%%%%%%%%%%%%%%%%%%
% Written By Michael Brodskiy
% Class: General Relativity and Cosmology
% Professor: J. Blazek
%%%%%%%%%%%%%%%%%%%%%%%%%%%%%%%%%%%%%%%%%%%%%%%%%%%%%%%%%%%%%%%%%%%%%%%%%%%%%%%%%%%%%%%%%%%%%%%%%%%%%%%%%%%%%%%%%%%%%%%%%%%%%%%%%%%%%%%%%%%%%%%%%%%%%%%%%%%%%%%%%%%

\include{Includes.tex}

\title{Lecture 4 — Manifolds and Curved Spacetime}
\date{\today}
\author{Michael Brodskiy\\ \small Professor: J. Blazek}

\begin{document}

\maketitle

\begin{itemize}

  \item We now move from Minkowski to General Space:

    $$\eta_{\mu\nu}\to g_{\mu\nu}$$

  \item Differentiable Manifolds

    \begin{itemize}

      \item Manifold: A space (in $n$-dimensions) that looks locally like $\mathbb{R}^n$ and can be constructed by smoothly stitching together these regions

      \item Rotations in $\mathbb{R}^n$ $\to$ Lie Groups are manifolds with a group structure

      \item To be more precise, we have a set $M$ with a set of (all possible) charts of open subsets to $\mathbb{R}^n$

        \begin{itemize}
            
          \item Chart $\leftrightarrow$ coordinate system

        \end{itemize}

      \item These charts must be smooth, continuous, invertible, and differentiable

      \item Now we will define (co)tangent spaces on these manifolds, with metrics that map (dual) vectors to $\mathbb{R}$

    \end{itemize}

  \item The Equivalence Principle

    \begin{itemize}

      \item In special relativity, we had the principle that the laws of physics were the same in all inertial frames

      \item Einstein's ``happiest though'': If someone falls from a roof, nothing falls in their frame

      \item Equivalence of inertial frames should be generalized to include gravity

      \item Weak Equivalence Principle (WEP)

        \begin{itemize}

          \item Inertial mass = gravitational mass

            $$F=m_ia\text{ (inertial)}$$
            $$F=-m_g\nabla\Phi\text{ (gravitational "charge")}$$
            $$m_i=m_g\text{ (WEP: E\"otu\"os experiments, late 19th century)}$$

          \item All freely falling bodies behave the same/are indistinguishable ($a=-\nabla\Phi$)

          \item Define inertial trajectory as unaccelerated (subject only to gravity)

          \item In small enough regions of space-time, freely falling particles behave the same in a gravitational field or a uniformly accelerated field (physicist in a box, accelerating reference frame)

        \end{itemize}

      \item Strong Equivalence Principle (SEP)

        \begin{itemize}

          \item All laws of physics, including gravitation, look like SR

            \begin{itemize}

              \item Einstein Equivalence Principle (EEP) plus the impact of gravitational binding energy

              \item Rules out ``fifth force''

            \end{itemize}

        \end{itemize}

    \end{itemize}

  \item Tidal Forces

    \begin{itemize}

      \item Causes tides on Earth

      \item Locally inertial frames

    \end{itemize}

  \item Gravitational Redshift

    $$\Delta v=\frac{az}{c}$$

    \begin{itemize}

      \item Relativistic Doppler Shift:

        $$\lambda_{obs}=\lambda_o\left( \frac{1+\dfrac{v}{c}}{1-\dfrac{v}{c}} \right)^{\frac{1}{2}}$$

        \begin{itemize}

          \item Using Taylor expansion, we may simply write this as:

            $$\lambda_{obs}=1+\frac{v}{c}$$

          \item Bringing this together, we get:

            $$\frac{\Delta\lambda}{\lambda_o}\approx\frac{\Delta v}{c}=\frac{az}{c^2}$$

        \end{itemize}

      \item EEP says that this must be the same as a gravitational field:

        $$\frac{\Delta\lambda}{\lambda_o}=\frac{a_gz}{c^2}=\frac{\Delta\Phi}{c^2}$$

      \item This is the time from start to end of wavelength, and can be used to compare clocks

      \item If we have a case where $\Delta t_o=\lambda_oc^{-1}$ and $\Delta t_1=\lambda_1c^{-1}$, and $\lambda_1>\lambda_o$ then $\Delta t_1>\Delta t_o$, which indicates gravitational time dilation

    \end{itemize}

  \item Classic Tests of General Relativity

    \begin{enumerate}

      \item Precession of the perihelion of Mercury — \nth{19} century: 43'' per century discrepancy successful ``post-diction'' of GR (about 10\% of total effect)

      \item Bending of star light by sun (gravitational lensing) — GR predicts a factor of 2 larger deflection (1919 Eddington Expedition to observe the solar eclipse)

      \item Gravitational Redshift — 1954: Popper measurement of a white dwarf, 1959: Pound-rebka at Jefferson lab (Harvard), 22.5m

    \end{enumerate}

  \item Vectors and Tensors on Manifolds (Curved Spacetime)

    \begin{itemize}

      \item We already saw $V=V^{\mu}\hat{e}_{\mu}$ at point $P$ on $T_p$

      \item What is the basis?

        \begin{itemize}

          \item We want to define tangent vectors before we have a vector space on $M$

          \item Instead, consider a function $f$ and a curve $\lambda$. The directional derivative is:

            $$\frac{d}{d\lambda}x^{\mu}\frac{\partial}{\partial x^{\mu}}f=\frac{d}{d\lambda}x^{\mu}\partial_{\mu}f\quad\text{ (gradient $\cdot$ tangent $\vec{v}$)}$$

          \item $f$ could have been anything, so we define the tangent vector:

            $$\frac{d}{d\lambda}=\frac{dx^{\mu}}{d\lambda}\partial_{\mu}$$
            
          \item $\left\{ \hat{e}_{\mu}=\partial_{\mu} \right\}$ is the coordinate basis (``points'' in the direction of $x^{\mu}$)

          \item Not orthonormal, but always well defined

          \item In this basis, things transform according to:

            $$\partial_{\mu}\prime=\frac{\partial}{\partial x^{\mu}\prime}=\frac{\partial}{\partial x^{\mu}}\frac{\partial x^{\mu}}{\partial x^{\mu}\prime}=\frac{\partial x^{\mu}}{\partial x^{\mu}\prime}\partial_{\mu}$$

          \item Similarly, $V=V^{\mu}\partial_{\mu}$ is preserved, so:

            $$V^{\mu}\prime\partial_{\mu}\prime=V^{\mu}\partial_{\mu}\Rightarrow V^{\mu}\prime=\frac{\partial x^{\mu}\prime}{\partial x^{\mu}}V^{\mu}$$

        \end{itemize}

    \end{itemize}

  \item General Coordinate Transform

    \begin{itemize}

      \item In flat space: $x^{\mu}\prime=\Lambda^{\mu}_{\mu}\primex^{\mu}$

        $$\frac{dx^{\mu}\prime}{dx^{\mu}}=\Lambda^{\mu}_{\mu}\prime$$

      \item We recover the transform of vectors

      \item Vector Fields:

        \begin{itemize}

          \item $X$: One vector at each point on the manifold

          \item $X,Y$: Both define a field that can be used to take directional derivatives of functions on $\mu$

            $$[X,Y](f)=X(Y(f))-Y(X(f))\quad\text{ \underline{commutator}}$$

        \end{itemize}

      \item Dual Vectors

        \begin{itemize}

          \item Recall we defined the gradient: $df$

            $$df\left( \frac{d}{d\lambda} \right)=\frac{df}{d\lambda}\quad\text{ map a vector to $\mathbb{R}$}$$

          \item Basis for dual vectors $dx^{\mu}$

          \item Gradient of the coordinate function:

            $$dx^{\mu}(\partial_{\nu})=\frac{dx^{\mu}}{dx^{\nu}}=\delta^{\mu}_{\nu}$$
            $$V=V^{\mu}\partial_{\mu}$$
            $$\omega=\omega_{\nu}dx^{\nu}$$
            $$\omega_{\mu\prime}=\frac{\partial x^{\mu}}{\partial x^{\mu\prime}}\omega_{\mu}$$

        \end{itemize}

      \item We can now write the transformation of an arbitrary $(k,l)$ tensor on a manifold:

        $$T^{\mu_1\prime\ldots\mu_{k}\prime}_{\nu_1\prime\ldots\nu_l\prime}=\frac{\partial x^{\mu_1\prime}}{\partial x^{\mu_1}}\cdots\frac{\partial x^{\mu_k\prime}}{\partial x^{\mu_k}}\frac{\partial x^{\nu_1}}{\partial x^{\nu_1\prime}}\cdots\frac{\partial x^{\nu_k}}{\partial x^{\nu_k\prime}}T^{\mu_1\ldots\mu_k}_{\nu_1\ldots\nu_l}$$

        \begin{itemize}

          \item Warning: in curved space $\partial_{\mu}W_{\nu}$ is not a tensor; unlike in flat space, the derivative of the transform can be non-zero ($\Lambda$ is the same everywhere)

        \end{itemize}

    \end{itemize}

  \item The Metric

    \begin{itemize}

      \item $\eta_{\mu\nu}$ in Minkowski space

      \item $g_{\mu\nu}$ in general curved spacetime

        $$g_{\mu\nu}g^{\nu\sigma}=\delta_{\mu}^{\sigma}\quad\text{(defines inverse)}$$

      \item Metric really describes basically everything about a spacetime

        $$ds^2=g_{\mu\nu}dx^{\mu}dx^{\nu}$$

        \begin{itemize}

          \item (0,2)-tensor components, metric components, with $ds^2$ being called the ``line element'' or ``metric''

          \item Usually we just write $g_{\mu\nu}$

          \item In 3D Flat Space:

            $$ds^2=dx^2+dy^2+dz^2=dr^2+r^2d\theta^2+r^2\sin^2(\theta)d\phi^2$$

            \begin{itemize}

              \item Components and bases both change, while $ds^2$ does not

            \end{itemize}

          \item Canonical form: coordinate transform to diagonalize and normalize

            $$g_{\mu\nu}=\left( \begin{matrix} -1 & \cdots & \cdots & \cdots & \cdots & 0\\ \vdots & \ddots & & &\\ \vdots & & -1 & &\\ \vdots & & & 1 & \\ \vdots & & & & \ddots & \\ 0 & & & & & 1 \end{matrix}\\\right)$$

          \item ``Signature'' is -+++, etc.

          \item All positive: Euclidean or Riemannian

          \item One negative: Lorentzian of pseudo-Riemannian

          \item Any zeros? Degenerate

          \item At some point $p$, you can always put a metric into canonical form, and make the first derivatives vanish

            \begin{itemize}

              \item There is always enough freedom to choose coordinates that do this

              \item Second derivatives will not generally vanish

            \end{itemize}

          \item In our case, choose:

            $$x^{\hat{\mu}} \text{ at }p\text{ such that:}$$

            $$g_{\hat{\mu}\hat{\nu}}(p)=\eta_{\hat{\mu}\hat{\nu}}\quad\quad\partial_{\hat{\sigma}}g_{\hat{\mu}\hat{\nu}}\quad\quad \partial_{\hat{\rho}}\partial_{\hat{\sigma}}g_{\hat{\mu}\hat{\nu}}\neq0$$

          \item Note: not a coordinate system, so not a coordinate basis

            $$x^{\mu}\to\text{ locally inertial coordinates}$$

            \begin{itemize}

              \item Locally inertial/Lorentz frame

              \item Do calculations in this frame, express in tensor (covariant) form — usually can just assume $\eta_{\mu\nu}$

            \end{itemize}

        \end{itemize}

    \end{itemize}

  \item Coordinate Basis

    $$V=V^{\mu}\hat{e}_{\mu}$$
    $$\hat{e}_{\mu}=\left\{ \partial_{\mu} \right\}$$
    $$\partial_{\mu}=\frac{\partial}{\partial x^{\mu}}$$

  \item Tensor Density

    $$\tilde{\epsilon}_{\mu_1\ldots\mu_n}=\left\{ \begin{matrix} +1, & \text{even perm (0123)}\\ -1, & \text{odd perm (0213)}\\ 0, & \text{other (0112)}\end{matrix}$$

    \begin{itemize}

      \item Our general transformation $\dfrac{\partial x^{\mu}}{\partial x^{\mu'}}$ is a particular case of $M^{\mu'}_{\mu}$

        $$\tilde{\epsilon}_{\mu_1'\ldots\mu_2'}=\Big|\frac{\partial x^{\mu'}}{\partial x^{\mu}}\Big|\tilde{\epsilon}_{\mu_1\ldots\mu_n}\frac{\partial x^{\mu_1}}{\partial x^{\mu_1'}}\cdots\frac{\partial x^{\mu_n}}{\partial x^{\mu_n'}}$$

      \item We may see that there is an extra factor compared to standard tensor transform

      \item What about $|g_{\mu\nu}|=g$?

        $$g_{\mu'\nu'}=\frac{\partial x^{\mu}}{\partial x^{\hat{\mu}}}\frac{\partial x^{\nu}}{\partial x^{\hat{\nu}}}g_{\mu\nu}$$

      \item In general, for $\tilde{t}$ that transforms with $\Big|\dfrac{\partial x^{\mu'}}{\partial x^{\hat{\mu}}}\Big|^{\omega}$, we can make a real tensor $t=\tilde{t}|g|^{\omega/2}$, since this will transform with:

        $$\Big|\dfrac{\partial x^{\mu}}{\partial x^{\hat{\mu'}}}\Big|^{\omega}\Big|\dfrac{\partial x^{\mu'}}{\partial x^{\hat{\mu}}}\Big|^{\omega}=1$$
        $$\epsilon_{\mu_1\ldots\mu_n}=\sqrt{|g|}\tilde{\epsilon}_{\mu_1\ldots\mu_n}$$
        $$\epsilon^{\mu_1\ldots\mu_n}=\epsilon_{\mu_1\ldots\mu_n}\cdot\text{sign}(g)$$

    \end{itemize}

  \item Differential Forms

    \begin{itemize}

      \item p-form is a $(0,p)$ anti-symmetric tensor with:

        $$A_{\mu\nu}=-A_{\nu\mu}$$
        $$A_{\mu\nu\sigma}=-A_{\nu\mu\sigma}=A_{\nu\sigma\mu}$$

        \begin{itemize}

          \item Scalars: 0-forms

          \item Dual vectors: 1-forms

        \end{itemize}

      \item Wedge product is anti-symmetrized tensor product

        $$(A\wedge B)_{\mu_1\ldots\mu_{p+q}}=\frac{(p+q)!}{p!q!}A_{[\mu_1\ldots\mu_p]}B_{[\mu_{p+1}\ldots\mu_{p+q}]}$$

    \end{itemize}

  \item Exterior Derivative

    \begin{itemize}

      \item A tensor, unlike the partial derivative, but only acts on forms

        $$(dA)_{\mu\ldots\mu_{p+1}}=(p+1)\partial_{\mu_1}A_{\mu_2\ldots\mu_{p+1}}$$

      \item We have already seen this operator, the gradient (of a scalar):

        $$(d\phi)_{\mu}=\partial_{\mu}\phi$$

      \item Because partials commute:

        $$d(dA)=0$$

    \end{itemize}

  \item Electrodynamics

    $$F_{\mu\nu}$$
    $$\partial_{[\mu}F_{\nu\lambda]}=0$$
    $$dF=0$$
    $$F=dA\text{ (vector potential }A_{\mu})$$

  \item Hodge Star

    \begin{itemize}

      \item $*A$ takes $p$-forms to $n-p$ forms

    \end{itemize}

  \item Integration on Manifolds

    \begin{itemize}

      \item In general, when changing coordinates in an integral, we multiply by the Jacobian

        $$x,y,z\to r,\theta,\phi$$
        $$\Big| \frac{\partial x^{\mu'}}{\partial x^{\mu}}\Big=r^2\sin(\theta)$$
        $$\int dx\,dy\,dz\,f(\vec{r})\to\int r^2\sin(\theta)\,dr\,d\theta\,d\phi\, f(\vec{r})$$

        \begin{itemize}

          \item This is a particular example of integration on a manifold

        \end{itemize}

        $$\int w(x)\,dx=w$$

      \item Represents a component $w(x)$ with basis $dx$

      \item We may write:

      $$d\mu\left( U,V,W\right)\leftarrow \mathbb{R}$$

      \begin{itemize}

        \item This maps vectors at a point to their volume

      \end{itemize}

    \item We generalize:

      $$d^nx=dx^o\wedge\cdots\wedge dx^{n-1}$$

      \begin{itemize}

        \item This is not yet a tensor because it is coordinate independent

      \end{itemize}

      $$\sqrt{|g|}d^nx=\sqrt{|g|}dx^o\wedge\cdots\wedge dx^{n-1}=\sqrt{|g'|}dx^{o'}\wedge\cdots\wedge dx^{(n-1)'}$$
      $$I=\int\phi(x)\sqrt{|g|}\,d^nx\to\int\phi(x)\,dx\,dy\,dz\text{ if }|g|=1$$

    \item You can then evaluate as normal

    \end{itemize}

  \item FLRW Metric (or FRW, or RW)

    \begin{itemize}

      \item Friedmann, Lemaitre, Robertson, Walker $\to$ Solution to Einstein's equations for a spatially homogenous, isotropic spacetime. Can be curved or flat. Flat FLRW:

        $$dx^2=-dt^2+a^2(t)(dx^2+dy^2+dz^2)$$

        \begin{itemize}

          \item Where $a(t)$ is the scale factor

          \item For $a(t)=t^q$, $0<q<1$

          \item $t=(1-q)^{\frac{1}{1-q}}(\pm x-x_o)^{\frac{1}{1-q}}$

          \item Light not always at a $45^{\circ}$ angle

          \item Singularity at $t=0\to$ cosmic horizon $\to p$ and $s$ are completely disconnected

        \end{itemize}

    \end{itemize}

  \item Curvature, Covariant Derivatives, Geodesics

    \begin{itemize}

      \item For $S^2$:

        $$ds^2=\frac{R^2dr}{R^2-r^2}+r^2d\theta^2$$

      \item The metric reflects the curvature; for $R\to\infty$, we get back to flat 2D space

      \item From the metric, we will derive the ``connection,'' which tells us the impact of curvature, including defining straight lines

      \item The connection ($\Gamma^{\lambda}_{\mu\nu}$, not a tensor) tells us how to compare vectors at nearby points
        
    \end{itemize}

  \item Covariant Derivatives

    \begin{itemize}

      \item Recall $\partial_{\mu}A_{\nu}$ is not a covariant because of an extra term

      \item External derivative $dA=\partial_{\mu}A_{\nu}-\partial_{\nu}A_{\mu}$ is covariant but only valid on forms

      \item We define the covariant derivative of a vector and dual vector:

        $$\nabla_{\mu}V^{\nu}=\partial_{\mu}V^{\nu}+\Gamma^{\nu}_{\mu\lambda}V^{\lambda}$$
        $$\nabla_{\mu}\omega^{\nu}=\partial_{\mu}\omega^{\nu}-\Gamma^{\nu}_{\mu\lambda}\omega^{\lambda}$$

        \begin{itemize}

          \item Note: $\Gamma$ is not a tensor, do no raise or lower indices!

          \item Covariant refers to transforming like a tensor

        \end{itemize}

        $$\partial_{\mu'}A_{\nu'}=\left( \frac{\partial x^{\mu}}{\partial x^{\mu'}}\partial_{\mu} \right)\left( \frac{\partial x^{\nu}}{\partial x^{\nu'}}\partial_{\nu} \right)=\frac{\partial x^{\mu}}{\partial x^{\mu'}}\frac{\partial x^{\nu}}{\partial x^{\nu'}}\partial_{\mu}A_{\nu}+\frac{\partial x^{\mu}}{\partial x^{\mu'}}A_{\nu}\partial_{\mu}\frac{\partial x^{\nu}}{\partial x^{\nu'}}$$

      \item We can detemine how $\Gamma$ must transform such that $\nabla_{\mu}V^{\nu}$ and $\nabla_{\mu}\omega_{\nu}$ are covariant

      \item Four requirements on covariant derivatives:

        \begin{enumerate}

          \item Linearity: $\nabla(T+S)=\nabla T+\nabla S$

          \item Product Rule: $\nabla(T\otimes S)=\nabla T\otimes S+T\otimes\nabla S$

          \item Commute with Contractions: $\nabla_{\mu}(T^{\lambda}_{\lambda\rho})=(\nabla T)_{\mu}^{\lambda}_{\lambda\rho}$

          \item Partial Derivative on Scalars: $\nabla_{\mu}\phi=\partial_{\mu}\phi$

        \end{enumerate}

      \item For a general $(k,l)$ tensor, we have:

        $$\nabla_{\sigma}T^{\mu_1\ldots\mu_k}_{\nu_1\ldots\nu_l}=\partial_{\sigma T^{\mu_1\ldots\mu_k}_{\nu_1\ldots\nu_l}}+\Gamma^{\mu_1}_{\sigma\lambda}T^{\mu_1\ldots\mu_k}_{\nu_1\ldots\nu_l}+\cdots+\Gamma^{\mu_k}_{\sigma \lambda}T^{\mu_1\ldots\mu_k}_{\nu_1\ldots\nu_l}-\Gamma^{\lambda}_{\sigma\nu_1}T^{\mu_1\ldots\mu_k}_{\nu_1\ldots\nu_l}-\cdots-\Gamma^{\lambda}_{\sigma \nu_l}T^{\mu_1\ldots\mu_k}_{\nu_1\ldots\nu_l}$$

      \item Notation:

        $$\nabla_{\sigma}A_{\mu}=A_{\mu;\sigma}$$
        $$\partial_{\sigma}A_{\mu}=A_{\mu,\sigma}$$

      \item $\Gamma^{\mu}_{\sigma\lambda}$ must obey a particular (non-covariant) transform under coordinate change

      \item We have a lot of freedom to define $\Gamma^{\mu}_{\sigma\lambda}$, but we will impose two additional conditions (for convenience):

        \begin{itemize}

          \item Torsion-free: $T^{\lambda}_{\mu\nu}=\Gamma^{\lambda}_{\mu\nu}-\Gamma^{\lambda}_{\nu\mu}=0\Rightarrow \Gamma^{\lambda}_{\mu\nu}=\Gamma^{\lambda}_{\nu\mu}$

          \item Metric-compatible: $\nabla_{\propto}g_{\mu\nu}=0$

            $$\nabla_{\lambda}\epsilon_{\mu\nu\rho\sigma}=0$$
            $$\nabla_{\rho}g^{\mu\nu}=0$$

        \end{itemize}

      \item Christoffel Symbol:

        $$\Gamma_{\mu\nu}^{\sigma}=\frac{1}{2}g^{\sigma\rho}\left( \partial_{\mu}g_{\nu\rho}+\partial_{\nu}g_{\rho\mu}-\partial_{\rho}g_{\mu\nu} \right)$$

    \end{itemize}

  \item Parallel Transport and Geodesics

    \begin{itemize}

      \item What is parallel transport?

        \begin{itemize}

          \item Keep the vetor fixed as you ``move it.'' This is generally path-dependent

          \item We must do this to compare vectors

          \item Let's be a bit more mathematical:

            $$\frac{d}{d\lambda}V^{\mu}=0$$
            $$\frac{d}{d\lambda}T^{\mu_1\ldots\mu_k}_{\nu_1\ldots\nu_l}=0$$

          \item Before, we had:

            $$\frac{d}{d\lambda}=\frac{dx^{\mu}}{d\lambda}\frac{\partial}{\partial x^{\mu}}$$

          \item Replace with covariant directional derivative:

            $$\frac{D}{d\lambda}\equiv\frac{dx^{\mu}}{d\lambda}\nabla_{\mu}$$

        \end{itemize}

      \item Parallel transport is defined by:

        $$\left( \frac{D}{d\lambda} T \right)^{\mu_1\ldots\mu_k}_{\nu_1\ldots\nu_l}=0$$

    \end{itemize}

  \item Straight Lines $\to$ Geodesics

    \begin{itemize}

      \item Straight line is the shortest distance between two points; in flat space, the direction of your tangent vector doesn't change

      \item We generalize this notion: A geodesic is a path that parallel transports its own tangent vector:

        $$\frac{d^2x^{\mu}}{d\lambda^2}+\Gamma^{\mu}_{\rho\sigma}\frac{dx^{\rho}}{d\lambda}\frac{dx^{\sigma}}{d\lambda}=0\quad\text{ (geodesic equation)}$$

      \item In flat space with Cartesian coordinates, $\Gamma^{\mu}_{\rho\sigma}=0$

        $$\frac{d^2x^{\mu}}{d\lambda^2}=0\quad\text{ (a ``straight'' line)}$$

    \end{itemize}
    
  \item Geodesics as the ``Shortest Distance''

    \begin{itemize}
        
      \item Consider a timelike path; recall proper time along $x^{\mu}(\lambda)$:

        $$d\tau^2=ds^2$$
        $$\tau=\int\left(-g_{\mu\nu}\frac{dx^{\mu}}{d\lambda}\frac{dx^{\nu}}{d\lambda}\right)^{\frac{1}{2}}\,d\lambda$$

      \item Recall (Twin Paradox) that the ``shortest spacetime interval'' corresponds to maximum proper time, due to the minus sign in Lorentzian spacetimes

      \item We want to find the stationary points of $\tau$ (max or min) by varying $x^{\mu}(\lambda)$; the resulting path will end up being (Note: very common method in physics, same as the Lagrangian $\to$ action formulation, which allows the derivation of EoM):

        \begin{itemize}

          \item We can simplify:

            $$f=g_{\mu\nu}(dx^{\mu}/d\lambda)(dx^{\nu}/d\lambda)$$

          \item Variation is:

            $$\delta t=\int \delta\sqrt{-f}\,d\lambda=-\int\frac{1}{2}(-f)^{-\frac{1}{2}}\,\delta f\,d\lambda$$

          \item We can simplify by taking $\lambda=\tau$:

            $$\frac{dx^{\mu}}{d\tau}=U^{\mu}\quad\text{(four velocity)}$$
            $$\Rightarrow f=g_{\mu\nu}U^{\mu}U^{\nu}=-1\quad\text{definition of $\tau$}$$
            $$\delta t=-\frac{1}{2}\int\,\delta f\,d\tau$$

        \end{itemize}

      \item So, we can find the stationary points of:

        $$I=\frac{1}{2}\int f\,d\tau=\frac{1}{2}\int g_{\mu\nu}\frac{dx^{\mu}}{d\tau}\frac{dx^{\nu}}{d\tau d\tau$$

        \item We can use the Euler-Lagrange, or just directly calculate variation:

          $$x^{\mu}\to x^{\mu}+\delta x^{\mu}$$
          $$\Rightarrow g_{\mu\nu}\to g_{\mu\nu}+(\partial_{\sigma}g_{\mu\nu})\delta x^{\sigma}$$
          $$\delta I=I_{new}-I_{old}$$

        \item Bringing all of our terms together, we get:

        $$\delta I=-\int\left[ g_{\mu\sigma}\frac{d^2x^{\mu}}{d\tau^2}+\frac{1}{2}(\partial_{\mu}g_{\nu\sigma}+\partial_{\nu}g_{\sigma\mu}-\partial_{\sigma} g_{\mu\nu}})\frac{dx^{\mu}}{d\tau}\frac{dx^{\nu}}{d\tau} \right]\delta x^{\sigma}\,d\tau$$

        \item Our final form takes:

          $$\frac{d^2x^{\rho}}{d\tau^2}+\underbrace{\frac{1}{2}g^{\rho\sigma}(\partial_{\mu}g_{\nu\sigma}+\partial_{\nu}g_{\sigma\mu}-\partial_{\sigma}g_{\mu\nu})\frac{dx^{\mu}}{d\tau}}_{\Gamma^{\rho}_{\mu\nu}}\frac{dx^{\nu}}{d\tau}=0$$

    \end{itemize}

  \item Affine Parameters

    \begin{itemize}

      \item For timelike geodesics: $\lambda=a\tau+b\quad\quad$ affine parameters

      \item Geodesic equation is satisfied (usually just use $\tau$)

      \item Parallel transport of $dx^{\mu}/d\lambda$ defines both $x^{\mu}$ and $\lambda$

      \item We can express timelike geodesics using $U^{\mu}$ or $p^{\mu}=mU^{\mu}$

      \item For null paths: $\tau=0$

        \begin{itemize}

          \item Instead, we typically choose $p^{\mu}=\frac{dx^{\mu}}{d\lambda}$ to define $\lambda$

          \item In either case, then $E=-p_{\mu}U^{\mu}$

        \end{itemize}

    \end{itemize}

  \item The Riemann Curvature Tensor

    \begin{itemize}

      \item What is curvature at some point $p$?

      \item Flatness, in a slightly more technical sense, means:

        \begin{enumerate}

          \item Parallel transport does not depend on path

            \begin{itemize}

              \item No change on a closed loop

            \end{itemize}

          \item Covariant derivatives commute

          \item Parallel geodesics remain parallel

        \end{enumerate}

      \item Transporting some vector $V^{\rho}$ from point $p$ along a path, we can take, at leading order (fine in a local region) the change will be a linear function of $V^{\rho}$, $A^{\mu}$, and $B^{\nu}$:

        $$\delta V^{\rho}=R^{\rho}_{\sigma\mu\nu}V^{\sigma}A^{\mu}B^{\nu}$$
        
      \item Thus, we see the Riemann Tensor ($R^{\rho}_{\sigma\mu\nu}$)

      \item We can define the Torsion Tensor (will be zero for us) below as $T^{\lambda}_{\mu\nu}$:

        $$[\nabla_{\mu},\nabla_{\nu}]V^{\rho}=R^{\rho}_{\sigma\mu\nu}V^{\sigma}-T^{\lambda}_{\mu\nu}\nabla_{\lambda}V^{\rho}$$

      \item The Riemann Tensor can be defined as:

        $$R^{\rho}_{\sigma\mu\nu}=\partial_{\mu}\Gamma^{\rho}_{\nu\sigma}-\partial_{\nu}\Gamma^{\rho}_{\mu\sigma}+\Gamma^{\rho}_{\mu\lambda}\Gamma^{\lambda}_{\nu\sigma}-\Gamma^{\rho}_{\nu\lambda}\Gamma^{\lambda}_{\mu\sigma}$$

        \begin{itemize}

          \item A tensor, based on non-tensors

          \item Valid for any valid connection, but we will use Christoffel

          \item Has second derivatives of the metric; will not vanish at $p$ for locally inertial coordinates

          \item If the metric is constant in some coordinate system, $R^{\rho}_{\sigma\mu\nu}=0$

          \item If $R^{\rho}_{\sigma\mu\nu}=0$, we can always construct a coordinate system with constant metric (flat)

        \end{itemize}

    \end{itemize}

  \item Properties of the Riemann Tensor:

    $$R_{\rho\sigma\mu\nu}=g_{\rho\alpha}R^{\alpha}_{\sigma\mu\nu}$$

    \begin{itemize}

      \item We can work in locally inertial coordinates to derive generally applicable rules (covariance)

        \begin{enumerate}

          \item $R_{\rho\sigma\mu\nu}=-R_{\sigma\rho\mu\nu}$

          \item $R_{\rho\sigma\mu\nu}=-R_{\rho\sigma\nu\mu}$

          \item $R_{\rho\sigma\mu\nu}=R_{\nu\mu\rho\sigma}$

          \item $R_{\rho\sigma\mu\nu}+R_{\rho\mu\nu\sigma}+R_{\rho\nu\sigma\mu}=0$

          \item $R_{\rho[\sigma\mu\nu]}=0$

          \item $R_{[\rho\sigma\mu\nu]}=0$

        \end{enumerate}

      \item For $n$ dimensions, $n^4$ possible components, but to satisfy these conditions: $(1/12)n^2(n^2-1)=20$ for $n=4$

        $$\nabla_{\hat{\lambda}}R_{\hat{\rho}\hat{\sigma}\hat{\mu}\hat{\nu}}=\partial_{\hat{\lambda}}R_{\hat{\rho}\hat{\sigma}\hat{\mu}\hat{\nu}}$$

        \begin{itemize}

          \item Cycling through first 3 indices:

            $$\nabla_{[\lambda}R_{\rho\sigma]\mu\nu}=0\quad\text{ (Bianchi Identity)}$$

          \item Tells us about derivatives on the surface:

            $$\vec{a}\times(\vec{b}\times\vec{c})+\vec{b}(\vec{c}\times\vec{a})+\vec{c}\times(\vec{a}\times\vec{b})=0\quad\text{ (Jacobi identity)}$$

        \end{itemize}

      \item We want to decomose $R^{\rho}_{\sigma\mu\nu}$ into smaller pieces with more direct physical interpretation

        $$X_{\mu\nu}=X_{(\mu\nu)}+X_{[\mu\nu]}\quad\text{ general (0,2) tensor}$$

      \item We may define the Ricci Tensor:

        $$R_{\mu\nu}=R^{\lambda}_{\mu\lambda\nu}$$

        \begin{itemize}

          \item Performs a contraction (the only independent contraction for Christoffel)

            $$R_{\mu\nu}=R_{\nu\mu}$$

        \end{itemize}

      \item The Ricci Scalar becomes:

        $$R=R^{\mu}_{\mu}=g^{\mu\nu}R_{\mu\nu}$$

      \item This leaves us with the Weyl Tensor (Riemann Tensor with all contractions removed)

        $$C_{\rho\sigma\mu\nu}=R_{\rho\sigma\mu\nu}-\frac{2}{n-2}(g_{\rho[\mu}R_{\nu]\sigma}-g_{\sigma[\mu}R_{\nu]\rho})+\frac{2}{(n-1)(n-2)}g_{\rho[\mu}g_{\nu]\sigma}R$$

        \begin{itemize}

          \item The Ricci Tensor with ``removed contractions''

        \end{itemize}

      \item We can re-express the Bianchi identity:

        $$\nabla^{\mu}R_{\rho\mu}=\frac{1}{2}\nabla_{\rho}R\quad\text{ or } \quad\nabla^{\mu}G_{\mu\nu}=0$$

        \begin{itemize}

          \item Where $G_{\mu\nu}=R_{\mu\nu}-\dfrac{1}{2}Rg_{\mu\nu}$ is the EInstein Tensor!

        \end{itemize}

    \end{itemize}

  \item Geodesic Deviation

    \begin{itemize}

      \item How does curvature impact ``parallel lines'' $\to$ neighboring geodesics? $T^{\mu}$ is tangent and $S^{\mu}$ is the deviation

      \item We can define a relative velocity:

        $$V^{\mu}=(\nabla_T S)^{\mu}=T^{\rho}\nabla_{\rho}S^{\mu}$$

      \item And a relative acceleration to accompany it:

        $$A^{\mu}=(\nabla_T V)^{\mu}=T^{\rho}\nabla_{\rho}V^{\mu}$$

      \item $V^{\mu}=0 \Rightarrow$ initially ``parallel''

      \item Through a bit of algebra, we get:

        $$A^{\mu}=\frac{D^2}{dt^2}S^{\mu}=R^{\mu}_{\nu\rho\sigma}T^{\nu}T^{\rho}S^{\sigma}$$

        \begin{itemize}

          \item Applies to massive particles or photons

          \item An observer may experience weak lensing with smaller particles

          \item An observer may experience strong lensing with more massive particles

        \end{itemize}

    \end{itemize}

  \item Where are we?

    \begin{itemize}

      \item Special relativity: equivalence of inertial frames and constant $c$

      \item Notion of a manifold which has a coordinate system and (co)tangent spaces

      \item Curved spacetime as a manifold

      \item Gravity is universal $\to$ equivalence principle; gravity as a property of spacetime, not a force

      \item The metric measures distances in spacetime ($\eta_{\mu\nu}\to g_{\mu\nu}$)

        \begin{itemize}

          \item Invariant distance $ds^2=g_{\mu\nu}dx^{\mu}dx^{\nu}$

        \end{itemize}

      \item Covariant derivatives require a connection between nearby points on the manifold ($\Gamma \approx (d/dx)g$)

      \item Metric compatible, torsion-free connection: Christoffel

      \item Connection expresses geodesics (straight lines $\leftrightarrow$ parallel transport)

      \item Parallel transport in a loop: Riemann Curvature Tensor

        \begin{itemize}

          \item Ricci Tensor, Ricci Scalar, Weyl Tensor

        \end{itemize}

      \item Energy and Momentum in Curved Spacetime

        $$U^{\mu}=\frac{dx^{\mu}}{d\tau}$$
        $$p^{\mu}=mU^{\mu}$$
        $$T^{\mu\nu}:\,\, p^{\mu}\text{ across }x^{\nu}$$
        $$T^{\mu\nu}=(\rho+p)U^{\mu}U^{\nu}+p\eta^{\mu\nu}$$

    \end{itemize}

  \item Minimal Coupling Principle

    \begin{itemize}

      \item Law of physics in inertial, flat spacetime

      \item Write it in a covariant form

      \item Profit! (assert that this law remains true in curved spacetime)

        $$\eta_{\mu\nu}\to g_{\mu\nu}\quad\text{ and }\partial_{\mu}\nabla_{\mu}$$

        \begin{itemize}

          \item We have already been implicitly thinking this way

        \end{itemize}

      \item Geodesic equation:

        $$\partial_{\mu}T^{\mu\nu}\to \nabla_{\mu}T^{\mu\nu}$$

      \item Does the geodesic equation actually describe gravitational motion in the Newtonian limit?

        \begin{itemize}

          \item Weak field (perturbation of flat space)

          \item Slow moving ($ v<< c$)

          \item Constant in time

            $$v<<c\to \frac{dx^i}{d\tau}<<\frac{dt}{d\tau}$$
            $$\frac{d^2x^{\mu}}{d\tau^2}+\Gamma^{\mu}_{00}\left( \frac{dt}{d\tau} \right)^2$$
            $$\Gamma ^{\sigma}_{\mu\nu}=\frac{1}{2}g^{\sigma\rho}(\partial_{\mu}g_{\nu\rho}+\partial_{\nu}g_{\rho\mu}-\partial_{\rho}g_{\mu\nu})$$

        \end{itemize}

      \item Plugging in the Newtonian limit, we may write:

        $$\frac{d^2x^{\mu}}{d\tau^2}=\frac{1}{2}\eta^{\mu\lambda}\partial_{\lambda}h_{00}\left( \frac{dt}{d\tau} \right)^2$$

    \end{itemize}

  \item Einstein's Equation (1907-1915)

    \begin{itemize}

      \item We want $\nabla^2\Phi=4\pi G\rho$

      \item Trying conservation of energy and combining with the Bianchi Identity, we can get:

        $$\nabla^{\mu}\underbrace{(R_{\mu\mu}-\frac{1}{2}g_{\mu\nu}R)}_{G_{\mu\nu}}=0$$

        \begin{itemize}

          \item Where $G_{\mu\nu}$ is the Einstein Tensor

        \end{itemize}

      \item We can obtain:

        $$R_{\mu\nu}-\frac{1}{2}g_{\mu\nu}R=G_{\mu\nu}=kT_{\mu\nu}$$

      \item Contracting both sides, we can get:

        $$R=-kT$$

        \begin{itemize}

          \item This means:

            $$R_{\mu\nu}=k\left( T_{\mu\nu}-\frac{1}{2}Tg_{\mu\nu} \right)$$

        \end{itemize}

      \item After a series (pun intended) of Taylor expansions, and discarding many unneeded terms, and, finally, combining with Poisson's Equation, we may write:

        $$R_{\mu\nu}-\frac{1}{2}Rg_{\mu\nu}=G_{\mu\nu}=8\pi GT_{\mu\nu}$$

        \begin{itemize}

          \item Equivalently, we may write:

            $$R_{\mu\nu}=8\pi G\left( T_{\mu\nu}-\frac{1}{2}Tg_{\mu\nu} \right)$$

        \end{itemize}

    \end{itemize}

  \item Quantum Gravity

    $$g_{\mu\nu}\to \tilde{g}_{\mu\nu}\,\quad\text{ quantum field}$$

    \begin{itemize}
        
      \item Essentially how QED cam from E\&M

      \item So far, haven't succeeded ``non-renormalizable theory''

    \end{itemize}

\end{itemize}

\end{document}

