%%%%%%%%%%%%%%%%%%%%%%%%%%%%%%%%%%%%%%%%%%%%%%%%%%%%%%%%%%%%%%%%%%%%%%%%%%%%%%%%%%%%%%%%%%%%%%%%%%%%%%%%%%%%%%%%%%%%%%%%%%%%%%%%%%%%%%%%%%%%%%%%%%%%%%%%%%%%%%%%%%%
% Written By Michael Brodskiy
% Class: General Relativity and Cosmology
% Professor: J. Blazek
%%%%%%%%%%%%%%%%%%%%%%%%%%%%%%%%%%%%%%%%%%%%%%%%%%%%%%%%%%%%%%%%%%%%%%%%%%%%%%%%%%%%%%%%%%%%%%%%%%%%%%%%%%%%%%%%%%%%%%%%%%%%%%%%%%%%%%%%%%%%%%%%%%%%%%%%%%%%%%%%%%%

\documentclass[12pt]{article} 
\usepackage{alphalph}
\usepackage[utf8]{inputenc}
\usepackage[russian,english]{babel}
\usepackage{titling}
\usepackage{amsmath}
\usepackage{graphicx}
\usepackage{enumitem}
\usepackage{amssymb}
\usepackage[super]{nth}
\usepackage{everysel}
\usepackage{ragged2e}
\usepackage{geometry}
\usepackage{multicol}
\usepackage{fancyhdr}
\usepackage{cancel}
\usepackage{siunitx}
\usepackage{physics}
\usepackage{tikz}
\usepackage{mathdots}
\usepackage{yhmath}
\usepackage{cancel}
\usepackage{color}
\usepackage{array}
\usepackage{multirow}
\usepackage{gensymb}
\usepackage{tabularx}
\usepackage{extarrows}
\usepackage{booktabs}
\usepackage{lastpage}
\usetikzlibrary{fadings}
\usetikzlibrary{patterns}
\usetikzlibrary{shadows.blur}
\usetikzlibrary{shapes}

\geometry{top=1.0in,bottom=1.0in,left=1.0in,right=1.0in}
\newcommand{\subtitle}[1]{%
  \posttitle{%
    \par\end{center}
    \begin{center}\large#1\end{center}
    \vskip0.5em}%

}
\usepackage{hyperref}
\hypersetup{
colorlinks=true,
linkcolor=blue,
filecolor=magenta,      
urlcolor=blue,
citecolor=blue,
}


\title{Lecture 6 — The Expanding Universe}
\date{\today}
\author{Michael Brodskiy\\ \small Professor: J. Blazek}

\begin{document}

\maketitle

\begin{itemize}

  \item Olber's Paradox: Why is the Sky Dark?

    \begin{itemize}

      \item Absorbing Matter $\to$ doesn't work since matter would heat up

      \item Finite Size

      \item Finite Time (and Finite Speed of Light)

      \item Dimming of Light (``redshift'')

    \end{itemize}

  \item Universe had some beginning (``Big Bang'') around 13.7 billion years ago

  \item Some units:

    $$1[\text{light year}]=9.5\cdot10^{15}[\si{\meter}]$$
    $$1[\text{yr}]\approx \pi\cdot10^{7}[\si{\second}]$$
    $$c\approx 3\cdot10^8\left[ \si{\meter\over\second} \right]$$
    $$1[\si{\parsec}]=3.26[\text{light years}]$$

    \begin{itemize}

      \item If $\theta=1[\text{arcsec}]$, then $d=1[\si{\parsec}]$

        $$1[\si{\parsec}]=2.1\cdot10^5[\text{AU}]$$

    \end{itemize}

  \item The Cosmological Principle

    \begin{itemize}

      \item Copernicus: the Sun, not the Earth, is the center of the Universe

      \item Cosmological Principle: There is no center to the Universe

        \begin{itemize}

          \item The Universe is statistically isotropic (same in all directions) and homogenous (same everywhere)

        \end{itemize}

    \end{itemize}

  \item Expanding Universe

    \begin{itemize}
        
      \item All observers see things moving away from them

      \item Statements are all statistical! Distinguish between structure in the universe and the geometry of the homogenous universe (about 100$[\si{\mega\parsec}]$ scales for homogeneity)

      \item We don't experience the FLRW metric

        \begin{itemize}

          \item Homogenous/geometry

          \item Structure

        \end{itemize}

      \item Conservation of Energy Solution to Expanding Cloud:

        $$\frac{1}{2}\dot{R}^2-\frac{2GM}{R}=C$$

        \begin{itemize}

          \item What is the physical meaning of $C$?

            \begin{itemize}

              \item $C=0$

                You are just at escape velocity. As $R\to\infty$, $\dot{R}=v\to0$. Potential and kinetic both go to zero.
                
              \item $C>0$

                Positive total energy. You have more than enough energy to escape. $\dot{R}>0$ as $R\to\infty$
                
              \item $C<0$

                Negative total energy. You won't make it out to $R=\infty$, you will stop and turn around at some finite $t$
                
            \end{itemize}

          \item These cases capture an ideal universe's expansion with only real matter

          \item For a matter-only universe, $C$ describes the spatial geometry, with zero indicating flat, $C>0$ indicating open (negative curvature), and $C<0$ indicating closed (positive curvature)

            \begin{itemize}

              \item Note that $\Lambda$ complicates this

            \end{itemize}

        \end{itemize}

      \item Using the first-order equation, we may write:

        $$\frac{1}{2}\dot{R}^2=\frac{4}{3}\pi G\rho R^2+C$$
        $$\left(\frac{\dot{R}}{R}\right)=\frac{8\pi G\rho}{3}+C$$

      \item At $C=0$, we find $\rho_{crit}$:

        $$\rho_{crit}=\frac{3H^2}{8\pi G}$$

        Note that we define the Hubble parameter as:

        $$H=\frac{\dot{R}}{R}$$

    \end{itemize}

  \item Comoving Coordinates

    \begin{itemize}

      \item Coordinates (and distances) scale with the size of the universe:

        $$x=a(t)r$$

        \begin{itemize}

          \item $x$ is the ``proper'' coordinate

          \item $r$ is the ``comoving'' coordinate

          \item $a(t)$ is the scale factor, with $a(t_o)=1$ indicating ``today''

        \end{itemize}

      \item Ex. $x_{12}=a(t)r_{12}$

        $$\frac{dx_{12}}{dt}\quad\text{ is the recession velocity}$$

      \item Two Things:

        \begin{enumerate}

          \item Velocity is proportional to distance (Hubble law)

          \item The proportionality term is $\dfrac{\dot{a}}{a}=H(t)$

        \end{enumerate}

      \item Hubble constant is approximately:

        $$H_o=70\left[ \si{\kilo\meter\over\second\mega\parsec} \right]$$

        \begin{itemize}

          \item We define $h$, such that:

            $$H_o=100h\left[ \si{\kilo\meter\over\second\mega\parsec} \right]$$

          \item We can see that the units are actually just $\si{\hertz}$, which gives us:

          $$H_o\approx 3.24\cdot10^{-18}\left[\si{1\over\second}\right]$$

        \end{itemize}

      \item If $a$ is constant, this gives us the doubling time or the time to go back to $a=0$, which gives us the ``Hubble'' time, or the age of the universe

      \item For $h=.7\to 1.4\cdot10^{10}\left[ \frac{1}{\text{yr}} \right]$

        \begin{itemize}

          \item Fairly accurate approximation of 14 billion years

        \end{itemize}

    \end{itemize}

  \item FLRW Metric

    \begin{itemize}

      \item Cosmological principle tells us that spacetime (on large scales) should be homogenous (translation invariant) and isotropic (rotation invariant)

      \item We seek a general form of a metric that obeys these assumptions:

        $$ds^2=\bar{c}^2\,dt^2+R^2(t)\,d\sigma^2$$

      \item We are able to derive:

        $$d\sigma^2=\frac{d\bar{r}^2}{1-k\bar{r}^2}+\bar{r}^2\,d\Omega^2$$

      \item Where:

        \begin{itemize}

          \item $d\Omega^2$ is the 2-sphere metric

          \item $k\propto R$ (Ricci scalar on 3D space)

            $$k=\left\{\begin{array}{ll}+1,& \text{Positive Curvature (3-sphere), "closed"}\\ 0,& \text{Flat}\\ -1,&\text{Negative curvature (saddle), "open"}\end{array}$$

        \end{itemize}

      \item Thus, we may define the metric as:

        $$ds^2=-dt^2+a^2(t)\left[ \frac{dr^2}{1-\kappa r^2}+r^2\,d\Omega^2 \right]$$

        \begin{itemize}

          \item Where:

            $$\left\{\begin{array}{ll}\kappa>0,&\text{closed}\\\kappa=0,&\text{flat}\\\kappa<0,&\text{open}\end{array}$$

        \end{itemize}

    \end{itemize}

  \item Cosmological Redshift

    \begin{itemize}

      \item Like a D\"oppler shift, but use caution!

      \item $v=H_od$, locally $v<<c$

      \item Redshift in special relativity:

        $$\frac{\lambda_{obs}}{\lambda_{em}}=\sqrt{\frac{1+v/c}{1-v/c}}\approx1+\frac{v}{c}$$

      \item From this, we know:

        $$\frac{\lambda_{obs}}{\lambda_{em}}=1+z\approx 1+\frac{H_o d}{c}$$

      \item More generally, we may say (for $a_o=$ today):

        $$1+z=\frac{a_{obs}}{a_{em}}=a_{em}^{-1}$$

    \end{itemize}

  \item Cosmological Redshift versus Peculiar Motions:

    \begin{itemize}

      \item For comoving coordinates:

        $$v\approx H_od=H_o|\vec{r}_2-\vec{r}_1|/a_o=H_o|\vec{r}_2-\vec{r}_1|$$
        
      \item For ``peculiar'' motions

        $$v\approx H_od+\frac{\Delta v}{c}$$

      \item In the FLRW metric, due to lack of simple time symmetry, energy is not conserved

        \begin{itemize}

          \item There is a Killing Tensor that reflects a symmetry:

            $$K_{\mu\nu}=a^2(g_{\mu\nu}+U_{\mu}U_{\nu})\quad\text{ for observer (comoving) $U^{\mu}=(1,0,0,0)$}$$

          \item We get:

            $$K^2=K_{\mu\nu}V^{\mu}V^{\nu}$$

            \begin{itemize}

              \item Is conserved, with

                $$V^{\mu}=\frac{dx^{\mu}}{d\lambda}$$

              \item For a photon on a null geodesic:

                $$V_{\mu}V^{\mu}=0$$

            \end{itemize}

          \item This can be simplified to:

            $$v=\frac{K}{a}$$

        \end{itemize}

    \end{itemize}

  \item ``Stuff'' That Can fill A Universe

    \begin{itemize}

      \item ``Baryons'' $\to$ All standard model particles with mass (interact with light, gravity)

        \begin{itemize}

          \item Non-photon force carriers

          \item ``In a box'' with scale factor $a$ scale proportionally to $a^{-3}$

        \end{itemize}

      \item Dark matter $\to$ Only interacts through gravity (?)

        \begin{itemize}

          \item Neutrinos (?)

          \item New particles

          \item ``Cold'' dark matter (non-relativistic), scales ``in a box'' with scale factor $a$ proportionally to $a^{-3}$

        \end{itemize}

      \item Dark energy (``cosmological constant'')

      \item Radiation

        \begin{itemize}

          \item Photons

            \begin{itemize}

              \item ``In a box'' with scale factor $a$ scale proportionally to $a^{-4}$

            \end{itemize}

          \item Relativistic Particles

        \end{itemize}

      \item Black Holes (acts more or less like dark matter, but forms in other ways)

      \item Cosmological constant ``in a box'' does not scale; that is, it is a constant, so it does not change

    \end{itemize}

  \item Curvature in the Universe

  \item Entropy in the Universe

    \begin{itemize}

      \item ``Information''

    \end{itemize}

  \item Formal Momenta Redshift:

    \begin{itemize}

      \item Photon redshift $E\propto \dfrac{1}{a},\, \lambda\propto a$

    \end{itemize}

  \item Energy Evolution

    \begin{itemize}

      \item We know: $T_{\mu\nu}=(\rho+p)U_{\mu}U_{\nu}+pg_{\mu\nu}$

      \item Working in a fluid's rest frame, $U^{\mu}=(1,0,0,0)$, which gives us:

        $$T_{\mu\nu}=\left[ \begin{matrix}\rho & 0 & 0 & 0\\ 0 & 0 & 0 & 0\\ 0 & 0 & 0 & 0\\ 0 & 0 & 0 & 0 \end{matrix} \right]$$

      \item Dark Energy: For $\Lambda$, $\rho_{\Lambda}=$ constant

      \item We may obtain, from a formal General Relativity derivation:

        $$\rho=\rho_oa^{-3(1+w)}$$

      \item We may also obtain:

        $$H^2(a)=\left( \frac{\dot{a}}{a} \right)^2=\frac{8\pi G}{3}\left[ \rho_{m,o}a^{-3}+\rho_{r,o}a^{-4}+\rho_{\lambda,o}-2\kappa a^{-2} \right]$$

        \begin{itemize}

          \item Note that, once we have $\Lambda$, a closed universe won't recollapse

        \end{itemize}

      \item From the critical density, we write the ratio as:

        $$\Omega_{i,o}=\frac{\rho_{ip}}{\rho_{crit}}$$

      \item This allows us to rewrite the Hubble parameter as:

        $$\frac{H(a)}{H_o}=\left( \Omega_{m}a^{-3}+\Omega_ra^{-4}+\Omega_{\Lambda}+\Omega_{\kappa}a^{-2} \right)^{\frac{1}{2}}$$

      \item We can then write in terms of the redshift:

        $$\frac{H(z)}{H_o}=\left( \Omega_{m}(1+z)^{3}+\Omega_r(1+z)^{4}+\Omega_{\Lambda}+\Omega_{\kappa}(1+z)^{2} \right)^{\frac{1}{2}}$$

      \item Furthermore, we see:

        $$\Omega_{m}+\Omega_r+\Omega_{\Lambda}+\Omega_{\kappa}=1$$

      \item There are several important naming conventions:

        \begin{itemize}

          \item $\Omega_{m}=1$: Flat, matter-only ``Einstein-deSitter Universe''

          \item $\Omega_{\kappa}=0$: Flat

          \item $\Omega_{\Lambda}=1$: deSitter Universe

          \item $\Omega_{\Lambda}=-1$: Anti-deSitter Universe

          \item $\Omega_{\kappa}=1$: Empty

        \end{itemize}

    \end{itemize}

  \item How Do We Measure Distances?

    \begin{itemize}
        
      \item Angular Size (Object seems larger if it is closer)

      \item Noise (Object seems louder if it is closer)

      \item Light (Object seems brighter if it is closer)

    \end{itemize}

  \item Distances:

    \begin{itemize}
        
      \item Comoving (radial) distance: follow a photon as it travels from a distant source

        $$ds=0 \to dt=a(t)\frac{dr}{\left( 1-\kappa r^2 \right)^{\frac{1}{2}}}$$
        $$\chi=\int_{t_{em}}^{t_o}\frac{dt}{a(t)}=\int_0^r \frac{dr'}{\left( 1-\kappa r'^2 \right)^{\frac{1}{2}}}$$

      \item We will redefine the $r$ coordinate such that $\chi$ is always $r$:

        $$ds^2=-dt^2+a^2(t)\left[ dr^2+S^{2}_{k}(r)d\Omega^2 \right]$$

        \begin{itemize}

          \item The term becomes:

            $$S_{\kappa}(r)=\left\{\begin{array}{ll} r,& \Omega_{\kappa}=0\quad\text{ (flat)}\\H_o^{-1}(\Omega_{\kappa})^{-\frac{1}{2}}\sinh\left( H_o(\Omega_{\kappa})^{\frac{1}{2}}r \right),& \Omega_{\kappa}>0\quad\text{ (open)}\\H_o^{-1}(|\Omega_{\kappa}|)^{-\frac{1}{2}}\sin\left( H_o(-\Omega_{\kappa})^{\frac{1}{2}}r \right),& \Omega_{\kappa}<0\quad\text{ (closed)}\end{array}$$

        \end{itemize}

      \item We may rewrite $\chi$ in terms of $a$ to see:

        $$\chi(a)=\int_{a(t_{em})}^{a(t_o)=1}\frac{da}{a^2H(a)}$$

    \end{itemize}

  \item The Horizon:

    $$\chi_{hor}=\int_0^1\frac{da}{a^2H(a)}\xlongrightarrow{\text{EdS Universe}}\int_0^1\frac{da}{a^2H_oa^{-\frac{3}{2}}}=\frac{2}{H_o}$$

    \begin{itemize}
        
      \item Note if we restore $c$:

        $$\chi_{hor}=\frac{2c}{H_o}$$

    \end{itemize}
    
  \item Age of the Universe

    \begin{itemize}

      \item We imagine an empty universe ($\Omega_{\kappa}=1$):

        $$\int_0^1\,da=\int_0^{t_o}H_o\,dt$$
        $$1=t_oH_o$$
        $$t_o=\frac{1}{H_o}\quad\text{ (Hubble time)}$$

      \item Similarly, for a matter-dominated universe ($\Omega_m=1$):

        $$\int_0^1 a^{\frac{1}{2}}\,da=\int_0^{t_o}H_o\,dt$$
        $$\frac{2}{3}a^{\frac{3}{2}}=t_oH_o$$
        $$t_o=\frac{2}{3H_o}$$

    \end{itemize}

  \item Luminosity Distance

    \begin{itemize}

      \item ``Standard Candles''

      \item Luminosity is Energy per time

      \item Flux is the Energy per Area per time

      \item In normal three dimensions:

        $$L=F(4\pi R^2)$$

        \begin{itemize}

          \item This means:

            $$F\propto \frac{L}{R^2}$$

        \end{itemize}

      \item In comoving coordinates, we may write:

        $$F\propto\frac{L}{\chi^2}\cdot\frac{1}{1+z}\cdot\frac{1}{1+z}$$

      \item We may thus find that, in a flat universe:

        $$d_L=\chi(1+z)$$

      \item In a non-flat universe, we see:

        $$d_L=S_{\kappa}(\chi)(1+z)$$

    \end{itemize}

  \item Angular Diameter Distance

    \begin{itemize}

      \item Object of physical size $l$

        \begin{itemize}

          \item Assume a flat universe:

            $$l=d_A\theta$$
            $$d_A^{flat}=\chi a=\frac{\chi}{1+z}=\frac{d_L}{(1+z)^2}$$
            $$d_A=aS_{\kappa}(\chi)$$

          \item Importantly, $d_A$ can increase, reach a max, and then decrease

            \begin{itemize}

              \item This means physical $l$ takes up larger fraction of a smaller universe

            \end{itemize}

          \item This is slightly more complicated in a non-flat universe, since we need to properly account for $d\Omega$ factor

        \end{itemize}

    \end{itemize}

  \item Evidence for Dark Energy

    \begin{itemize}

      \item Until the early 1990's, people did not take $\Lambda$ too seriously. Evidence from galaxy clusters, galaxy clustering, and CMB started to raise some questions. Universe appeared roughly flat, but $\Omega_m<1$

      \item Mapping out $a(t)$ has been a useful way to probe the universe

    \end{itemize}

  \item Dark Matter

    \begin{itemize}

      \item In 1781, William Herschel discovered Uranus; over the next 60 years, astronomers carefully mapped out it orbit, but it didn't quite match Newtonian theory

      \item Urbain Le Verrier: Showed that Uranus' orbit could be explained if there were another, more distant planet acting gravitationally

      \item In September 1846, he mailed a letter to a colleague at the Berlin Observatory with precise predictions

      \item In 1930s, Fritz Zwicky studied the Coma Cluster

        \begin{itemize}

          \item Galaxy Clusters: Largest gravitationally bound objects in the Universe

        \end{itemize}

      \item Zwicky measured spectra of many galaxies and calculated the velocity dispersion

      \item With a quick calculation, the orbital velocity of a galazy cluster may be expressed as:

        $$M=\frac{r_cv^2}{\alpha G}$$

      \item MACHOs (Massive Compact Halo Objects)

        \begin{itemize}
            
          \item Planet or asteroid-sized objects: ``microlensing'' constraints mostly rule this out

        \end{itemize}

      \item Axions

      \item Weakly-Interacting Massive Particles (WIMPs)

      \item Primordial Black Holes (PBHs)

      \item CDM: Cold Dark Matter $\to \Lambda$CDM model

      \item Can generalize $w$CDM such that:

        $$p_{DE}=w\rho_{DE}$$

        \begin{itemize}

          \item $\Lambda$: $w=-1$

            $$w(a)=w_o+w_a(a-a_o)$$

        \end{itemize}

    \end{itemize}

  \item What is Dark Energy?

    \begin{itemize}

      \item ``Cosmological Constant Problem''

      \item Universe expansion is accelerating; $\Lambda$ can do this

      \item Quantum field theory says that vacuum fluctuations have an energy density

    \end{itemize}

  \item Hot Big Bang

    \begin{itemize}

      \item Thermal Equilibrium

        $$\frac{n_1}{n_2}=e^{-(E_1-E_2)/kT}$$

        \begin{itemize}

          \item State 1: $E_1$

          \item State 2: $E_2$

        \end{itemize}

    \end{itemize}

  \item Photons

    \begin{itemize}

      \item Photons are bosons

      \item This means they follow the Bose-Einstein distribution:

        $$\bar{n}_i=\frac{g_i}{e^{(\epsilon_i-\mu)/kT}-1}\Rightarrow \rho_i=\bar{n}_i\epsilon_i$$

        \begin{itemize}

          \item $g_i$ is the ``degeneracy''

          \item This expresses how many particles are in a given energy state $i$

        \end{itemize}

      \item To calculate the energy density, we need to consider the ``phase space'' $\left\{ \vec{x},\vec{p} \right\}$ and integrate over all states with a given $|p|$

      \item Heisenberg's uncertainty principle: $d^3xd^3p$ has $\dfrac{d^3xd^3p}{(2\pi\hbar)^3}$ phase space elements. For photons, $\mu=0$:

        $$\rho=\int\frac{d^3p}{(2\pi\hbar)^3}\frac{2p}{e^{p/kT}-1}$$

        \begin{itemize}

          \item We may obtain:

            $$\rho(f)\,df=\frac{8\pi h}{c^3}\left( \frac{f^3\,df}{e^{hf/kT}-1} \right)$$

            \item For photons between $f$ and $f+df$

        \end{itemize}

      \item Blackbody Spectrum

        \begin{itemize}

          \item Blackbody: Perfectly absorptive system which emits a spectrum given by photon thermal equilibrium

            $$n_c(f_c)=\frac{8\pi}{c^3}\frac{f_c^2}{e^{hf_c/akT}-1}$$

            \begin{itemize}

              \item Preserved if $T\to T/a$

            \end{itemize}

        \end{itemize}

      \item Temperature of the universe is really defined by the distribution of particles in thermal equilibrium

    \end{itemize}

  \item Cosmic Microwave Background

    \begin{itemize}
        
      \item In an expanding universe, we required:

        $$\Gamma>H$$

        \begin{itemize}

          \item $\Gamma\propto n_2v_1\sigma_{12}$

          \item $H$ is the Hubble rate, which is the inverse of Hubble time

          \item ``Freeze-out'' $n_i$ is fixed

        \end{itemize}

    \end{itemize}

  \item Big Bang Nucleosynthesis: ``The First Three Minutes''

    \begin{itemize}

      \item Fission versus Fusion

      \item Binding energy: Energy to break apart nucleus or, equivalentl, energy tou release forming the nucleus

    \end{itemize}

\end{itemize}

\end{document}

