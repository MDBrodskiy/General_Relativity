%%%%%%%%%%%%%%%%%%%%%%%%%%%%%%%%%%%%%%%%%%%%%%%%%%%%%%%%%%%%%%%%%%%%%%%%%%%%%%%%%%%%%%%%%%%%%%%%%%%%%%%%%%%%%%%%%%%%%%%%%%%%%%%%%%%%%%%%%%%%%%%%%%%%%%%%%%%%%%%%%%%
% Written By Michael Brodskiy
% Class: General Relativity and Cosmology
% Professor: J. Blazek
%%%%%%%%%%%%%%%%%%%%%%%%%%%%%%%%%%%%%%%%%%%%%%%%%%%%%%%%%%%%%%%%%%%%%%%%%%%%%%%%%%%%%%%%%%%%%%%%%%%%%%%%%%%%%%%%%%%%%%%%%%%%%%%%%%%%%%%%%%%%%%%%%%%%%%%%%%%%%%%%%%%

\include{Includes.tex}

\title{Lecture 6 — The Expanding Universe}
\date{\today}
\author{Michael Brodskiy\\ \small Professor: J. Blazek}

\begin{document}

\maketitle

\begin{itemize}

  \item Olber's Paradox: Why is the Sky Dark?

    \begin{itemize}

      \item Absorbing Matter $\to$ doesn't work since matter would heat up

      \item Finite Size

      \item Finite Time (and Finite Speed of Light)

      \item Dimming of Light (``redshift'')

    \end{itemize}

  \item Universe had some beginning (``Big Bang'') around 13.7 billion years ago

  \item Some units:

    $$1[\text{light year}]=9.5\cdot10^{15}[\si{\meter}]$$
    $$1[\text{yr}]\approx \pi\cdot10^{7}[\si{\second}]$$
    $$c\approx 3\cdot10^8\left[ \si{\meter\over\second} \right]$$
    $$1[\si{\parsec}]=3.26[\text{light years}]$$

    \begin{itemize}

      \item If $\theta=1[\text{arcsec}]$, then $d=1[\si{\parsec}]$

        $$1[\si{\parsec}]=2.1\cdot10^5[\text{AU}]$$

    \end{itemize}

  \item The Cosmological Principle

    \begin{itemize}

      \item Copernicus: the Sun, not the Earth, is the center of the Universe

      \item Cosmological Principle: There is no center to the Universe

        \begin{itemize}

          \item The Universe is statistically isotropic (same in all directions) and homogenous (same everywhere)

        \end{itemize}

    \end{itemize}

  \item Expanding Universe

    \begin{itemize}
        
      \item All observers see things moving away from them

      \item Statements are all statistical! Distinguish between structure in the universe and the geometry of the homogenous universe (about 100$[\si{\mega\parsec}]$ scales for homogeneity)

      \item We don't experience the FLRW metric

        \begin{itemize}

          \item Homogenous/geometry

          \item Structure

        \end{itemize}

      \item Conservation of Energy Solution to Expanding Cloud:

        $$\frac{1}{2}\dot{R}^2-\frac{2GM}{R}=C$$

        \begin{itemize}

          \item What is the physical meaning of $C$?

            \begin{itemize}

              \item $C=0$

                You are just at escape velocity. As $R\to\infty$, $\dot{R}=v\to0$. Potential and kinetic both go to zero.
                
              \item $C>0$

                Positive total energy. You have more than enough energy to escape. $\dot{R}>0$ as $R\to\infty$
                
              \item $C<0$

                Negative total energy. You won't make it out to $R=\infty$, you will stop and turn around at some finite $t$
                
            \end{itemize}

          \item These cases capture an ideal universe's expansion with only real matter

          \item For a matter-only universe, $C$ describes the spatial geometry, with zero indicating flat, $C>0$ indicating open (negative curvature), and $C<0$ indicating closed (positive curvature)

            \begin{itemize}

              \item Note that $\Lambda$ complicates this

            \end{itemize}

        \end{itemize}

      \item Using the first-order equation, we may write:

        $$\frac{1}{2}\dot{R}^2=\frac{4}{3}\pi G\rho R^2+C$$
        $$\left(\frac{\dot{R}}{R}\right)=\frac{8\pi G\rho}{3}+C$$

      \item At $C=0$, we find $\rho_{crit}$:

        $$\rho_{crit}=\frac{3H^2}{8\pi G}$$

        Note that we define the Hubble parameter as:

        $$H=\frac{\dot{R}}{R}$$

    \end{itemize}

  \item Comoving Coordinates

    \begin{itemize}

      \item Coordinates (and distances) scale with the size of the universe:

        $$x=a(t)r$$

        \begin{itemize}

          \item $x$ is the ``proper'' coordinate

          \item $r$ is the ``comoving'' coordinate

          \item $a(t)$ is the scale factor, with $a(t_o)=1$ indicating ``today''

        \end{itemize}

      \item Ex. $x_{12}=a(t)r_{12}$

        $$\frac{dx_{12}}{dt}\quad\text{ is the recession velocity}$$

      \item Two Things:

        \begin{enumerate}

          \item Velocity is proportional to distance (Hubble law)

          \item The proportionality term is $\dfrac{\dot{a}}{a}=H(t)$

        \end{enumerate}

      \item Hubble constant is approximately:

        $$H_o=70\left[ \si{\kilo\meter\over\second\mega\parsec} \right]$$

        \begin{itemize}

          \item We define $h$, such that:

            $$H_o=100h\left[ \si{\kilo\meter\over\second\mega\parsec} \right]$$

          \item We can see that the units are actually just $\si{\hertz}$, which gives us:

          $$H_o\approx 3.24\cdot10^{-18}\left[\si{1\over\second}\right]$$

        \end{itemize}

      \item If $a$ is constant, this gives us the doubling time or the time to go back to $a=0$, which gives us the ``Hubble'' time, or the age of the universe

      \item For $h=.7\to 1.4\cdot10^{10}\left[ \frac{1}{\text{yr}} \right]$

        \begin{itemize}

          \item Fairly accurate approximation of 14 billion years

        \end{itemize}

    \end{itemize}

  \item FLRW Metric

    \begin{itemize}

      \item Cosmological principle tells us that spacetime (on large scales) should be homogenous (translation invariant) and isotropic (rotation invariant)

      \item We seek a general form of a metric that obeys these assumptions:

        $$ds^2=\bar{c}^2\,dt^2+R^2(t)\,d\sigma^2$$

      \item We are able to derive:

        $$d\sigma^2=\frac{d\bar{r}^2}{1-k\bar{r}^2}+\bar{r}^2\,d\Omega^2$$

      \item Where:

        \begin{itemize}

          \item $d\Omega^2$ is the 2-sphere metric

          \item $k\propto R$ (Ricci scalar on 3D space)

            $$k=\left\{\begin{array}{ll}+1,& \text{Positive Curvature (3-sphere), "closed"}\\ 0,& \text{Flat}\\ -1,&\text{Negative curvature (saddle), "open"}\end{array}$$

        \end{itemize}

      \item Thus, we may define the metric as:

        $$ds^2=-dt^2+a^2(t)\left[ \frac{dr^2}{1-\kappa r^2}+r^2\,d\Omega^2 \right]$$

        \begin{itemize}

          \item Where:

            $$\left\{\begin{array}{ll}\kappa>0,&\text{closed}\\\kappa=0,&\text{flat}\\\kappa<0,&\text{open}\end{array}$$

        \end{itemize}

    \end{itemize}

  \item Cosmological Redshift

    \begin{itemize}

      \item Like a D\"oppler shift, but use caution!

      \item $v=H_od$, locally $v<<c$

      \item Redshift in special relativity:

        $$\frac{\lambda_{obs}}{\lambda_{em}}=\sqrt{\frac{1+v/c}{1-v/c}}\approx1+\frac{v}{c}$$

      \item From this, we know:

        $$\frac{\lambda_{obs}}{\lambda_{em}}=1+z\approx 1+\frac{H_o d}{c}$$

      \item More generally, we may say (for $a_o=$ today):

        $$1+z=\frac{a_{obs}}{a_{em}}=a_{em}^{-1}$$

    \end{itemize}

  \item Cosmological Redshift versus Peculiar Motions:

    \begin{itemize}

      \item For comoving coordinates:

        $$v\approx H_od=H_o|\vec{r}_2-\vec{r}_1|/a_o=H_o|\vec{r}_2-\vec{r}_1|$$
        
      \item For ``peculiar'' motions

        $$v\approx H_od+\frac{\Delta v}{c}$$

      \item In the FLRW metric, due to lack of simple time symmetry, energy is not conserved

        \begin{itemize}

          \item There is a Killing Tensor that reflects a symmetry:

            $$K_{\mu\nu}=a^2(g_{\mu\nu}+U_{\mu}U_{\nu})\quad\text{ for observer (comoving) $U^{\mu}=(1,0,0,0)$}$$

          \item We get:

            $$K^2=K_{\mu\nu}V^{\mu}V^{\nu}$$

            \begin{itemize}

              \item Is conserved, with

                $$V^{\mu}=\frac{dx^{\mu}}{d\lambda}$$

              \item For a photon on a null geodesic:

                $$V_{\mu}V^{\mu}=0$$

            \end{itemize}

          \item This can be simplified to:

            $$v=\frac{K}{a}$$

        \end{itemize}

    \end{itemize}

  \item ``Stuff'' That Can fill A Universe

    \begin{itemize}

      \item ``Baryons'' $\to$ All standard model particles with mass (interact with light, gravity)

        \begin{itemize}

          \item Non-photon force carriers

          \item ``In a box'' with scale factor $a$ scale proportionally to $a^{-3}$

        \end{itemize}

      \item Dark matter $\to$ Only interacts through gravity (?)

        \begin{itemize}

          \item Neutrinos (?)

          \item New particles

          \item ``Cold'' dark matter (non-relativistic), scales ``in a box'' with scale factor $a$ proportionally to $a^{-3}$

        \end{itemize}

      \item Dark energy (``cosmological constant'')

      \item Radiation

        \begin{itemize}

          \item Photons

            \begin{itemize}

              \item ``In a box'' with scale factor $a$ scale proportionally to $a^{-4}$

            \end{itemize}

          \item Relativistic Particles

        \end{itemize}

      \item Black Holes (acts more or less like dark matter, but forms in other ways)

      \item Cosmological constant ``in a box'' does not scale; that is, it is a constant, so it does not change

    \end{itemize}

  \item Curvature in the Universe

  \item Entropy in the Universe

    \begin{itemize}

      \item ``Information''

    \end{itemize}

  \item Formal Momenta Redshift:

    \begin{itemize}

      \item Photon redshift $E\propto \dfrac{1}{a},\, \lambda\propto a$

    \end{itemize}

  \item Energy Evolution

    \begin{itemize}

      \item We know: $T_{\mu\nu}=(\rho+p)U_{\mu}U_{\nu}+pg_{\mu\nu}$

      \item Working in a fluid's rest frame, $U^{\mu}=(1,0,0,0)$, which gives us:

        $$T_{\mu\nu}=\left[ \begin{matrix}\rho & 0 & 0 & 0\\ 0 & 0 & 0 & 0\\ 0 & 0 & 0 & 0\\ 0 & 0 & 0 & 0 \end{matrix} \right]$$

      \item Dark Energy: For $\Lambda$, $\rho_{\Lambda}=$ constant

      \item We may obtain, from a formal General Relativity derivation:

        $$\rho=\rho_oa^{-3(1+w)}$$

      \item We may also obtain:

        $$H^2(a)=\left( \frac{\dot{a}}{a} \right)^2=\frac{8\pi G}{3}\left[ \rho_{m,o}a^{-3}+\rho_{r,o}a^{-4}+\rho_{\lambda,o}-2\kappa a^{-2} \right]$$

        \begin{itemize}

          \item Note that, once we have $\Lambda$, a closed universe won't recollapse

        \end{itemize}

      \item From the critical density, we write the ratio as:

        $$\Omega_{i,o}=\frac{\rho_{ip}}{\rho_{crit}}$$

      \item This allows us to rewrite the Hubble parameter as:

        $$\frac{H(a)}{H_o}=\left( \Omega_{m}a^{-3}+\Omega_ra^{-4}+\Omega_{\Lambda}+\Omega_{\kappa}a^{-2} \right)^{\frac{1}{2}}$$

      \item We can then write in terms of the redshift:

        $$\frac{H(z)}{H_o}=\left( \Omega_{m}(1+z)^{3}+\Omega_r(1+z)^{4}+\Omega_{\Lambda}+\Omega_{\kappa}(1+z)^{2} \right)^{\frac{1}{2}}$$

      \item Furthermore, we see:

        $$\Omega_{m}+\Omega_r+\Omega_{\Lambda}+\Omega_{\kappa}=1$$

      \item There are several important naming conventions:

        \begin{itemize}

          \item $\Omega_{m}=1$: Flat, matter-only ``Einstein-deSitter Universe''

          \item $\Omega_{\kappa}=0$: Flat

          \item $\Omega_{\Lambda}=1$: deSitter Universe

          \item $\Omega_{\Lambda}=-1$: Anti-deSitter Universe

          \item $\Omega_{\kappa}=1$: Empty

        \end{itemize}

    \end{itemize}

  \item How Do We Measure Distances?

    \begin{itemize}
        
      \item Angular Size (Object seems larger if it is closer)

      \item Noise (Object seems louder if it is closer)

      \item Light (Object seems brighter if it is closer)

    \end{itemize}

  \item Distances:

    \begin{itemize}
        
      \item Comoving (radial) distance: follow a photon as it travels from a distant source

        $$ds=0 \to dt=a(t)\frac{dr}{\left( 1-\kappa r^2 \right)^{\frac{1}{2}}}$$
        $$\chi=\int_{t_{em}}^{t_o}\frac{dt}{a(t)}=\int_0^r \frac{dr'}{\left( 1-\kappa r'^2 \right)^{\frac{1}{2}}}$$

      \item We will redefine the $r$ coordinate such that $\chi$ is always $r$:

        $$ds^2=-dt^2+a^2(t)\left[ dr^2+S^{2}_{k}(r)d\Omega^2 \right]$$

        \begin{itemize}

          \item The term becomes:

            $$S_{\kappa}(r)=\left\{\begin{array}{ll} r,& \Omega_{\kappa}=0\quad\text{ (flat)}\\H_o^{-1}(\Omega_{\kappa})^{-\frac{1}{2}}\sinh\left( H_o(\Omega_{\kappa})^{\frac{1}{2}}r \right),& \Omega_{\kappa}>0\quad\text{ (open)}\\H_o^{-1}(|\Omega_{\kappa}|)^{-\frac{1}{2}}\sin\left( H_o(-\Omega_{\kappa})^{\frac{1}{2}}r \right),& \Omega_{\kappa}<0\quad\text{ (closed)}\end{array}$$

        \end{itemize}

      \item We may rewrite $\chi$ in terms of $a$ to see:

        $$\chi(a)=\int_{a(t_{em})}^{a(t_o)=1}\frac{da}{a^2H(a)}$$

    \end{itemize}

  \item The Horizon:

    $$\chi_{hor}=\int_0^1\frac{da}{a^2H(a)}\xlongrightarrow{\text{EdS Universe}}\int_0^1\frac{da}{a^2H_oa^{-\frac{3}{2}}}=\frac{2}{H_o}$$

    \begin{itemize}
        
      \item Note if we restore $c$:

        $$\chi_{hor}=\frac{2c}{H_o}$$

    \end{itemize}
    
  \item Age of the Universe

    \begin{itemize}

      \item We imagine an empty universe ($\Omega_{\kappa}=1$):

        $$\int_0^1\,da=\int_0^{t_o}H_o\,dt$$
        $$1=t_oH_o$$
        $$t_o=\frac{1}{H_o}\quad\text{ (Hubble time)}$$

      \item Similarly, for a matter-dominated universe ($\Omega_m=1$):

        $$\int_0^1 a^{\frac{1}{2}}\,da=\int_0^{t_o}H_o\,dt$$
        $$\frac{2}{3}a^{\frac{3}{2}}=t_oH_o$$
        $$t_o=\frac{2}{3H_o}$$

    \end{itemize}

  \item Luminosity Distance

    \begin{itemize}

      \item ``Standard Candles''

      \item Luminosity is Energy per time

      \item Flux is the Energy per Area per time

      \item In normal three dimensions:

        $$L=F(4\pi R^2)$$

        \begin{itemize}

          \item This means:

            $$F\propto \frac{L}{R^2}$$

        \end{itemize}

      \item In comoving coordinates, we may write:

        $$F\propto\frac{L}{\chi^2}\cdot\frac{1}{1+z}\cdot\frac{1}{1+z}$$

      \item We may thus find that, in a flat universe:

        $$d_L=\chi(1+z)$$

      \item In a non-flat universe, we see:

        $$d_L=S_{\kappa}(\chi)(1+z)$$

    \end{itemize}

  \item Angular Diameter Distance

    \begin{itemize}

      \item Object of physical size $l$

        \begin{itemize}

          \item Assume a flat universe:

            $$l=d_A\theta$$
            $$d_A^{flat}=\chi a=\frac{\chi}{1+z}=\frac{d_L}{(1+z)^2}$$
            $$d_A=aS_{\kappa}(\chi)$$

          \item Importantly, $d_A$ can increase, reach a max, and then decrease

            \begin{itemize}

              \item This means physical $l$ takes up larger fraction of a smaller universe

            \end{itemize}

          \item This is slightly more complicated in a non-flat universe, since we need to properly account for $d\Omega$ factor

        \end{itemize}

    \end{itemize}

  \item Evidence for Dark Energy

    \begin{itemize}

      \item Until the early 1990's, people did not take $\Lambda$ too seriously. Evidence from galaxy clusters, galaxy clustering, and CMB started to raise some questions. Universe appeared roughly flat, but $\Omega_m<1$

      \item Mapping out $a(t)$ has been a useful way to probe the universe

    \end{itemize}

  \item Dark Matter

    \begin{itemize}

      \item In 1781, William Herschel discovered Uranus; over the next 60 years, astronomers carefully mapped out it orbit, but it didn't quite match Newtonian theory

      \item Urbain Le Verrier: Showed that Uranus' orbit could be explained if there were another, more distant planet acting gravitationally

      \item In September 1846, he mailed a letter to a colleague at the Berlin Observatory with precise predictions

      \item In 1930s, Fritz Zwicky studied the Coma Cluster

        \begin{itemize}

          \item Galaxy Clusters: Largest gravitationally bound objects in the Universe

        \end{itemize}

      \item Zwicky measured spectra of many galaxies and calculated the velocity dispersion

      \item With a quick calculation, the orbital velocity of a galazy cluster may be expressed as:

        $$M=\frac{r_cv^2}{\alpha G}$$

      \item MACHOs (Massive Compact Halo Objects)

        \begin{itemize}
            
          \item Planet or asteroid-sized objects: ``microlensing'' constraints mostly rule this out

        \end{itemize}

      \item Axions

      \item Weakly-Interacting Massive Particles (WIMPs)

      \item Primordial Black Holes (PBHs)

      \item CDM: Cold Dark Matter $\to \Lambda$CDM model

      \item Can generalize $w$CDM such that:

        $$p_{DE}=w\rho_{DE}$$

        \begin{itemize}

          \item $\Lambda$: $w=-1$

            $$w(a)=w_o+w_a(a-a_o)$$

        \end{itemize}

    \end{itemize}

  \item What is Dark Energy?

    \begin{itemize}

      \item ``Cosmological Constant Problem''

      \item Universe expansion is accelerating; $\Lambda$ can do this

      \item Quantum field theory says that vacuum fluctuations have an energy density

    \end{itemize}

  \item Hot Big Bang

    \begin{itemize}

      \item Thermal Equilibrium

        $$\frac{n_1}{n_2}=e^{-(E_1-E_2)/kT}$$

        \begin{itemize}

          \item State 1: $E_1$

          \item State 2: $E_2$

        \end{itemize}

    \end{itemize}

  \item Photons

    \begin{itemize}

      \item Photons are bosons

      \item This means they follow the Bose-Einstein distribution:

        $$\bar{n}_i=\frac{g_i}{e^{(\epsilon_i-\mu)/kT}-1}\Rightarrow \rho_i=\bar{n}_i\epsilon_i$$

        \begin{itemize}

          \item $g_i$ is the ``degeneracy''

          \item This expresses how many particles are in a given energy state $i$

        \end{itemize}

      \item To calculate the energy density, we need to consider the ``phase space'' $\left\{ \vec{x},\vec{p} \right\}$ and integrate over all states with a given $|p|$

      \item Heisenberg's uncertainty principle: $d^3xd^3p$ has $\dfrac{d^3xd^3p}{(2\pi\hbar)^3}$ phase space elements. For photons, $\mu=0$:

        $$\rho=\int\frac{d^3p}{(2\pi\hbar)^3}\frac{2p}{e^{p/kT}-1}$$

        \begin{itemize}

          \item We may obtain:

            $$\rho(f)\,df=\frac{8\pi h}{c^3}\left( \frac{f^3\,df}{e^{hf/kT}-1} \right)$$

            \item For photons between $f$ and $f+df$

        \end{itemize}

      \item Blackbody Spectrum

        \begin{itemize}

          \item Blackbody: Perfectly absorptive system which emits a spectrum given by photon thermal equilibrium

            $$n_c(f_c)=\frac{8\pi}{c^3}\frac{f_c^2}{e^{hf_c/akT}-1}$$

            \begin{itemize}

              \item Preserved if $T\to T/a$

            \end{itemize}

        \end{itemize}

      \item Temperature of the universe is really defined by the distribution of particles in thermal equilibrium

    \end{itemize}

  \item Cosmic Microwave Background

    \begin{itemize}
        
      \item In an expanding universe, we required:

        $$\Gamma>H$$

        \begin{itemize}

          \item $\Gamma\propto n_2v_1\sigma_{12}$

          \item $H$ is the Hubble rate, which is the inverse of Hubble time

          \item ``Freeze-out'' $n_i$ is fixed

        \end{itemize}

    \end{itemize}

  \item Big Bang Nucleosynthesis (BBN): ``The First Three Minutes''

    \begin{itemize}

      \item Fission versus Fusion

      \item Binding energy: Energy to break apart nucleus or, equivalently, energy to release forming the nucleus

    \end{itemize}

  \item Matter-Antimatter Asymmetry

    $$\eta=\frac{n_b}{n_{\gamma}}=6\cdot10^{-10}\Rightarrow\text{6 baryons per 10 billion photons}$$

    \begin{itemize}

      \item Standard model (mostly) treats particles and antiparticles the same

        $$e^{-}+e^{+} \leftrightarrow \gamma+\gamma$$

      \item Not due to freeze out: we don't see equal amounts of anti-matter; annihilation was complete for anti-matter

      \item At early time, $kT>>150[\si{\mega eV}]$ (quarks, baryons): some asymmetry was present

        $$\frac{n_q-n_{q^-}}{n_q}\approx 4\cdot10^{-9}$$

      \item All of the anti-quarks will annihilate, leaving the value of $\eta$ from above

      \item 2 photons per annihilation, 3 quarks per baryon

    \end{itemize}

  \item Baryogenesis (and probably related ``leptogenesis'') is one of the big open questions in cosmology and particle physics

  \item It may be possible within the Standard Model, but it's not clear how. The Sakharov conditions (1967) for baryogenesis state:

    \begin{enumerate}

      \item Baryon number violation

      \item Charge and charge parity-symmetry violation

      \item Out of thermal equilibrium

    \end{enumerate}

  \item WIMP Dark Matter and Freeze-out

    \begin{itemize}

      \item Weakly interacting massive particles (WIMPs, $M\geq 100[\si{\giga eV}]$) has been a very popular DM candidate. New particle could be the result of supersymmetry between fermions and bosons

    \end{itemize}

  \item Recombination and Decoupling

    \begin{itemize}

      \item Discusses when universe became neutral enough to (at least mostly) stop photon scattering

      \item Recombination: When ionization fraction is 1/2

      \item Decoupling: When $H\approx\Gamma_T$, photons no longer interact through Thomson scattering

      \item Last scattering: THe time (or surface) when the typicaly photon last scattered

      \item Not really a surface, more of a narrow shell (think: analogy of looking into fog, you can see a little way in)

    \end{itemize}

  \item We may observe the Saha solution for ionization fraction as:

    $$\frac{\chi_e^2}{1-\chi_e}=\frac{1}{n_e+n_H}\left( \frac{m_eT}{2\pi} \right)^{\frac{3}{2}}e^{-\frac{\epsilon_o}{T}}$$

    \begin{itemize}

      \item When $\chi_e\to 1$, everything is still ionized

      \item For recombination, we want $\chi_e\approx 1/2$

      \item The Saha equation assumes chemical equilibrium; when not in equilibrium, use the Boltzmann equation

    \end{itemize}

  \item CMB Fluctuations

    \begin{itemize}

      \item We may recall that:

        $$\frac{\Delta T}{T}\approx 10^{-5}$$

    \end{itemize}

  \item Fourier versus Real Space (Called Configuration Space in Cosmology)

    \begin{itemize}

      \item Configuration space in position (function of $x$), Fourier space defined with respect to wave number $k$

      \item We may write (in 3D):

        $$F(\vec{x})=\int\frac{d^3\vec{k}}{(2\pi)^{3}}e^{i\vec{k}\vec{x}}\tilde{F}(\vec{k})$$

      \item Plane waves may be expressed by the wave number, $\vec{k}=(k_x,k_y,k_z)$

        \begin{itemize}

          \item A higher $k$ means smaller-scale fluctuations

          \item Plane wave pointing in $\hat{k}$ direction with $\lambda=\frac{2\pi}{k}$

        \end{itemize}

    \end{itemize}

  \item Correlation Functions

    \begin{itemize}

      \item Correlation functions are incredibly important in cosmology because we can predict the statistics of the initial conditions, but the initial conditions themselves are random

      \item A related concept is the power spectrum

    \end{itemize}

  \item Sound Waves

    \begin{itemize}

      \item Start with an initial overdenstiy: $\delta_c=\delta_b=\delta_{\gamma}$

      \item Before $z_{*}$, this creates a travelling compression wave

      \item At $z_{*}$, this wave has travelled a comoving distance:

        $$\text{comoving sound horizon: }\eta^{*}_s=\int_0^{t_{*}}\frac{dt}{a}c_s(t)$$
        $$\text{physical sound horizon: }ds=\eta^{*}_sa$$

      \item For a purely relativistic gas, $c_s=\sqrt{\frac{1}{3}}c$ (and we drop the $c$ in our units)

    \end{itemize}

  \item The Sachs-Wolfe Effect (1967)

    $$\frac{\delta T}{T}=\frac{1}{3}\frac{\delta \Phi}{c^2}$$

    \begin{itemize}

      \item Overdensity means $|\Phi|$, the gravitational redshift, increases

      \item The integrated Sachs-Wolfe Effect (ISW) caused by changing potentials after $z_{*}$ (bot due to radiation $\to$ matter and matter $\to\Lambda$)

    \end{itemize}

  \item Flatness Problem

    \begin{itemize}

      \item The universe looks close to flat today. This means it must have been \underline{very} flat early on

    \end{itemize}

  \item Monopole Problem

    \begin{itemize}

      \item GUT phase transition leaves ``defects'' $\to$ roughyl oneper $\chi_{GUT}$

    \end{itemize}

  \item Initial conditions: 

    \begin{itemize}

      \item Why does the universe start expanding? Where do the fluctuations come from?

      \item $t\to 0,\, a\to0,\, T\to\infty$: avoid singularity! (Quantum gravity is also relevant here)

    \end{itemize}

  \item Inflation

    \begin{itemize}

      \item Horizon, Conformal Time, Hubble Distance

        \begin{itemize}

          \item $\chi$ is the comoving distance

          \item We may write:

            $$\chi(a_1,a_2)=\int_{a_1}^{a_2}\frac{da}{a^2H(a)}$$

          \item Typically, we measure from us today $(a=1)$

          \item The comoving horizon today is the distance to $a=0$ (or $a\to0$)

            $$\chi_h=\chi(a=0)=\int_0^{1} \frac{da}{a^2H(a)}$$

          \item Often, the horizon gets a new variable:

            $$\eta(a)=\chi(a=0,a)$$

          \item $\eta$ is also sometimes called the conformal time

          \item It is monotonically increasing, perfectly good time variable

          \item We can rewrite the FRW metric:

            $$ds^2=a^2(\eta)\left[-d\eta^2+\frac{dr^2}{1-\kappa r^2}+r^2\,d\Omega^2\right]$$

          \item Overall scaling of the metric: ``conformal transformation''

          \item The horizon expresses global causality; how far could light have travelled since $t=0$?

          \item Our cosmic event horizon is the distance from which light could \underline{ever} readh us out to $t\to\infty$

          \item For a $\Lambda$CDM universe, this is \underline{not} infinite

        \end{itemize}

      \item Developed by Guth, Starobinsky, and Linde (1979-1981), originally as a way to solve the monopole problem

      \item $\rho_{inf}$ comes from some new field in the universe

      \item This is a scalar field (think electric field, $\vec{E}$ carries energy density $E^2$)

      \item Changing $\vec{E}\to\vec{B}$ makes total $\rho\propto E^2+B^2$ (the ``potential'' and ``kinetic'' energy in the field)

    \end{itemize}

  \item Galaxy Formation

    \begin{itemize}

      \item Dark matter and baryons collapse to form a halo

      \item Baryons cool through radiation and fall to center of the halo, forming a galazy

      \item Formatiom looks simple from a statistical perspective (linear galaxy bias):

        $$\delta_g=b\delta_m$$

        \begin{itemize}

          \item The overdensity in galaxies is proportional to the overdensity of dark matter

          \item Overdensity is expressed as:

            $$\delta=\frac{\rho-\bar{\rho}}{\rho}$$

        \end{itemize}

    \end{itemize}

  \item Redshift Space Distortions

    \begin{itemize}

      \item Strength of effect given by:

        $$f\equiv \frac{d\ln(D)}{d\ln(a)}$$

        \begin{itemize}

          \item More growth $\to$ larger peculiar velocities

        \end{itemize}

    \end{itemize}

  \item Gravitational Lensing

    \begin{itemize}

      \item All geodesics are impacted by mass/energy density, including photons

      \item The path of light is bent $\to$ gravitational lensing

      \item For a point source: deflection

      \item For an extended source (\textit{e}.\textit{g} a galaxy), you also get distortions:

        \begin{itemize}

          \item Shear

          \item Magnification

        \end{itemize}

      \item Recall geodesic deviation

        $$A^{\mu}=\frac{D^2}{dt^2}\,S^{\mu}=R^{\mu}_{\nu\rho\sigma}T^{\nu}T^{\rho}S^{\sigma}$$

    \end{itemize}

  \item Regimes of Lensing (NB: All within ``weak field'' limit; differences depend on geometry of observation)

    \begin{itemize}

      \item Strong Lensing: Multiple images, arcs, Einstein rings, strong magnification

        $$\theta\leq \theta_E$$

      \item Weak Lensing: Small distrotions, must be statistically studied

        $$\theta > \theta_E$$

      \item Microlensing: Unresolved strong lensing (\textit{e}.\textit{g}. by an exoplanet)

    \end{itemize}

  \item Lensing Derivations

    \begin{itemize}

      \item Assuming that the curvature is small (weak field), we can write the metric perturbatively:

        $$g_{\mu\nu}=g^{(o)}_{\mu\nu}+h_{\mu\nu}$$

      \item Initially, we can assume that $g_{\mu\nu}^{(o)}=\eta_{\mu\nu}$; however, for cosmology, we need to remember that $g^{(o)}_{\mu\nu}$ is actually the homogenous FLRW metric

      \item Decomposing the perturbation, we get:

        $$\left\{\begin{array}{ll} h_{oo}=-2\Phi, & \Phi,\Psi:\text{ scalar perturbations}\\h_{oi}=w_i, & w_i:\text{ vector perturbations}\\h_{ij}=2s_{ij}-2\Psi\delta_{ij}, & s_{ij}:\text{ tensor perturbations}\end{array}$$

        \item Cranking through some algebra and plugging into Einstein's equation, we get:

          $$R_{\mu\nu}-\frac{1}{2}Rg_{\mu\nu}=8\pi GT_{\mu\nu}$$

          \begin{itemize}

            \item From this, we find:

              $$\nabla^2\Psi=4\pi G\rho$$
              $$w_i=s_{ij}=0$$
              $$\Psi=\Phi$$

            \item The last expression is an important General Relativity result, valid only when Tr($T_{ij}$)=0

          \end{itemize}

        \item We may recall that, in the ``Newtonian limit'' only the $h_{oo}$ was important for geodesics; now we deal with photons, so we need both time and space perturbations. Considering a null geodesic, we write:

          $$x^{\mu}(\lambda)=x^{(o)\mu}(\lambda)+x^{(1)\mu}(\lambda)$$

        \item We define:

          $$k^{\mu}=\frac{dx^{(o)\mu}}{d\lambda}\quad\text{ and }\quad l^{\mu}=\frac{dx^{(1)\mu}}{d\lambda}$$

        \item The null geodesic expression becomes:

          $$(\eta_{\mu\nu}+h_{\mu\nu})(k^{\mu}+l^{\mu})(k^{\nu}+l^{\nu})=0$$

        \item This can be evaluated to:

          $$\frac{dl^{\mu}}{d\lambda}=-\Gamma^{\mu}_{\rho\sigma}k^{\rho}k^{\sigma}$$
          $$\text{Time: }\quad\frac{dl^{o}}{d\lambda}=-2k(\vec{k}\cdot\vec{\nabla}\Phi)$$
          $$\text{Space: }\quad\frac{d\vec{l}}{d\lambda}=-2k^2\vec{\nabla}_{\perp}\Phi$$

        \item We want to find the deflection angle, $\hat{\alpha}$, which can be expressde as:

          $$\frac{\Delta\vec{l}}{k}=\Delta \theta\equiv \hat{\alpha}$$

        \item Thus, we can get:

          $$\Delta\vec{l}=\int\frac{d\vec{l}}{d\lambda}\,d\lambda=-2k^2\int\vec{\nabla}_{\perp}\Phi\,d\lambda$$

        \item We can thus write $\hat{\alpha}$ as:

          $$\hat{\alpha}=2\int\vec{\nabla}_{\perp}\Phi\,ds$$

        \item We obtain the lens equation as:

          $$\vec{\beta}=\vec{\theta}-\vec{\alpha}=\vec{\theta}-\frac{D_{LS}}{D_{S}}\hat{\alpha}$$

        \item For a point mass, we substitute to get:

          $$\beta=\theta-\frac{D_{LS}}{D_LD_S}\frac{4GM}{\theta}$$

          \begin{itemize}

            \item Considering perfect lens-source alignment ($\beta=0$), we get the Einstein angle:

              $$\theta_E=\sqrt{\frac{4GMD_{LS}}{D_LD_S}}$$

            \item From here, we get the Einstein radius:

              $$R_E=D_L\theta_E=\sqrt{\frac{4GMD_LD_{LS}}{D_S}}$$

            \item For imperfect lens-source alignment, we get:

              $$\theta^2-\beta\theta-\theta_E^=0$$
              $$\theta=\frac{1}{2}\left[ \beta\pm\sqrt{\beta^2-4\theta_E^2} \right]$$

            \item When $\beta>>\theta_E$, we have:

              $$\theta=\frac{1}{2}\left[ \beta\pm\beta\left( 1+\frac{2\theta_E}{\beta} \right) \right]$$

            \item This gets us:

              $$\theta_+=\beta+\frac{\theta_E^2}{\beta}$$
              $$\theta_-&=-\frac{\theta_E^2}{\beta}\to0$$

          \end{itemize}

    \end{itemize}

  \item Lensing Potential

    $$\psi(\vec{\theta})=\frac{2D_{Ls}}{D_LD_S}\int \Phi(D_L\vec{\theta},s)\,ds$$

    \begin{itemize}

      \item We may write:

        $$\vec{\alpha}=\vec{\nabla}_\theta\psi=D_L\vec{\nabla}_{\perp}\psi$$

      \item We define the convergence as:

        $$\kappa(\vec{\theta})=\frac{1}{2}\nabla^2_{\theta}\psi=\frac{D_LD_{LS}}{D_s}\int\nabla^2\Phi\,ds$$

      \item By Poisson's equation, this is simply the integral of the overdensity, so $\kappa$ is the lens-geometry-weighted mass density at some position $\theta$ on the sky

        $$\kappa(\vec{\theta})=\frac{D_LD_{LS}}{D_s}\int\delta\,ds$$

      \item In terms of $\kappa$, the deflection angle can be written:

        $$\vec{\alpha}(\vec{\theta})=\frac{1}{\pi}\int\kappa(\vec{\theta}')\left[ \frac{\vec{\theta}-\vec{\theta}'}{|\vec{\theta}-\vec{\theta}'|^2} \right]\,d^2\theta'$$

      \item The 2D lensing transformation can be described as a matrix:

        $$A_{ij}\equiv\frac{\partial\beta_i}{\partial\theta_j}=\delta_{ij}-\frac{\partial \alpha_i}{\partial \theta_j}=\delta_{ij}-\frac{\partial^2\psi}{\partial\theta_i\partial\theta_j}\equiv \delta_{ij}-\psi_{ij}$$

      \item We may describe:

        \begin{itemize}

          \item Convergence: $\kappa=\dfrac{1}{2}(\psi_{11}+\psi_{22})$ (isotropic focusing)

          \item Shear: $\gamma_1=\dfrac{1}{2}(\psi_{11}-\psi_{22})=\gamma\cos(2\phi)$ and $\gamma_2=\psi_{12}=\psi_{21}$

        \end{itemize}

        $$A=\left[ \begin{matrix} 1-\kappa-\gamma_1 & -\gamma_2\\ -\gamma_2 & 1-\kappa+\gamma_1\end{matrix} \right]$$

      \item $\mu=A^{-1}$ is the magnification tensor

      \item Surface brightness is conserved, so total observed flux is:

        $$F_{obs}=\mu F_{true}$$

      \item For $\mu>1$, a given patch of sky will appear larger, and for $\mu<1$ it will appear smaller

    \end{itemize}

  \item Technical Terms

    \begin{itemize}

      \item ``Cosmic Shear'' — Correlations in shapes due to weak lensing

      \item Galaxy-Galaxy Lensing — The lensing of galaxies by galaxies (essentially determining the mass of galaxy based on its lensing contribution to a different galaxy)

      \item CMB Lensing — CMB photons are also affected by shear and lensing, so the CMB map could be used

      \item Strong Lensing — Mass mapping, time delays, dark matter constraints

    \end{itemize}

  \item Gravitational Waves

    \begin{itemize}

      \item Ripples in spacetime

      \item ``Gauge Transformations''

      \item Using the same ``chirp'' from above, we may write:

        $$h_{\mu\nu}^{TT}=\left( \begin{matrix} 0 & 0 & 0 & 0\\ 0 & & &\\0 & & s_{ij} &\\ 0 & & & \end{matrix} \right)$$

        \begin{itemize}

          \item This is the transverse-traceless gauge (TT)

          \item Often referred to by a box: $\square$

        \end{itemize}

      \item This results in a wave equation

        \begin{itemize}

          \item Note that this obeys the same generic solution set:

            $$f(x,t)=Ae^{i(kx-\omega t)}$$

          \item This plane wave follows:

            $$k=\frac{\omega}{c_s}=\frac{2\pi}{\lambda c_s}$$

          \item Can add multiple plane waves to match boundary/initial conditions

          \item We can note:

            $$k\Delta x=\omega\Delta t$$

        \end{itemize}

      \item Gravitational waves have quadrupolar symmetry ($180^{\circ}$ rotation invariant), unlike electromagnetic waves, which are dipoles ($360^{\circ}$ rotation invariant)

      \item For gravitational radiation, this dipole scenario doesn't work, as there is only one gravitational ``charge''  (universality of gravity)

      \item We don't have a quantized theory of gravity (proven) yet, but we know it has a spin-2 particle

      \item We could have derived Einstein's equations by taking the particle physics perspective and writing down $\mathcal{L}$ including a new spin-2 particle $h_{\mu\nu}$

      \item String theory naturally gives rise to 2 spin-2 degrees of freedom and spin-0 degrees of freedom

    \end{itemize}

  \item Production and detection of gravitational waves

    \begin{itemize}

      \item This is in general a complex problem: perturbation theory breaks down, and numerical techniques are needed

      \item Once you have the form of the wave, the propagation is easily expressed as a sum of plane waves

      \item Gravitational waves are produced by changing quadrupole moment of mass distribution:

        $$\bar{h}_{ij}(t,\vec{x})=\frac{2G}{r}\frac{d^2I_{ij}}{dt^2}$$

        \begin{itemize}

          \item Where:

            $$I_{ij}(t)=\int y^iy^jT^{oo}(t,y)\,d^3y$$

          \item Is the quadrupole of mass/energy distribution

        \end{itemize}

      \item Using Newtonian orbits (which, technically, are not correct, especially as objects are close to merging, i.e. strong gravity), but good enough for order of magnitude:

        $$\Omega=\frac{2\pi}{T}=\sqrt{\frac{GM}{4R^3}}$$

        $$\bar{h}_{ij}(t,\vec{x})=\frac{8GM}{r}\Omega^2R^2\left( \begin{matrix} -\cos(2\Omega t) & -\sin(2\Omega t) & 0\\ -\sin(2\Omega t) & \cos(2\Omega t) & 0\\ 0 & 0 & 0 \end{matrix} \right)$$

    \end{itemize}

  \item Binary Pulsar: Hulse and Taylor (1974 discovery, 1993 Nobel Prize)

    \begin{itemize}

      \item Pulsar: a spinning neutron star

      \item Two neutron stars, one of which is a pulsar, which pulses 17 times per second

      \item Precise timing allows us to track the orbit with very good precision
        
    \end{itemize}

  \item Gravitational Wave Sources (in order of decreasing frequency)

    \begin{itemize}

      \item Mergers of approximately stellar mass black holes, neutron stars

      \item Supernovae

      \item Supermassive black hole mergers

      \item Inflation, early universe phase transitions

    \end{itemize}

  \item Gravitational Wave Detection

    \begin{itemize}

      \item LIGO: Laser Interferometer Gravitational-Wave Observatory

      \item Pulsar Timing Arrays

      \item CMB Polarization

    \end{itemize}

\end{itemize}

\end{document}

