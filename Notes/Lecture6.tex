%%%%%%%%%%%%%%%%%%%%%%%%%%%%%%%%%%%%%%%%%%%%%%%%%%%%%%%%%%%%%%%%%%%%%%%%%%%%%%%%%%%%%%%%%%%%%%%%%%%%%%%%%%%%%%%%%%%%%%%%%%%%%%%%%%%%%%%%%%%%%%%%%%%%%%%%%%%%%%%%%%%
% Written By Michael Brodskiy
% Class: General Relativity and Cosmology
% Professor: J. Blazek
%%%%%%%%%%%%%%%%%%%%%%%%%%%%%%%%%%%%%%%%%%%%%%%%%%%%%%%%%%%%%%%%%%%%%%%%%%%%%%%%%%%%%%%%%%%%%%%%%%%%%%%%%%%%%%%%%%%%%%%%%%%%%%%%%%%%%%%%%%%%%%%%%%%%%%%%%%%%%%%%%%%

\documentclass[12pt]{article} 
\usepackage{alphalph}
\usepackage[utf8]{inputenc}
\usepackage[russian,english]{babel}
\usepackage{titling}
\usepackage{amsmath}
\usepackage{graphicx}
\usepackage{enumitem}
\usepackage{amssymb}
\usepackage[super]{nth}
\usepackage{everysel}
\usepackage{ragged2e}
\usepackage{geometry}
\usepackage{multicol}
\usepackage{fancyhdr}
\usepackage{cancel}
\usepackage{siunitx}
\usepackage{physics}
\usepackage{tikz}
\usepackage{mathdots}
\usepackage{yhmath}
\usepackage{cancel}
\usepackage{color}
\usepackage{array}
\usepackage{multirow}
\usepackage{gensymb}
\usepackage{tabularx}
\usepackage{extarrows}
\usepackage{booktabs}
\usepackage{lastpage}
\usetikzlibrary{fadings}
\usetikzlibrary{patterns}
\usetikzlibrary{shadows.blur}
\usetikzlibrary{shapes}

\geometry{top=1.0in,bottom=1.0in,left=1.0in,right=1.0in}
\newcommand{\subtitle}[1]{%
  \posttitle{%
    \par\end{center}
    \begin{center}\large#1\end{center}
    \vskip0.5em}%

}
\usepackage{hyperref}
\hypersetup{
colorlinks=true,
linkcolor=blue,
filecolor=magenta,      
urlcolor=blue,
citecolor=blue,
}


\title{Lecture 6 — The Expanding Universe}
\date{\today}
\author{Michael Brodskiy\\ \small Professor: J. Blazek}

\begin{document}

\maketitle

\begin{itemize}

  \item Olber's Paradox: Why is the Sky Dark?

    \begin{itemize}

      \item Absorbing Matter $\to$ doesn't work since matter would heat up

      \item Finite Size

      \item Finite Time (and Finite Speed of Light)

      \item Dimming of Light (``redshift'')

    \end{itemize}

  \item Universe had some beginning (``Big Bang'') around 13.7 billion years ago

  \item Some units:

    $$1[\text{light year}]=9.5\cdot10^{15}[\si{\meter}]$$
    $$1[\text{yr}]\approx \pi\cdot10^{7}[\si{\second}]$$
    $$c\approx 3\cdot10^8\left[ \si{\meter\over\second} \right]$$
    $$1[\si{\parsec}]=3.26[\text{light years}]$$

    \begin{itemize}

      \item If $\theta=1[\text{arcsec}]$, then $d=1[\si{\parsec}]$

        $$1[\si{\parsec}]=2.1\cdot10^5[\text{AU}]$$

    \end{itemize}

  \item The Cosmological Principle

    \begin{itemize}

      \item Copernicus: the Sun, not the Earth, is the center of the Universe

      \item Cosmological Principle: There is no center to the Universe

        \begin{itemize}

          \item The Universe is statistically isotropic (same in all directions) and homogenous (same everywhere)

        \end{itemize}

    \end{itemize}

  \item Expanding Universe

    \begin{itemize}
        
      \item All observers see things moving away from them

      \item Statements are all statistical! Distinguish between structure in the universe and the geometry of the homogenous universe (about 100$[\si{\mega\parsec}]$ scales for homogeneity)

      \item We don't experience the FLRW metric

        \begin{itemize}

          \item Homogenous/geometry

          \item Structure

        \end{itemize}

      \item Conservation of Energy Solution to Expanding Cloud:

        $$\frac{1}{2}\dot{R}^2-\frac{2GM}{R}=C$$

        \begin{itemize}

          \item What is the physical meaning of $C$?

            \begin{itemize}

              \item $C=0$

                You are just at escape velocity. As $R\to\infty$, $\dot{R}=v\to0$. Potential and kinetic both go to zero.
                
              \item $C>0$

                Positive total energy. You have more than enough energy to escape. $\dot{R}>0$ as $R\to\infty$
                
              \item $C<0$

                Negative total energy. You won't make it out to $R=\infty$, you will stop and turn around at some finite $t$
                
            \end{itemize}

          \item These cases capture an ideal universe's expansion with only real matter

          \item For a matter-only universe, $C$ describes the spatial geometry, with zero indicating flat, $C>0$ indicating open (negative curvature), and $C<0$ indicating closed (positive curvature)

            \begin{itemize}

              \item Note that $\Lambda$ complicates this

            \end{itemize}

        \end{itemize}

      \item Using the first-order equation, we may write:

        $$\frac{1}{2}\dot{R}^2=\frac{4}{3}\pi G\rho R^2+C$$
        $$\left(\frac{\dot{R}}{R}\right)=\frac{8\pi G\rho}{3}+C$$

      \item At $C=0$, we find $\rho_{crit}$:

        $$\rho_{crit}=\frac{3H^2}{8\pi G}$$

        Note that we define the Hubble parameter as:

        $$H=\frac{\dot{R}}{R}$$

    \end{itemize}

  \item Comoving Coordinates

    \begin{itemize}

      \item Coordinates (and distances) scale with the size of the universe:

        $$x=a(t)r$$

        \begin{itemize}

          \item $x$ is the ``proper'' coordinate

          \item $r$ is the ``comoving'' coordinate

          \item $a(t)$ is the scale factor, with $a(t_o)=1$ indicating ``today''

        \end{itemize}

      \item Ex. $x_{12}=a(t)r_{12}$

        $$\frac{dx_{12}}{dt}\quad\text{ is the recession velocity}$$

      \item Two Things:

        \begin{enumerate}

          \item Velocity is proportional to distance (Hubble law)

          \item The proportionality term is $\dfrac{\dot{a}}{a}=H(t)$

        \end{enumerate}

      \item Hubble constant is approximately:

        $$H_o=70\left[ \si{\kilo\meter\over\second\mega\parsec} \right]$$

        \begin{itemize}

          \item We define $h$, such that:

            $$H_o=100h\left[ \si{\kilo\meter\over\second\mega\parsec} \right]$$

          \item We can see that the units are actually just $\si{\hertz}$, which gives us:

          $$H_o\approx 3.24\cdot10^{-18}\left[\si{1\over\second}\right]$$

        \end{itemize}

      \item If $a$ is constant, this gives us the doubling time or the time to go back to $a=0$, which gives us the ``Hubble'' time, or the age of the universe

      \item For $h=.7\to 1.4\cdot10^{10}\left[ \frac{1}{\text{yr}} \right]$

        \begin{itemize}

          \item Fairly accurate approximation of 14 billion years

        \end{itemize}

    \end{itemize}

  \item FLRW Metric

    \begin{itemize}

      \item Cosmological principle tells us that spacetime (on large scales) should be homogenous (translation invariant) and isotropic (rotation invariant)

      \item We seek a general form of a metric that obeys these assumptions:

        $$ds^2=\bar{c}^2\,dt^2+R^2(t)\,d\sigma^2$$

      \item We are able to derive:

        $$d\sigma^2=\frac{d\bar{r}^2}{1-k\bar{r}^2}+\bar{r}^2\,d\Omega^2$$

      \item Where:

        \begin{itemize}

          \item $d\Omega^2$ is the 2-sphere metric

          \item $k\propto R$ (Ricci scalar on 3D space)

            $$k=\left\{\begin{array}{ll}+1,& \text{Positive Curvature (3-sphere), "closed"}\\ 0,& \text{Flat}\\ -1,&\text{Negative curvature (saddle), "open"}\end{array}$$

        \end{itemize}

      \item Thus, we may define the metric as:

        $$ds^2=-dt^2+a^2(t)\left[ \frac{dr^2}{1-\kappa r^2}+r^2\,d\Omega^2 \right]$$

        \begin{itemize}

          \item Where:

            $$\left\{\begin{array}{ll}\kappa>0,&\text{closed}\\\kappa=0,&\text{flat}\\\kappa<0,&\text{open}\end{array}$$

        \end{itemize}

    \end{itemize}

  \item Cosmological Redshift

    \begin{itemize}

      \item Like a D\"oppler shift, but use caution!

      \item $v=H_od$, locally $v<<c$

      \item Redshift in special relativity:

        $$\frac{\lambda_{obs}}{\lambda_{em}}=\sqrt{\frac{1+v/c}{1-v/c}}\approx1+\frac{v}{c}$$

      \item From this, we know:

        $$\frac{\lambda_{obs}}{\lambda_{em}}=1+z\approx 1+\frac{H_o d}{c}$$

      \item More generally, we may say (for $a_o=$ today):

        $$1+z=\frac{a_{obs}}{a_{em}}=a_{em}^{-1}$$

    \end{itemize}

  \item Cosmological Redshift versus Peculiar Motions:

    \begin{itemize}

      \item For comoving coordinates:

        $$v\approx H_od=H_o|\vec{r}_2-\vec{r}_1|/a_o=H_o|\vec{r}_2-\vec{r}_1|$$
        
      \item For ``peculiar'' motions

        $$v\approx H_od+\frac{\Delta v}{c}$$

      \item In the FLRW metric, due to lack of simple time symmetry, energy is not conserved

        \begin{itemize}

          \item There is a Killing Tensor that reflects a symmetry:

            $$K_{\mu\nu}=a^2(g_{\mu\nu}+U_{\mu}U_{\nu})\quad\text{ for observer (comoving) $U^{\mu}=(1,0,0,0)$}$$

          \item We get:

            $$K^2=K_{\mu\nu}V^{\mu}V^{\nu}$$

            \begin{itemize}

              \item Is conserved, with

                $$V^{\mu}=\frac{dx^{\mu}}{d\lambda}$$

              \item For a photon on a null geodesic:

                $$V_{\mu}V^{\mu}=0$$

            \end{itemize}

          \item This can be simplified to:

            $$v=\frac{K}{a}$$

        \end{itemize}

    \end{itemize}

\end{itemize}

\end{document}

