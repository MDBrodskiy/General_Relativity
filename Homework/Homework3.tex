%%%%%%%%%%%%%%%%%%%%%%%%%%%%%%%%%%%%%%%%%%%%%%%%%%%%%%%%%%%%%%%%%%%%%%%%%%%%%%%%%%%%%%%%%%%%%%%%%%%%%%%%%%%%%%%%%%%%%%%%%%%%%%%%%%%%%%%%%%%%%%%%%%%%%%%%%%%%%%%%%%%
% Written By Michael Brodskiy
% Class: General Relativity and Cosmology
% Professor: J. Blazek
%%%%%%%%%%%%%%%%%%%%%%%%%%%%%%%%%%%%%%%%%%%%%%%%%%%%%%%%%%%%%%%%%%%%%%%%%%%%%%%%%%%%%%%%%%%%%%%%%%%%%%%%%%%%%%%%%%%%%%%%%%%%%%%%%%%%%%%%%%%%%%%%%%%%%%%%%%%%%%%%%%%

\documentclass[12pt]{article} 
\usepackage{alphalph}
\usepackage[utf8]{inputenc}
\usepackage[russian,english]{babel}
\usepackage{titling}
\usepackage{amsmath}
\usepackage{graphicx}
\usepackage{enumitem}
\usepackage{amssymb}
\usepackage[super]{nth}
\usepackage{everysel}
\usepackage{ragged2e}
\usepackage{geometry}
\usepackage{multicol}
\usepackage{fancyhdr}
\usepackage{cancel}
\usepackage{siunitx}
\usepackage{physics}
\usepackage{tikz}
\usepackage{mathdots}
\usepackage{yhmath}
\usepackage{cancel}
\usepackage{color}
\usepackage{array}
\usepackage{multirow}
\usepackage{gensymb}
\usepackage{tabularx}
\usepackage{extarrows}
\usepackage{booktabs}
\usepackage{lastpage}
\usetikzlibrary{fadings}
\usetikzlibrary{patterns}
\usetikzlibrary{shadows.blur}
\usetikzlibrary{shapes}

\geometry{top=1.0in,bottom=1.0in,left=1.0in,right=1.0in}
\newcommand{\subtitle}[1]{%
  \posttitle{%
    \par\end{center}
    \begin{center}\large#1\end{center}
    \vskip0.5em}%

}
\usepackage{hyperref}
\hypersetup{
colorlinks=true,
linkcolor=blue,
filecolor=magenta,      
urlcolor=blue,
citecolor=blue,
}


\title{Homework 3}
\date{\today}
\author{Michael Brodskiy\\ \small Professor: J. Blazek}

\begin{document}

\maketitle

\begin{enumerate}

  \item

    \begin{enumerate}

      \item 

        Per the given metric, we may write:

        $$I=\frac{1}{2}\int ds(\dot{\psi}^2+\sin^2(\psi)\dot{\theta}^2+\sin^2(\psi)\sin^2(\theta)\dot{\phi}^2)$$

        This gives us:

        $$\delta I=\frac{1}{2}\int ds(2\dot{\psi}\,\delta\dot{\psi}+\sin(2\psi)\dot{\theta}^2\,\delta\psi+\sin(2\psi)\sin^2(\theta)\dot{\phi}^2\,\delta\psi+2\sin^2(\psi)\dot{\theta}\,\delta\dot{\theta}$$
        $$+\sin(2\theta)\sin^2(\psi)\dot{\phi}^2\,\delta\theta+2\sin^2(\psi)\sin^2(\theta)\dot{\phi}\,\delta\dot{\phi})$$

        We can apply the property that $\delta\dot{x}=\frac{d}{ds}\delta x$, and integrate by parts to get:

        $$\delta I=\frac{1}{2}\int ds[(-2\ddot{\psi}+\sin(2\psi)\dot{\theta}^2+\sin(2\psi)\sin^2(\theta)\dot{\phi}^2)\,\delta\psi+(-2\sin^2(\psi)\ddot{\theta}-2\sin(2\psi)\dot{\theta}\dot{\psi}$$
        $$+\sin(2\theta)\sin^2(\psi)\dot{\phi}^2)\,\delta\theta-(2\sin(2\psi)\sin^2(\theta)\dot{\psi}\dot{\phi}+2\sin^2(\psi)\sin(2\theta)\dot{\theta}\dot{\phi}+2\sin^2(\psi)\sin^2(\theta)\ddot{\phi})\,\delta\phi]$$

        From this long expression, we obtain three expressions, which we know must evaluate to zero, independently, due to the fact that the geodesic must be zero independent of variations. Thus, we see:

        $$-\ddot{\psi}+\frac{1}{2}\sin(2\psi)\dot{\theta}^2+\frac{1}{2}\sin(2\psi)\sin^2(\theta)\dot{\phi}^2=0$$
        $$-\sin^2(\psi)\ddot{\theta}-\sin(2\psi)\dot{\theta}\dot{\psi}+\frac{1}{2}\sin(2\theta)\sin^2(\psi)\dot{\phi}^2=0$$
        $$-\sin(2\psi)\sin^2(\theta)\dot{\psi}\dot{\phi}-\sin^2(\psi)\sin(2\theta)\dot{\theta}\dot{\phi}-\sin^2(\psi)\sin^2(\theta)\ddot{\phi}=0$$

        We want to put these in terms that mach the geodesic equation form, or with an isolated second order derivative. This gives us:

        $$\ddot{\psi}-\frac{1}{2}\sin(2\psi)\dot{\theta}^2-\frac{1}{2}\sin(2\psi)\sin^2(\theta)\dot{\phi}^2=0$$
        $$\ddot{\theta}+2\cot(\psi)\dot{\theta}\dot{\psi}-\frac{1}{2}\sin(2\theta)\dot{\phi}^2=0$$
        $$\ddot{\phi}+2\cot(\psi)\dot{\psi}\dot{\phi}+2\cot(\theta)\dot{\theta}\dot{\phi}=0$$

        With this form, we can now compare with the formula:

        $$\ddot{x}^a+\Gamma^a_{bc}\dot{x}^b\dot{x}^c=0$$

        First, taking $a=\psi$, we may see that the symbol is non-zero only for $b=c=\theta$ or $b=c=\phi$. Thus, we see the only non-zero terms with $a=\psi$ are:

        $$\Gamma^{\psi}_{\theta\theta}=-\frac{1}{2}\sin(2\psi)$$
        $$\Gamma^{\psi}_{\phi\phi}=-\frac{1}{2}\sin(2\psi)\sin^2(\theta)$$

        Next, we take $a=\theta$. We may see that the symbol is non-zero for $b=\theta$ and $c=\psi$, or vice versa, and $b=c=\phi$. Thus, we get the non-zero terms as:

        $$\Gamma^{\theta}_{\psi\theta}=\Gamma^{\theta}_{\theta\psi}=2\cot(\psi)$$
        $$\Gamma^{\theta}_{\phi\phi}=-\frac{1}{2}\sin(2\theta)$$

        Finally, we take $a=\phi$, and see that one of the $b$ or $c$ must be $\phi$, with the other being $\psi$ or $\theta$. This gives us the non-zero terms as:

        $$\Gamma^{\phi}_{\psi\phi}=\Gamma^{\phi}_{\phi\psi}=2\cot(\psi)$$
        $$\Gamma^{\phi}_{\theta\phi}=\Gamma^{\phi}_{\phi\theta}=2\cot(\theta)$$

        Thus, we see the non-zero terms are:

        $$\boxed{\left\{\begin{array}{lll} \Gamma^{\psi}_{\theta\theta}&=&-\frac{1}{2}\sin(2\psi)\\\Gamma^{\psi}_{\phi\phi}&=&-\frac{1}{2}\sin(2\psi)\sin^2(\theta)\\\Gamma^{\theta}_{\psi\theta}=\Gamma^{\theta}_{\theta\psi}&=&2\cot(\psi)\\\Gamma^{\theta}_{\phi\phi}&=&-\frac{1}{2}\sin(2\theta)\\\Gamma^{\phi}_{\psi\phi}&=&\Gamma^{\phi}_{\phi\psi}=2\cot(\psi)\\\Gamma^{\phi}_{\theta\phi}=\Gamma^{\phi}_{\phi\theta}&=&2\cot(\theta) \end{array}}$$

      \item 

        We know the Riemann Tensor is defined as:

        $$R^{\rho}_{\sigma\mu\nu}=\partial_{\mu}\Gamma^{\rho}_{\mu\sigma}-\partial_{\nu}\Gamma^{\rho}_{\mu\sigma}+\Gamma^{\rho}_{\mu\lambda}\Gamma^{\lambda}_{\nu\sigma}-\Gamma^{\rho}_{\mu\lambda}\Gamma^{\lambda}_{\mu\sigma}$$

        Let us begin with $\rho=\psi$:

        $$R^{\psi}_{\sigma\mu\nu}=\partial_{\mu}\Gamma^{\psi}_{\mu\sigma}-\partial_{\nu}\Gamma^{\psi}_{\mu\sigma}+\Gamma^{\psi}_{\mu\lambda}\Gamma^{\lambda}_{\nu\sigma}-\Gamma^{\psi}_{\mu\lambda}\Gamma^{\lambda}_{\mu\sigma}$$

        By observation, we may see that this is non-zero only for $\mu=\sigma=\theta\text{ or }\phi$ or $\mu=\lambda=\theta\text{ or }\phi$. Thus, let us begin with the former:

      \item 

    \end{enumerate}

  \item

    \begin{enumerate}

      \item 

        The Schwarzschild metric is given by:

        $$ds^2=-\left( 1-\frac{2GM}{r} \right)\,dt^2+\left( 1-\frac{2GM}{r} \right)^{-1}\,dr^2+r^2\,d\Omega^2$$

        We know that $\Omega$ refers to the angular trajectory of the particle; we may assume that the trajectory is spherically similar to get $d\Omega=0$, which gives us:

        $$ds^2=-\left( 1-\frac{2GM}{r} \right)\,dt^2+\left( 1-\frac{2GM}{r} \right)^{-1}\,dr^2$$

        Since the proper time is taken with fixed spatial coordinates, we may combine the above, along with:

        $$d\tau^2=\left( 1-\frac{2GM}{r} \right)\,dt^2$$

        to get:

        $$ds^2=-d\tau^2+\left( 1-\frac{2GM}{r} \right)^{-1}\,dr^2$$

        Given that the particle must be time or light-like, we know that $ds^2\leq 0$, which allows us to write:

        $$-d\tau^2+\left( 1-\frac{2GM}{r} \right)^{-1}\,dr^2\leq 0$$

        We can now rearrange to get:

        $$\left( 1-\frac{2GM}{r} \right)^{-1}\,dr^2\leq d\tau^2$$
        $$\left( 1-\frac{2GM}{r} \right)^{-1}\leq \left(\frac{d\tau}{dr}\right)^2$$
        $$\left(\frac{dr}{d\tau}\right)^2\leq 1-\frac{2GM}{r}$$

        And finally, we get:

        $$\boxed{\Big|\frac{dr}{d\tau}\Big|\geq \sqrt{\frac{2GM}{r}-1}}$$

      \item 

        We may integrate along the particle's trajectory to find the proper time. Let us use the result from part (a) at its maximum to find the maximum proper time. Thus, we use:

        $$\Big|\frac{dr}{d\tau}\Big|=\sqrt{\frac{2GM}{r}-1}$$

        To get:

        $$\int d\tau_{max}=\int_0^{2GM} \frac{dr}{\sqrt{\frac{2GM}{r}-1}}$$
        $$\tau_{max}=\left[ -2GM\tan^{-1}\left( \sqrt{\frac{2GM}{r}-1} \right)-GM\sin\left( 2\tan^{-1}\left( \sqrt{\frac{2GM}{r}}-1 \right) \right) \right]_0^{2GM}$$

        We may see that the terms go to zero at $r=2GM$. At $r=0$, we get:

        $$\boxed{\tau_{max}=\pi GM}$$

        In solar masses, we return the $c$ component to write:

        $$\boxed{\tau_{max}=\frac{\pi GM}{c^3}\approx 4.9\pi\left[ \frac{\text{M}}{\text{M}_{\odot}}\si{\micro\second} \right]}$$

      \item 

        We now combine the fact that $d\tau^2=-ds^2$ and $E=\left( 1-\dfrac{2GM}{r} \right)\dfrac{dt}{d\tau}$ is constant to write:

        $$d\tau^2=\left( 1-\frac{2GM}{r} \right)\,dt^2-\left( 1-\frac{2GM}{r} \right)^{-1}\,dr^2$$

        This may also be written as:

        $$d\tau^2=-E^2\left( \frac{2GM}{r}-1 \right)^{-1}\,dt^2+\left( \frac{2GM}{r}-1 \right)^{-1}\,dr^2$$

        Because $2GM/r$ is positive within the Schwarzschild radius ($r<2GM$), we know that the term multiplied by $-E^2$ is subtracting a quantity from the $dr$ term. Thus, we see that the greatest lifetime may be written when $E\to0$ as:

        $$\boxed{d\tau^2=\left( \frac{2GM}{r}-1 \right)^{-1}\,dr^2}$$

    \end{enumerate}

  \item

    \begin{enumerate}

      \item 

      \item 

      \item 

    \end{enumerate}

\end{enumerate}

\end{document}

