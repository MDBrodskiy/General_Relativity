%%%%%%%%%%%%%%%%%%%%%%%%%%%%%%%%%%%%%%%%%%%%%%%%%%%%%%%%%%%%%%%%%%%%%%%%%%%%%%%%%%%%%%%%%%%%%%%%%%%%%%%%%%%%%%%%%%%%%%%%%%%%%%%%%%%%%%%%%%%%%%%%%%%%%%%%%%%%%%%%%%%
% Written By Michael Brodskiy
% Class: General Relativity and Cosmology
% Professor: J. Blazek
%%%%%%%%%%%%%%%%%%%%%%%%%%%%%%%%%%%%%%%%%%%%%%%%%%%%%%%%%%%%%%%%%%%%%%%%%%%%%%%%%%%%%%%%%%%%%%%%%%%%%%%%%%%%%%%%%%%%%%%%%%%%%%%%%%%%%%%%%%%%%%%%%%%%%%%%%%%%%%%%%%%

\documentclass[12pt]{article} 
\usepackage{alphalph}
\usepackage[utf8]{inputenc}
\usepackage[russian,english]{babel}
\usepackage{titling}
\usepackage{amsmath}
\usepackage{graphicx}
\usepackage{enumitem}
\usepackage{amssymb}
\usepackage[super]{nth}
\usepackage{everysel}
\usepackage{ragged2e}
\usepackage{geometry}
\usepackage{multicol}
\usepackage{fancyhdr}
\usepackage{cancel}
\usepackage{siunitx}
\usepackage{physics}
\usepackage{tikz}
\usepackage{mathdots}
\usepackage{yhmath}
\usepackage{cancel}
\usepackage{color}
\usepackage{array}
\usepackage{multirow}
\usepackage{gensymb}
\usepackage{tabularx}
\usepackage{extarrows}
\usepackage{booktabs}
\usepackage{lastpage}
\usetikzlibrary{fadings}
\usetikzlibrary{patterns}
\usetikzlibrary{shadows.blur}
\usetikzlibrary{shapes}

\geometry{top=1.0in,bottom=1.0in,left=1.0in,right=1.0in}
\newcommand{\subtitle}[1]{%
  \posttitle{%
    \par\end{center}
    \begin{center}\large#1\end{center}
    \vskip0.5em}%

}
\usepackage{hyperref}
\hypersetup{
colorlinks=true,
linkcolor=blue,
filecolor=magenta,      
urlcolor=blue,
citecolor=blue,
}


\title{Homework 2}
\date{\today}
\author{Michael Brodskiy\\ \small Professor: J. Blazek}

\begin{document}

\maketitle

\begin{enumerate}

  \item Per the affine connection, we may use the Christoffel Symbol to write:

    $$\Gamma^{\sigma}_{\mu\nu}=\frac{1}{2}g^{\sigma\rho}\left[ \partial_{\mu}g_{\nu\rho}+\partial_{\nu}g_{\rho\mu}-\partial_{\rho}g_{\mu\nu}  \right]$$

    We are given the metric for polar coordinates as:

    $$ds^2=dr^2+r^2\,d\theta^2$$

    Which gives us $g^{rr}=1$ and $g^{\theta\theta}=r^{-2}$.

    \begin{enumerate}

      \item 

        We can begin with what Carroll supplied:

        $$\boxed{\left\{\begin{array}{ll} \Gamma^r_{rr} &= 0\\\Gamma^r_{\theta\theta}&=-r\\\Gamma^r_{\theta r}=\Gamma^r_{r\theta}&= 0\\\Gamma^{\theta}_{rr}&= 0\\\Gamma^{\theta}_{r\theta}=\Gamma^{\theta}_{\theta r}&= (1/r)\\\Gamma^{\theta}_{\theta\theta}&= 0\end{array}}$$

      \item 

        We can continue to find the divergence of $V$ using the simplified formula:

        $$\nabla_{\mu}V^{\mu}=\frac{1}{\sqrt{|g|}}\partial_{\mu}\left( \sqrt{|g|}V^{\mu} \right)$$

        Which gives us:

        $$\boxed{\nabla\cdot\bold{V}=\frac{1}{\sqrt{|g|}}\partial_{r}\left( \sqrt{|g|}V^{r} \right)+\frac{1}{\sqrt{|g|}}\partial_{\theta}\left( \sqrt{|g|}V^{\theta} \right)}$$

        The gradient can be found as:

        $$\boxed{\nabla\bold{V}=\frac{\partial V}{\partial r}\bold{e}_r+\frac{1}{r}\frac{\partial V}{\partial \theta}\bold{e}_{\theta}}$$

      \item 

        In general, we may write:

        $$\frac{d^2x^{\mu}}{d\lambda^2}+\Gamma^{\mu}_{\rho\sigma}\frac{dx^{\rho}}{d\lambda}\frac{dx^{\sigma}}{d\lambda}=0$$

        From here, we can expand to:

        $$\frac{d^2x^{\rho}}{d\tau^2}+\frac{1}{2}g^{\rho\sigma}\left[  \partial_{\mu}g_{\nu\sigma}+\partial_{\nu}g_{\sigma\mu}-\partial_{\sigma}g_{\mu\nu} \right]\frac{dx^{\rho}}{d\lambda}\frac{dx^{\sigma}}{d\lambda}=0$$

        From (a), we know that $\Gamma^{\mu}_{\rho\sigma}$ is non-zero for only two combinations, $\Gamma^{r}_{\theta\theta}$ and $\Gamma^{\theta}_{r\theta}$. With this, we are able to construct two equations:

        $$\frac{d^2x^{r}}{d\lambda^2}-r\frac{d^2x^{\theta}}{d\lambda^2}=0$$
        $$\frac{d^2x^{\theta}}{d\lambda^2}+\frac{2}{r}\frac{dx^{r}}{d\lambda}\frac{dx^{\theta}}{d\lambda}=0$$

        This gives us the equations:

        $$\boxed{\frac{d^2x^{r}}{d\lambda^2}=r\frac{dx^{\theta}}{d\lambda}\frac{dx^{\theta}}{d\lambda}}$$
        $$\boxed{\frac{d^2x^{\theta}}{d\lambda^2}=-\frac{2}{r}\frac{dx^{r}}{d\lambda}\frac{dx^{\theta}}{d\lambda}}$$

      \item 

        Using the equation for a line, we may write:

        $$ax+by=c$$

        In polar, this would be equivalent to:

        $$ar\cos(\theta)+br\sin(\theta)=c$$

        We can differentiate to get:

        $$(a\cos(\theta)+b\sin(\theta))\,dr=(ar\sin(\theta)-br\cos(\theta))\,d\theta$$
        $$dr=\frac{(ar\sin(\theta)-br\cos(\theta))}{(a\cos(\theta)+b\sin(\theta))}\,d\theta$$

        Plugging this into our metric, we get:

        $$ds^2=\left(\frac{(ar\sin(\theta)-br\cos(\theta))}{(a\cos(\theta)+b\sin(\theta))}\,d\theta\right)^2+r^2\,d\theta$$

    \end{enumerate}

  \item

    \begin{enumerate}

      \item 

        We begin with the expression for the Christoffel Symbol:

          $$\Gamma^{\sigma}_{\mu\nu}=\frac{1}{2}g^{\sigma\rho}\left[ \partial_{\mu}g_{\nu\rho}+\partial_{\nu}g_{\rho\mu}-\partial_{\rho}g_{\mu\nu}  \right]$$

        Given the diagonal matrix, we know that these terms may be non-zero only for $\sigma=\rho$, which allows us to obtain:

          $$\Gamma^{t}_{\mu\nu}=\frac{1}{2}g^{tt}\left[ \partial_{\mu}g_{\nu t}+\partial_{\nu}g_{t\mu}-\partial_{t}g_{\mu\nu}  \right]$$
          $$\Gamma^{\theta}_{\mu\nu}=\frac{1}{2}g^{\theta\theta}\left[ \partial_{\mu}g_{\nu\theta}+\partial_{\nu}g_{\theta\mu}-\partial_{\theta}g_{\mu\nu}  \right]$$
          $$\Gamma^{\phi}_{\mu\nu}=\frac{1}{2}g^{\phi\phi}\left[ \partial_{\mu}g_{\nu\phi}+\partial_{\nu}g_{\phi\mu}-\partial_{\phi}g_{\mu\nu}  \right]$$

          Furthermore, we can expand the diagonal nature of the metric to write:

          $$\Gamma^{t}_{\mu\nu}=\frac{1}{2}g^{tt}\left[ \partial_{\mu}g_{tt}+\partial_{\nu}g_{tt}-\partial_{t}g_{\mu\nu}  \right]$$
          $$\Gamma^{\theta}_{\mu\nu}=\frac{1}{2}g^{\theta\theta}\left[ \partial_{\mu}g_{\theta\theta}+\partial_{\nu}g_{\theta\theta}-\partial_{\theta}g_{\mu\nu}  \right]$$
          $$\Gamma^{\phi}_{\mu\nu}=\frac{1}{2}g^{\phi\phi}\left[ \partial_{\mu}g_{\phi\phi}+\partial_{\nu}g_{\phi\phi}-\partial_{\phi}g_{\mu\nu}  \right]$$

          This can be simplified as:

          $$\Gamma^{t}_{\mu\nu}=-\frac{1}{2}\left[ \partial_{\mu}[-1]+\partial_{\nu}[-1]-\partial_{t}g_{\mu\nu}  \right]$$
          $$\Gamma^{\theta}_{\mu\nu}=\frac{1}{2}\left[ R^2 \right]\left[ \partial_{\mu}(R^2)+\partial_{\nu}(R^2)-\partial_{\theta}g_{\mu\nu}  \right]$$
          $$\Gamma^{\phi}_{\mu\nu}=\frac{1}{2}(R^2\sin^2(\theta))\left[ \partial_{\mu}(R^2\sin^2(\theta))+\partial_{\nu}(R^2\sin^2(\theta))-\partial_{\phi}g_{\mu\nu}  \right]$$

          By inspection, we can tell that the $\sigma=t$ case will always be zero, since the partials of constant become zero, and the partial with respect to time of any entry in the matrix is also zero (since they are not time-dependent). This leaves us with:

          $$\Gamma^{\theta}_{\mu\nu}=\frac{1}{2}\left[ R^2 \right]\left[ \partial_{\mu}(R^2)+\partial_{\nu}(R^2)-\partial_{\theta}g_{\mu\nu}  \right]$$
          $$\Gamma^{\phi}_{\mu\nu}=\frac{1}{2}(R^2\sin^2(\theta))\left[ \partial_{\mu}(R^2\sin^2(\theta))+\partial_{\nu}(R^2\sin^2(\theta))-\partial_{\phi}g_{\mu\nu}  \right]$$

          We begin by analyzing the $\sigma=\theta$ case:

          $$\Gamma^{\theta}_{tt}=\frac{1}{2}\left[ R^{-2} \right]\left[ \partial_{\mu}(R^2)+\partial_{\nu}(R^2)-\partial_{\theta}g_{\mu\nu}  \right]$$
          $$\Gamma^{\theta}_{t\theta}=\Gamma^{\theta}_{\theta t}=\frac{1}{2}\left[ R^{-2} \right]\left[ \partial_{\mu}(R^2)+\partial_{\nu}(R^2)-\partial_{\theta}g_{\mu\nu}  \right]$$
          $$\Gamma^{\theta}_{t\phi}=\Gamma^{\theta}_{\phi t}=\frac{1}{2}\left[ R^{-2} \right]\left[ \partial_{\mu}(R^2)+\partial_{\nu}(R^2)-\partial_{\theta}g_{\mu\nu}  \right]$$
          $$\Gamma^{\theta}_{\theta\theta}=\frac{1}{2}\left[ R^{-2} \right]\left[ \partial_{\mu}(R^2)+\partial_{\nu}(R^2)-\partial_{\theta}g_{\mu\nu}  \right]$$
          $$\Gamma^{\theta}_{\theta\phi}=\Gamma^{\theta}_{\phi\theta}=\frac{1}{2}\left[ R^{-2} \right]\left[ \partial_{\mu}(R^2)+\partial_{\nu}(R^2)-\partial_{\theta}g_{\mu\nu}  \right]$$
          $$\Gamma^{\theta}_{\phi\phi}=\frac{1}{2}\left[ R^{-2} \right]\left[ \partial_{\mu}(R^2)+\partial_{\nu}(R^2)-\partial_{\theta}g_{\mu\nu}  \right]$$

          Simplifying all of these, we get:

          $$\Gamma^{\theta}_{tt}=\frac{1}{2}\left[ R^{-2} \right]\left[ \partial_{\theta}(-1)\right]=0$$
          $$\Gamma^{\theta}_{t\theta}=\Gamma^{\theta}_{\theta t}=\frac{1}{2}\left[ R^{-2} \right]\left[ \partial_{t}(R^2)\right]=0$$
          $$\Gamma^{\theta}_{t\phi}=\Gamma^{\theta}_{\phi t}=\frac{1}{2}\left[ R^{-2} \right]\left[ 0\right]=0$$
          $$\Gamma^{\theta}_{\theta\theta}=\frac{1}{2}\left[ R^{-2} \right]\left[ \partial_{\theta}(R^2)+\partial_{\theta}(R^2)-\partial_{\theta}(R^2)  \right]=0$$
          $$\Gamma^{\theta}_{\theta\phi}=\Gamma^{\theta}_{\phi\theta}=\frac{1}{2}\left[ R^{-2} \right]\left[ \partial_{\phi}(R^2)\right]=0$$
          $$\Gamma^{\theta}_{\phi\phi}=\frac{1}{2}\left[ R^{-2} \right]\left[ \partial_{\phi}(R^2)+\partial_{\phi}(R^2)-\partial_{\theta}(R^2\sin^2(\theta))  \right]=-\sin(\theta)\cos(\theta)$$

          We continue with $\sigma=\phi$:

          $$\Gamma^{\phi}_{tt}=\frac{1}{2}\left[ [R\sin(\theta)]^{-2} \right]\left[ \partial_{\phi}(-1)\right]=0$$
          $$\Gamma^{\phi}_{t\theta}=\Gamma^{\phi}_{\theta t}=\frac{1}{2}\left[ [R\sin(\theta)]^{-2} \right]\left[ 0 \right]=0$$
          $$\Gamma^{\phi}_{t\phi}=\Gamma^{\phi}_{\phi t}=\frac{1}{2}\left[ [R\sin(\theta)]^{-2} \right]\left[ \partial_{t}(R^2\sin^2(\theta))\right]=0$$
          $$\Gamma^{\phi}_{\theta\theta}=\frac{1}{2}\left[ [R\sin(\theta)]^{-2} \right]\left[ -\partial_{\phi}(R^2)  \right]=0$$
          $$\Gamma^{\phi}_{\theta\phi}=\Gamma^{\phi}_{\phi\theta}=\frac{1}{2}\left[ [R\sin(\theta)]^{-2} \right]\left[ \partial_{\theta}(R^2\sin^2(\theta))\right]=\cot(\theta)$$
          $$\Gamma^{\phi}_{\phi\phi}=\frac{1}{2}\left[ [R\sin(\theta)]^{-2} \right]\left[ \partial_{\phi}(R^2\sin^2(\theta))+\partial_{\phi}(R^2\sin^2(\theta))-\partial_{\phi}(R^2\sin^2(\theta))  \right]=0$$

          Thus, the only non-zero terms are:

          $$\boxed{\Gamma_{\phi\phi}^{\phi}=-\sin(\theta)\cos(\theta)\quad\text{ and }\quad\Gamma^{\theta}_{\phi\theta}=\Gamma^{\theta}_{\theta\phi}=\cot(\theta)}$$

      \item 

        The geodesic equations give us:

        $$\frac{d^2x^{\mu}}{d\lambda^2}+\Gamma^{\mu}_{\rho\sigma}\frac{dx^{\rho}}{d\lambda}\frac{dx^{\sigma}}{d\lambda}=0$$

        Using the values obtained in (a), we may write:

        $$\frac{d^2x^{\phi}}{d\lambda^2}+\Gamma^{\phi}_{\phi\theta}\frac{dx^{\phi}}{d\lambda}\frac{dx^{\theta}}{d\lambda}=0$$
        $$\frac{d^2x^{\theta}}{d\lambda^2}+\Gamma^{\theta}_{\phi\phi}\frac{dx^{\phi}}{d\lambda}\frac{dx^{\phi}}{d\lambda}=0$$

        And we plug in known values to get:

        $$\boxed{\left\{\begin{array}{lll} \dfrac{d^2x^{\theta}}{d\lambda^2} &= &\sin(\theta)\cos(\theta)\dfrac{dx^{\phi}}{d\lambda}\dfrac{dx^{\phi}}{d\lambda}\\\\ \dfrac{d^2x^{\phi}}{d\lambda^2} &= &\cot(\theta)\dfrac{dx^{\phi}}{d\lambda}\dfrac{dx^{\theta}}{d\lambda}\end{array}}$$

    \end{enumerate}

  \item

    \begin{enumerate}

      \item 

        We may begin by using the values assigned to $x,y$, and $z$ to obtain expressions for $r,\theta$, and $\phi$:

        $$r=\frac{\cos(\lambda)}{\sin(\theta)\cos(\phi)}=\frac{\sin(\lambda)}{\sin(\theta)\sin(\phi)}$$

        Using this, we get:

        $$\frac{\cos(\lambda)}{\sin(\theta)\cos(\phi)}=\frac{\sin(\lambda)}{\sin(\theta)\sin(\phi)}$$
        $$\tan(\lambda)=\tan(\phi)$$
        $$\lambda=\phi$$

        Next, combining provided parametrization with the above, we get:

        $$r\cos(\theta)=\lambda$$
        $$r\sin(\theta)\sin(\phi)=\sin(\lambda)$$
        $$\frac{\lambda}{\cos(\theta)}=\frac{\sin(\lambda)}{\sin(\theta)\sin(\phi)}$$

        But, from what we obtained for $\lambda=\phi$, we can simplify:

        $$\lambda=\frac{\cos(\theta)}{\sin(\theta)}$$
        $$\theta=\cot^{-1}\left( \lambda \right)$$

        Finally, from our equation for $z$, we can say:

        $$r\sin(\theta)\cos(\phi)=\cos(\lambda)$$

        Which gets us:

        $$r=\frac{1}{\sin(\theta)}$$

        Using our trigonometric simplification rules, we may see:

        $$r=\sqrt{\lambda^2+1}$$

        Combining our findings, we may write the parametrization as:

        $$\boxed{\{r,\theta,\phi\}\to\{\sqrt{\lambda^2+1},\cot^{-1}(\lambda),\lambda\}}$$

      \item 

        The tangent vector to the curve may simply be found by taking the differential of the parametrizations. Let us begin by using the $x,y,z$ parametrization:

        $$\frac{d}{d\lambda}\left\{ x,y,z \right\}=\frac{d}{d\lambda}\left\{ \cos(\lambda),\sin(\lambda),\lambda \right\}$$
        $$\boxed{V^p=\left\{ -\sin(\lambda),\cos(\lambda),1 \right\}}$$

        Now, we use our result from (a):

        $$\frac{d}{d\lambda}\left\{ r,\theta,\phi \right\}=\frac{d}{d\lambda}\left\{ \sqrt{\lambda^2+1},\cot^{-1}(\lambda), \lambda\right\}$$
        $$\boxed{V^p=\left\{ \frac{\lambda}{\sqrt{\lambda^2+1}},-\frac{1}{\lambda^2+1}, 1 \right\}}$$

    \end{enumerate}

  \item

    \begin{enumerate}

      \item 

      \item 

    \end{enumerate}

  \item

\end{enumerate}

\end{document}

