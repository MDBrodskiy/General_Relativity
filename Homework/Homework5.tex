%%%%%%%%%%%%%%%%%%%%%%%%%%%%%%%%%%%%%%%%%%%%%%%%%%%%%%%%%%%%%%%%%%%%%%%%%%%%%%%%%%%%%%%%%%%%%%%%%%%%%%%%%%%%%%%%%%%%%%%%%%%%%%%%%%%%%%%%%%%%%%%%%%%%%%%%%%%%%%%%%%
% Written By Michael Brodskiy
% Class: General Relativity and Cosmology
% Professor: J. Blazek
%%%%%%%%%%%%%%%%%%%%%%%%%%%%%%%%%%%%%%%%%%%%%%%%%%%%%%%%%%%%%%%%%%%%%%%%%%%%%%%%%%%%%%%%%%%%%%%%%%%%%%%%%%%%%%%%%%%%%%%%%%%%%%%%%%%%%%%%%%%%%%%%%%%%%%%%%%%%%%%%%%%

\documentclass[12pt]{article} 
\usepackage{alphalph}
\usepackage[utf8]{inputenc}
\usepackage[russian,english]{babel}
\usepackage{titling}
\usepackage{amsmath}
\usepackage{graphicx}
\usepackage{enumitem}
\usepackage{amssymb}
\usepackage[super]{nth}
\usepackage{everysel}
\usepackage{ragged2e}
\usepackage{geometry}
\usepackage{multicol}
\usepackage{fancyhdr}
\usepackage{cancel}
\usepackage{siunitx}
\usepackage{physics}
\usepackage{tikz}
\usepackage{mathdots}
\usepackage{yhmath}
\usepackage{cancel}
\usepackage{color}
\usepackage{array}
\usepackage{multirow}
\usepackage{gensymb}
\usepackage{tabularx}
\usepackage{extarrows}
\usepackage{booktabs}
\usepackage{lastpage}
\usetikzlibrary{fadings}
\usetikzlibrary{patterns}
\usetikzlibrary{shadows.blur}
\usetikzlibrary{shapes}

\geometry{top=1.0in,bottom=1.0in,left=1.0in,right=1.0in}
\newcommand{\subtitle}[1]{%
  \posttitle{%
    \par\end{center}
    \begin{center}\large#1\end{center}
    \vskip0.5em}%

}
\usepackage{hyperref}
\hypersetup{
colorlinks=true,
linkcolor=blue,
filecolor=magenta,      
urlcolor=blue,
citecolor=blue,
}


\title{Homework 4}
\date{\today}
\author{Michael Brodskiy\\ \small Professor: J. Blazek}

\begin{document}

\maketitle

\begin{enumerate}

  \item Using our formula:

    $$\frac{\Delta T}{T}=\frac{v}{c}$$

    We may use $T=2.725[\si{K}]$, the given value of $\Delta T$, and the speed of light as $c=3\cdot10^5[\si{\km\per\second}]$ to get:

    $$v=\frac{3.36\cdot10^{-3}}{2.725}(3\cdot10^5)$$
    $$\boxed{v=369.908[\si{\kilo\meter\per\second}]}$$

    As a fraction of the speed of light, we may write this as:

    $$\boxed{v=.001233c}$$

  \item

    \begin{enumerate}

      \item 

      \item 

      \item 

    \end{enumerate}

  \item

    \begin{enumerate}

      \item Given one second since the Big Bang, we know:

        Integration leads us to find:

        $$t(a)=\frac{a^2}{2H_o\sqrt{\Omega_{r,0}}}$$

        We rearrange in terms of $a$ to write:

        $$a=\sqrt{2tH_o\sqrt{\Omega_{r,0}}}$$
        
        Using our known values, we get:

        $$a=\sqrt{2(1)\left(2.27\cdot10^{-18}\right)\sqrt{9\cdot10^{-5}}}$$
        $$\boxed{a=2.0753\cdot10^{-10}}$$

        From here, we know:

        $$a=\frac{1}{1+z}$$

        This gives us the redshift as:

        $$z=\frac{1}{a}-1$$
        $$z=\frac{1}{2.0753\cdot10^{-10}}-1$$
        $$\boxed{z=4.8186\cdot10^{9}}$$

        Using the scale factor, we know that the temperature is simply the current CMB over the factor:

        $$T(a)=a^{-1}T_o$$
        $$T(a)=\left( 2.0753\cdot10^{-10} \right)^{-1}(2.725[\si{K}])$$
        $$\boxed{T(a)=1.3131\cdot10^{10}[\si{K}]}$$

        Using the standard units, we know that the energy is proportional to the temperature; however, we need to adjust our units to electron-volts. This gives us:

        $$E\approx (1.3131\cdot10^{10})(8.617\cdot10^{-5})$$
        $$E=1.1315\cdot10^6[\si{eV}]$$
        $$\boxed{E=1.1315[\si{\mega eV}]}$$

        With standard units, mass is equal to the energy, which lets us simply convert units to say:

        $$\boxed{m=2.017\cdot10^{-30}\left[\si{\kilo\gram}\right]}$$

      \item We may begin by converting to Temperature (Kelvin):

        $$T=\frac{13\cdot10^{12}}{8.617\cdot10^{-5}}$$
        $$\boxed{T=1.5086\cdot10^{17}[\si{K}]}$$

        We can then find the scale factor:

        $$a=\frac{T_o}{T}$$
        $$a=\frac{2.725}{1.5086\cdot10^{17}}$$
        $$\boxed{a=1.8063\cdot10^{-17}}$$

        Using the time formula obtained in (a), we get:

        $$t=\frac{a^2}{2H_o\sqrt{\Omega_{r,0}}}$$
        $$t=\frac{(1.8063\cdot10^{-17})^2}{2(2.27\cdot10^{-18})\sqrt{9\cdot10^{-5}}}$$
        $$\boxed{t=7.58\cdot10^{-15}[\si{\second}]}$$

        The redshift may be found as:

        $$z=\frac{1}{a}-1$$
        $$z=\frac{1}{1.8063\cdot10^{-17}}-1$$
        $$\boxed{z=5.536\cdot10^{16}}$$

        The mass can finally be obtained as:

        $$m=\left( 1.78266\cdot10^{-30} \right)\left( 13\cdot10^{6} \right)$$
        $$\boxed{m=2.317\cdot10^{-23}[\si{\kilo\gram}]}$$

      \item With this mass, we may begin by calculating the energy:

        $$10^{-9}[\si{\gram}]=10^{-12}[\si{\kilo\gram}]$$
        $$E=\frac{10^{-12}}{1.78266\cdot10^{-30}}$$
        $$\boxed{E=5.609\cdot10^{11}[\si{TeV}]}$$

        The temperature then becomes:

        $$T=\frac{5.609\cdot10^{23}}{8.617\cdot10^{-5}}$$
        $$\boxed{T=6.509\cdot10^{27}[\si{K}]}$$

        We may obtain the scale factor:

        $$a=\frac{2.725}{6.509\cdot10^{27}}$$
        $$\boxed{a=4.1864\cdot10^{-28}}$$

        This gives us the redshift as:

        $$z=\frac{1}{a}-1$$
        $$z=\frac{1}{4.1864\cdot10^{-28}}-1$$
        $$\boxed{z=2.389\cdot10^{27}}$$
        
        And finally, we may find the time since the Big Bang as:

        $$t=\frac{(4.1864\cdot10^{-28})^2}{2(2.27\cdot10^{-18})\sqrt{9\cdot10^{-5}}}$$
        $$\boxed{t=4.069\cdot10^{-36}[\si{\second}]}$$

      \item In the case of temperature being given, we need to consider radiation and matter. This gives us:

        $$\int \frac{1}{H_o\sqrt{\Omega_{r}a^{-4}+(\Omega_m-\Omega_r)a^{-3}}}\,da=t$$

        Using a numerical solver gets us:

        $$t=\frac{2}{(\Omega_m-\Omega_r)^2}\left[ \frac{1}{3}\left( \Omega_r+\left( \Omega_m-\Omega_r \right)a \right)^{\frac{3}{2}}-\Omega_r\sqrt{\Omega_r+\left( \Omega_m-\Omega_r \right)a} \right]$$

        We may find the scale factor as:

        $$a=\frac{2.725}{3000}$$
        $$\boxed{a=.000908}$$

        We then plug this into the above to get time:

        $$t=\frac{2}{(.31-9\cdot10^{-5})^2}\left[ \frac{1}{3}\left( 9\cdot10^{-5}+\left( .31-9\cdot10^{-5} \right)(.000908) \right)^{\frac{3}{2}}-$$
        $$\left( 9\cdot10^{-5} \right)\sqrt{9\cdot10^{-5}+\left( .31-9\cdot10^{-5} \right)(.000908)} \right]$$
        $$\boxed{t=.000014[\si{\second}]}$$

        We can then proceed to find the rest of the values as usual:

        $$z=\frac{1}{a}-1$$
        $$z=\frac{1}{.000908}-1$$
        $$\boxed{z=1099.92}$$

        We convert the temperature to energy:

        $$E=3000(8.617\dot10^{-5})$$
        $$\boxed{E=.25851}$$

        And finally we get the mass as:

        $$m=(.25851\cdot10^{-6})(1.78266\cdot10^{-30})$$
        $$\boxed{m=4.6084\cdot10^{-37}[\si{\kilo\gram}]}$$

    \end{enumerate}

  \item

    \begin{enumerate}

      \item 

      \item 

    \end{enumerate}

  \item

    \begin{enumerate}
        
      \item 

      \item 

      \item 

    \end{enumerate}

\end{enumerate}

\end{document}

