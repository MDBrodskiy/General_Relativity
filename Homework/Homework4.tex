%%%%%%%%%%%%%%%%%%%%%%%%%%%%%%%%%%%%%%%%%%%%%%%%%%%%%%%%%%%%%%%%%%%%%%%%%%%%%%%%%%%%%%%%%%%%%%%%%%%%%%%%%%%%%%%%%%%%%%%%%%%%%%%%%%%%%%%%%%%%%%%%%%%%%%%%%%%%%%%%%%%
% Written By Michael Brodskiy
% Class: General Relativity and Cosmology
% Professor: J. Blazek
%%%%%%%%%%%%%%%%%%%%%%%%%%%%%%%%%%%%%%%%%%%%%%%%%%%%%%%%%%%%%%%%%%%%%%%%%%%%%%%%%%%%%%%%%%%%%%%%%%%%%%%%%%%%%%%%%%%%%%%%%%%%%%%%%%%%%%%%%%%%%%%%%%%%%%%%%%%%%%%%%%%

\documentclass[12pt]{article} 
\usepackage{alphalph}
\usepackage[utf8]{inputenc}
\usepackage[russian,english]{babel}
\usepackage{titling}
\usepackage{amsmath}
\usepackage{graphicx}
\usepackage{enumitem}
\usepackage{amssymb}
\usepackage[super]{nth}
\usepackage{everysel}
\usepackage{ragged2e}
\usepackage{geometry}
\usepackage{multicol}
\usepackage{fancyhdr}
\usepackage{cancel}
\usepackage{siunitx}
\usepackage{physics}
\usepackage{tikz}
\usepackage{mathdots}
\usepackage{yhmath}
\usepackage{cancel}
\usepackage{color}
\usepackage{array}
\usepackage{multirow}
\usepackage{gensymb}
\usepackage{tabularx}
\usepackage{extarrows}
\usepackage{booktabs}
\usepackage{lastpage}
\usetikzlibrary{fadings}
\usetikzlibrary{patterns}
\usetikzlibrary{shadows.blur}
\usetikzlibrary{shapes}

\geometry{top=1.0in,bottom=1.0in,left=1.0in,right=1.0in}
\newcommand{\subtitle}[1]{%
  \posttitle{%
    \par\end{center}
    \begin{center}\large#1\end{center}
    \vskip0.5em}%

}
\usepackage{hyperref}
\hypersetup{
colorlinks=true,
linkcolor=blue,
filecolor=magenta,      
urlcolor=blue,
citecolor=blue,
}


\title{Homework 4}
\date{\today}
\author{Michael Brodskiy\\ \small Professor: J. Blazek}

\begin{document}

\maketitle

\begin{enumerate}

  \item

    We may express the comoving distance function as (note that the numerator should contain $c$, but since we eliminated the speed of light, we instead get the below equation):

    $$\chi(z)=\int_0^z \frac{dz'}{H(z')}$$

    Furthermore, the Hubble function may be expressed as a function of parameter densities and the redshift, $z$, such that:

    $$H(z)=H_o\sqrt{\Omega_r(1+z)^4+\Omega_m(1+z)^3+\Omega_{\kappa}(1+z)^2+\Omega_{\Lambda}}$$

    Combining the two equations, we may obtain:

    $$\boxed{\chi(z)=\int_0^z \frac{dz'}{H_o\sqrt{\Omega_r(1+z')^4+\Omega_m(1+z')^3+\Omega_{\kappa}(1+z')^2+\Omega_{\Lambda}}}}$$

    Such a formula is logical since it contains the various densities of the universe, as well as being an expression in terms of redshift.

  \item First and foremost, we know that we may write:

    $$\frac{\dot{a}}{a}=H_o\left[ \Omega_ma^{-3}+\Omega_ra^{-4}+\Omega_{\Lambda}+\Omega_{\kappa}a^{-2} \right]^{\frac{1}{2}}$$

    \begin{enumerate}

      \item For the Einstein-de Sitter universe, we use the equation above to get:

        $$\frac{da}{dt}=a^{-.5}H_o$$

        From this, we can solve:

        $$\int a^{.5}\,da=\int H_o\,dt$$
        $$\frac{2}{3}a^{1.5}=H_ot$$

        This gives us the scale factor as:

        $$\boxed{a(t)=\frac{3}{2}\left( H_ot \right)^{\frac{2}{3}}}$$

        To find the age of the universe, we take $a\to1$ and $t\to t_o$ to get:

        $$\boxed{t_o=\frac{2}{3H_o}}$$

        Finally, we find the comoving horizon distance as (note that restoring $c$ would return this value to the numerator):

        $$\chi_{hor}=\int_0^1 \frac{da}{a^{.5}H_o}$$
        $$\chi_{hor}=\frac{2\sqrt{a}}{H_o}\Big_0^1$$
        $$\boxed{\chi_{hor}=\frac{2}{H_o}}$$

      \item 

        For a radiation-dominated universe, we get:

        $$\frac{da}{dt}=\frac{H_o}{a}$$
        $$\int a\,da=\int H_o\,dt$$
        $$\frac{1}{2}a^2=H_ot$$
        $$\boxed{a(t)=\sqrt{4H_ot}}$$

        From here, we can find the age as:

        $$1=\sqrt{4H_ot_o}$$
        $$\boxed{t_o=\frac{1}{4H_o}}$$

        Finally, we find the horizon distance:

        $$\chi_{hor}=\int_0^1\frac{da}{a^2H_o}$$
        $$\chi_{hor}=-\frac{1}{aH_o}\Big_0^1$$
        $$\chi_{hor}=-\frac{1}{H_o}$$

      \item 

      \item 

      \item 

      \item 

    \end{enumerate}

  \item Given that we are trying to find the point at which the two equal, we obtain:

    $$\Omega_{m,o}=a^3\Omega_{\Lambda,o}$$
    $$a_{m,\Lambda}=\sqrt[3]{\frac{\Omega_{m,o}}{\Omega_{\Lambda,o}}}$$

    This gives us:

    $$a_{m,\Lambda}=\sqrt[3]{\frac{.31}{.69}}$$
    $$\boxed{a_{m,\Lambda}=.7659}$$

    From here, we know that:

    $$z=\frac{1}{a_{m,\lambda}}-1$$
    $$z=\frac{1}{.7659}-1$$
    $$\boxed{z=.3057}$$

    Redshift is .3057 when densities in matter and $\Lambda$ are equivalent.  Proceeding to find the matter-radiation equality point, we get:

    $$\Omega_{r,o}=a\Omega_{m,o}$$
    $$a_{m,r}=\frac{\Omega_{r,o}}{\Omega_{m,o}}$$

    This gives us:

    $$a=\frac{9\cdot10^{-5}}{.31}$$
    $$\boxed{a_{m,r}=2.9032\cdot10^{-4}}$$

    From here, we get:

    $$z=(2.9032\cdot10^{-4})^{-1}-1$$
    $$z=3444.5-1$$
    $$\boxed{z=3443.5}$$

    Redshift is 3443.5 when densities in matter and radiation are equivalent.

  \item Let us begin by defining the Hubble radius as:

    $$R_H=\frac{1}{aH(a)}$$

    We differentiate both signs to get:

    $$\dot{R}_H=\frac{d}{dt}\left[ \frac{1}{aH(a)} \right]$$

    We know that the Hubble parameter is defined by:

    $$H(a)=\frac{\dot{a}}{a}$$

    And thus, we get:

    $$aH(a)=\dot{a}$$
    $$\frac{d}{dt}[aH(a)]=\ddot{a}$$

    Combining this with the differential equation above gives us:

    $$\dot{R}_H=-\frac{\ddot{a}}{[aH(a)]^2}$$

    We may observe that, because $\ddot{a}$ is strictly positive, and the denominator can not be negative since it is real and squared, the derivative, due to the negative sign, must always be less than zero. This can be express as:

    $$\boxed{\dot{R}_H=-\frac{\ddot{a}}{[aH(a)]^2}<0}$$

    And, therefore, \underline{the Hubble radius is strictly decreasing} for periods of accelerating expansion. This indicates that \underline{the distance over which light may act is contracting}, since the universe expands at an increasing rate and regions of spacetime move away from one another at a velocity greater than that of light. Using the second Friedmann equation, we know:

    $$\frac{\ddot{a}}{a}=-\frac{4\pi G}{3}(\rho +3p)$$

    With the relationship that $p=\omega\rho$, we may observe that the smallest (in magnitude) such $\omega$ is:

    $$3\omega\rho<-\rho$$
    $$\boxed{\omega<-\frac{1}{3}}$$

  \item

  \item We know that:

    $$\rho_{crit}=\frac{3H_o^2}{8\pi G}$$

    In standard units, $H_o=70\left[ \si{\kilo\meter\over\second\over\mega pc} \right]$, becomes:

    $$H_o=2.27\cdot10^{-18}\left[ \si{1\over\second} \right]$$

    Combining this and other known equations in our expression above, we get:

    $$\rho_{crit}=\frac{3(2.27\cdot10^{-18})^2}{8\pi(6.674\cdot10^{-11})}$$
    $$\boxed{\rho_{crit}=8.55\cdot10^{-27}\left[ \si{\kilo\gram\over\meter\cubed} \right]}$$

    We know that a solar mass may be expressed as $1[M_{\odot}]\approx1.989\cdot10^{30}[\si{\kilo\gram}]$. Furthermore, astronomical units may be written as: $1[au]\approx 1.496\cdot10^{11}[\si{\meter}]$. This gives us:

    $$\rho_{crit}=\left( \frac{8.55\cdot 3.3481}{1.989} \right)\cdot10^{-27}\cdot10^{33}\cdot10^{-30}$$
    $$\boxed{\rho_{crit}=1.4392\cdot10^{-23}\left[ \si{M_{\odot}\over au\cubed} \right]}$$

    Now, we find the value in terms of proton mass. We know that a proton mass is: $m_+\approx1.67\cdot10^{-27}[\si{\kilo\gram}]$. As such, we get:

    $$\rho_{crit}=\left( \frac{8.55}{1.67} \right)\cdot10^{27}\cdot10^{-27}$$
    $$\boxed{\rho_{crit}=5.1198\left[ \si{m_p\over\meter\cubed} \right]}$$

    Note that the densities are small, indicating that space itself is quite empty.

  \item We may begin by finding the mass in the universe, in terms of a density function (assuming spherical shape). This gives us:

    $$m(r)=\int_0^r 4\pi r^2\rho(r)\,dr$$

    We assume that the density may expressed in some terms such that $\rho(r)=r^{\alpha}$. This gives us:

    $$m(r)=4\pi\int_0^r r^{2+\alpha}\rho(r)\,dr$$
    $$m(r)=\frac{4\pi r^{3+\alpha}}{3+\alpha}$$

    Now, we want to relate gravitation to velocity. This can be done by setting the equations for gravitational force and centripetal acceleration equal to each other. This yields:

    $$\frac{Gm(r)m_2}{r^2}=\frac{m_2v(r)^2}{r}$$
    $$m(r)=\frac{rv(r)^2}{G}$$

    It is given that $v(r)$ does not actually depend on $r$ in a flat galaxy, and we can thus conclude that the mass is proportional to $r^1$. Thus, we may say:

    $$r^{3+\alpha}=r^1$$
    $$3+\alpha=1$$
    $$\boxed{\alpha=-2}$$

    As such, we know that the density profile must be of some form such that:

    $$\boxed{\rho(r)\propto r^{-2}}$$

\end{enumerate}

\end{document}

