%%%%%%%%%%%%%%%%%%%%%%%%%%%%%%%%%%%%%%%%%%%%%%%%%%%%%%%%%%%%%%%%%%%%%%%%%%%%%%%%%%%%%%%%%%%%%%%%%%%%%%%%%%%%%%%%%%%%%%%%%%%%%%%%%%%%%%%%%%%%%%%%%%%%%%%%%%%%%%%%%%
% Written By Michael Brodskiy
% Class: General Relativity and Cosmology
% Professor: J. Blazek
%%%%%%%%%%%%%%%%%%%%%%%%%%%%%%%%%%%%%%%%%%%%%%%%%%%%%%%%%%%%%%%%%%%%%%%%%%%%%%%%%%%%%%%%%%%%%%%%%%%%%%%%%%%%%%%%%%%%%%%%%%%%%%%%%%%%%%%%%%%%%%%%%%%%%%%%%%%%%%%%%%%

\include{Includes.tex}

\title{Homework 6}
\date{\today}
\author{Michael Brodskiy\\ \small Professor: J. Blazek}

\begin{document}

\maketitle

\begin{enumerate}

  \item

    \begin{enumerate}

      \item First and foremost, we can eliminate the pressure contribution, since we are assuming a case in which we treat matter as dust. Dust is assumed to have no pressure, since velocities are non-relativistic, while mass is most of the energy density.

      \item We can begin by decomposing the components of the fluid equations into ``perturbation form'' as follows:

        $$\rho=\rho_o+\delta \rho$$
        $$\vec{v}=\vec{v}_o+\delta \vec{v}$$
        $$\Phi=\Phi_o+\delta \Phi$$

        Using standard convention, we take $\delta \Phi\to\Phi$. This allows us to rewrite the equations as

        $$\left\{\begin{array}{ll} \frac{D(\vec{v}_o+\delta\vec{v})}{Dt}&=-\nabla\Phi\\ \frac{D(\rho_o+\delta \rho)}{Dt} &= -(\rho_o+\delta \rho)\nabla\cdot(\vec{v}_o+\delta\vec{v})\\ \nabla^2\Phi&=4\pi G(\rho_o+\delta \rho) \end{array}$$

        And finally we linearize (removing zeroth-order terms):

        $$\boxed{\left\{\begin{array}{ll} \left[\dfrac{\partial}{\partial t}+(\vec{v}_o+\delta\vec{v})\cdot\nabla\right]\vec{v}_o+\left[ \dfrac{\partial}{\partial t}+\vec{v}_o\cdot\nabla \right]\delta\vec{v}&=-\nabla\Phi\\\\ \dfrac{\partial\bar{\rho}}{\partial t}+\left( \dfrac{\partial}{\partial t}+\vec{v}_o\cdot\nabla \right)\bar{\rho}\delta &= -\bar{\rho}(\nabla\cdot(\vec{v}_o+\delta\vec{v})+\delta\nabla\cdot\vec{v}_o)\\\\ \nabla^2\Phi&=4\pi G\bar{\rho}\delta \end{array}}$$

        \item Incorporating the background velocity ($\vec{v}_o=H\vec{x}$), we may write:

          $$\boxed{\left\{\begin{array}{ll} H\delta\vec{v}+\frac{d(\delta\vec{v}}{dt}&= -\nabla \Phi\\\frac{d(\delta)}{dt}&=\nabla\cdot\delta\vec{v}\end{array}}$$

      \item To transition to comoving coordinates, we may use the following relationships:

        $$\vec{x}=a\vec{r}$$

        The peculiar velocity:

        $$\delta\vec{v}=a\vec{u}$$

        And the gradient:

        $$\nabla_c=\frac{1}{a}\nabla$$

        Incorporating this into the above, we get:

        $$\left\{\begin{array}{ll} a\dfrac{d\vec{u}}{dt}+2aH\vec{u}&=-\nabla_c\Phi\\\\ \dfrac{d(\delta)}{dt} &= -\nabla_c\cdot(\vec{u})\\\\ \nabla^2_c\Phi&=4\pi G\bar{\rho}a^2\delta \end{array}$$

        We can then simplify using dot notation to get the equations in terms of comoving coordinates:

        $$\boxed{\left\{\begin{array}{ll} \dot{\vec{u}}+2H\vec{u}&=-a^{-2}\nabla_c\Phi\\ \dot{\delta} &= -\nabla_c\cdot\vec{u}\\ \nabla^2_c\Phi&=4\pi G\bar{\rho}a^2\delta \end{array}}$$

        We may see that we have found the damping term proportional to twice the Hubble expansion.

        \item Taking the divergence of the first equation, we get:

          $$\nabla_c\cdot\dot{\vec{u}}+2H\nabla_c\cdot\vec{u}=-\frac{1}{a^2}\nabla_c^2\Phi$$

          We may observe that this can be combined with the third equation to get:

          $$\nabla_c\cdot\dot{\vec{u}}+2H\nabla_c\cdot\vec{u}=-4\pi G\bar{\rho}\delta$$

          We then take the time derivative of the second equation to write:

          $$\ddot{\delta}=-\nabla_c\cdot\dot{\vec{u}}$$
          $$\nabla_c\cdot\dot{\vec{u}}=-\ddot{\delta}$$

          We then plug this and the undifferentiated form of the second equation into the first and third combined equation to write:

          $$-\ddot{\delta}-2H\dot{\delta}=-4\pi G\bar{\rho}\delta$$

          We distribute the negative sign to get:

          $$\boxed{\ddot{\delta}+2H\dot{\delta}=4\pi G\bar{\rho}\delta}$$

      \item We know that the mean matter density can be written as:

        $$\bar{\rho}(a)=\rho_{crit}\Omega_m(a)$$

        Furthermore, we know that the critical density is:

        $$\rho_{crit}=\frac{3H_O^2}{8\pi G}$$

        Combining this with part (e), we get:

        $$\boxed{\ddot{\delta}+2H\dot{\delta}=\frac{3H^2\Omega_m(a)\delta}{2}}$$

      \item 

        \begin{itemize}

          \item Matter Domination

            In this case, we may see that:

            $$\ddot{\delta}+2H\dot{\delta}=\frac{3}{2}H^2\delta$$

            We can rewrite this in terms of $t$ to get:

            $$\ddot{\delta}+\frac{4}{3t}\dot{\delta}-\frac{2}{3t^2}\delta=0$$

            We can see that the only cosmologically relevant solution is when $\delta\propto t^{2/3}$, and that, in this case, since $a\propto t^{2/3}$, we can conclude:

            $$\boxed{\delta\propto a}$$

          \item Radiation Domination

            We may observe that $\Omega_m=0$, which gives us:

            $$\ddot{\delta}+2H\dot{\delta}=0$$

            We can rewrite in terms of $t$ to get:

            $$\ddot{\delta}+\frac{1}{t}\dot{\delta}=0$$

            We may observe that since the first-order time derivative is proportional to the inverse of $t$, $\delta\propto\ln(t)$. As such, we may conclude:

            $$\boxed{\delta\propto\frac{3}{2}\ln(a)}$$

          \item $\Lambda$ Domination

            Similarly to radiation, we get:

            $$\ddot{\delta}+2H\dot{\delta}=0$$

            However, we know that in $\Lambda$ domination, $H$ is constant. Thus, we can determine that there are two solutions for $\delta$, and only one that is cosmologically meaningful:

            $$\delta = c\quad\text{ or }\delta\propto e^{-2Ht}$$

            Given that the second would imply that $\delta\propto a^{-2}$, the only relevant solution is, for some constant $c$:

            $$\boxed{\delta=c}$$

            And thus, this term is constant for $\Lambda$ domination.

        \end{itemize}

      \item We use our solutions from (g) and the following formula:

        $$\nabla^2_c\Phi=4\pi G\bar{\rho}a^2\delta$$

        \begin{itemize}

          \item Matter Domination

            We may observe that, during this period, $\Phi$ remains constant, since:

            $$\Phi\propto \frac{\delta}{a}$$
            $$\Phi\propto \frac{a}{a}$$
            $$\boxed{\Phi\propto c}$$

          \item Radiation and $\Lambda$ Domination

            We may observe that $\Phi$ decays, since, for radiation, we see:

            $$\boxed{\Phi\propto \frac{\ln(a)}{a}}$$

            And for $\Lambda$ domination:

            $$\boxed{\Phi\propto \frac{c}{a}}$$

            Note that decay occurs much faster for the case of $\Lambda$ domination.

        \end{itemize}

      \item Based on the results from (h), we may conclude that, in a matter-dominated region, the photon would remain at the same energy, since the gravitational potential doesn't change; however, the photon would gain energy (experience the ISW effect) in a radiation or $\Lambda$ dominated universe, since the gravitational potential would decay, meaning that the decrease in potential would be gained by the photon. Note that, in an underdense region, the opposite would occur.

    \end{enumerate}

  \item We first use the Born approximation to find the perpendicular acceleration:

    $$a_{\perp}=\frac{GM}{r^2}\cos(\theta)$$

    This acceleration results in the deflection of the light ray. From here, we may define the angle $\hat{\alpha}$ as the integral of the perpendicular acceleration. We first define:

    $$r^2=\varepsilon^2+z^2$$

    And then:

    $$\cos(\theta)=\frac{\varepsilon}{\sqrt{\varepsilon^2+z^2}}$$

    This allows us to write:

    $$\hat{\alpha}=\int_{-\infty}^{\infty} \frac{GM\varepsilon}{(\varepsilon^2+z^2)^{\frac{3}{2}}}\,dz$$

    We integrate to obtain:

    $$\boxed{\hat{\alpha}=\frac{2GM}{\varepsilon}}$$

    We may observe that the General Relativity case predicts a deflection angle that is twice that of the Newtonian prediction.

  \item First and foremost, we know that gravitational lensing results in two effects: first, the magnification of luminosity, which results in observed luminosity $\mu L$ with magnification factor $\mu$ and intrinsic luminosity $L$; second, the apparent area of the sky is magnified by the same factor $\mu$, which results in the density of galaxies being decreased by a factor $\mu^{-1}$. Given that $n(L)$ corresponds to the number density of galaxies, we may write:

    $$n(L)\to n_{app}(L_{app})$$

    From here, we can define an ``unlensed'' function as:

    $$n_{unl}(L)\propto L^{-\alpha}$$

    We can integrate to find the quantity of observable galaxies:

    $$N_{unl}(L)\propto\int_{L_o}^{\infty}L^{-\alpha}\,dL$$
    $$N_{unl}(L)\propto L_o^{1-\alpha}$$

    We then substitute into the observed case:

    $$n_{app}(L)\propto \frac{1}{\mu}\left( \frac{\mu}{L_o} \right)^{\alpha-1}$$

    We may simplify to get:

    $$n_{app}(L)\propto \left( \mu^{\alpha-2}N_{unl} \right)$$

    Thus, we may observe that, when $\mu >2$, more galaxies will be observed. When $\mu<2$, less galaxies are observed since the magnification factor will be less than 1. When $\mu=2$, there is no difference in the quantity of observed galaxies.

  \item We may begin by calculating the luminosity distance as:

    $$d_L=\chi(1+z)$$

    For a $\Lambda$CDM universe, we know that $\chi$ may be obtained using:

    $$\chi=\int_0^z \frac{dz'}{H_o\sqrt{.31(1+z')^3+.69}}$$
    
    Since the redshift is given, we get:

    $$\chi=\int_0^{.01} \frac{dz'}{H_o\sqrt{.31(1+z')^3+.69}}$$

    Entering this into a numerical solver, we may obtain:

    $$\chi=\frac{.009977}{H_o}$$

    Which ultimately gives us:

    $$d_L=\frac{.009977(1+.01)c}{70}$$
    $$\boxed{d_L\approx 42.856[\si{\mega pc}]}$$

    This gives us a time of:

    $$t=4.4055\cdot10^{15}[\si{\second}]$$

    Using the time delay given, we may write:

    $$\frac{c_{GW}}{c}\approx \frac{1.7}{4.4055\cdot10^{15}}$$
    $$\frac{c_{GW}}{c}\approx 3.86\cdot10^{-16}$$

    Thus, we see the upper limit is, roughly, on the order of $10^{-16}$

\end{enumerate}

\end{document}

